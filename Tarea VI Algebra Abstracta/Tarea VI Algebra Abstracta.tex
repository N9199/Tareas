\documentclass[11pt]{article}
    \usepackage[spanish]{babel}
    \usepackage[utf8]{inputenc}
    \usepackage[margin=1in]{geometry}          
    \usepackage{graphicx}
    \usepackage{amsthm, amsmath, amssymb}
    \usepackage{mathtools}
    \usepackage{setspace}\onehalfspacing
    \usepackage[loose,nice]{units} 
    \usepackage{enumitem}
    \usepackage{hyperref}
    \hypersetup{
        colorlinks,
        citecolor=black,
        filecolor=black,
        linkcolor=black,
        urlcolor=black
    }
    
    \setcounter{secnumdepth}{0}
    
    \title{Tarea VI}
    \author{Nicholas Mc-Donnell}
    \date{2do semestre 2017}
    
    \renewcommand{\thesection}{}
    \renewcommand{\thesubsection}{}

    \renewcommand{\d}[1]{\ensuremath{\operatorname{d}\!{#1}}}
    \renewcommand{\vec}[1]{\mathbf{#1}}
    \newcommand{\set}[1]{\mathbb{#1}}
    \newcommand{\func}[5]{#1:#2\xrightarrow[#5]{#4}#3}
    \newcommand{\contr}{\rightarrow\leftarrow}
    
    \DeclareMathOperator{\Ima}{Im}
    
    \newtheorem{thm}{Teorema}[section]
    \newtheorem{lem}[thm]{Lema}
    \newtheorem{prop}[thm]{Proposición}
    \newtheorem*{cor}{Corolario}
    
    \theoremstyle{definition}
    \newtheorem{defn}{Definición}[section]
    \newtheorem{obs}{Observación}[section]
    \newtheorem{ejm}[thm]{Ejemplo:}

    \pagenumbering{gobble}

    \begin{document}
        \maketitle
        \newpage
        
        \clearpage\null\newpage

        \pagenumbering{arabic}
        \tableofcontents
        \newpage

        \clearpage\null\newpage
    \section{2. Dominios de factorización única, Dominios de Ideales Principales y Dominios Euclidianos}
    \subsection{1}
    Prove or disprove the following.
    \begin{enumerate}[label=\textbf{(\alph*)}]
        \item The polynomial ring $\set{R}[x,y]$ in two variables is a Euclidean domain.

        \item The ring $\set{Z}[x]$ is a principal ideal domain.
    \end{enumerate}
    \begin{proof}
        \
        \begin{enumerate}[label=\textbf{(\alph*)}]
            \item Tomamos el ideal $(x,y)$ y notamos que no es un ideal principal, por lo que concluimos que $\set{R}[x,y]$ no es un dominio Euclidiano.
    
            \item Recordamos que si un anillo $R$ es DIP, entonces $R/(a)$ es un cuerpo, si tomamos $\set{Z}[x]/(x)\simeq \set{Z}$ vemos que no es cuerpo, por lo que $\set{Z}[x]$ no es DIP.
        \end{enumerate}
    \end{proof}
    
    \subsection{3}
    Give an example showing that division with remainder need not be unique in a Euclidean domain.
    \begin{proof}
        Tomamos los enteros de Gauss con los siguientes elementos: $b/a=x, b=1+i, a=2$
        \[2*0+1+i=1+i\]
        \[2*1-1+i=1+i\]
        \[\sigma(1+i)=\sigma(1-i)\]
        Por lo que no necesariamente es única.
    \end{proof}

    \subsection{9}
    \begin{enumerate}[label=\textbf{(\alph*)}]
        \item Prove that $2,3,1\pm\sqrt{-5}$ are irreducible elements of the ring $R=\set{Z}[\sqrt{-5}]$ and that the units of this ring are $\pm 1$.

        \item Prove that the existence of factorization is true for this ring.
    \end{enumerate}
    \begin{enumerate}[label=\textbf{(\alph*)}]
        \item \begin{proof}
            Comenzamos por demostrar que las únicas unidades de este anillo son $\pm1$. Asumiremos que existe alguna unidad $u$.
            \[\therefore (u)=(1)\]
            \[\implies \exists r\in R: ur=1\]
            Notamos que $\bar{u}\bar{r}=1$. ($\overline{a+b\sqrt{-5}}=a-b\sqrt{-5}$)
            \[\implies (u\bar{u})=(1)\]
            \[u\bar{u}\in\set{Z}^+\]
            \[\therefore u\bar{u}=1\vee u\bar{u}>1\]
            Si $u\bar{u}>1$
            \[(u\bar{u})^2>u\bar{u}\]
            \[\implies (u\bar{u})\neq(1)\]
            \[\implies (u)\neq(1)\]
            \[\contr\]
            Si $u\bar{u}=1$, con $u=a+b\sqrt{-5}$.
            \[u\bar{u}=a^2+5b^2=1\]
            \[\implies b=0\quad a^2=1\]
            \[\implies u=\pm1\]
            Que es lo que queríamos demostrar. Para demostrar la irreductibilidad de $2,3,1\pm\sqrt{-5}$ definiremos una función $\func{\sigma}{R\setminus\{0\}}{\set{Z}}{}{}$.
            \[\sigma(a+b\sqrt{-5})=a^2+5b^2\]
            Sean $u,v\in R$
            \[\sigma(uv)=\sigma((a+b\sqrt{-5})(c+d\sqrt{-5}))=\sigma(ac-5bd+(ad+bc)\sqrt{-5})=(ac-5bd)^2+5(ad+bc)^2\]
            \[\sigma(u)\sigma(v)=(a^2+5b^2)(c^2+5d^2)=a^2c^2+25b^2d^2+5b^2c^2+5a^2d^2\]
            \[\sigma(u)\sigma(v)=a^2c^2+25b^2d^2+5b^2c^2+5a^2d^2+10abcd-10abcd=(ac-5bd)^2+5(ad+bc)^2\]
            \[\implies\sigma(uv)=\sigma(u)\sigma(v)\]
            Notamos que $u\in R$ unidad $\iff\sigma(u)=1$, sean $a,b\in R: a\mid b$.
            \[\therefore b=ar\quad r\in R\]
            \[\implies \sigma(b)=\sigma(a)\sigma(r)\]
            \[\implies \sigma(a)\mid\sigma(b)\]
            Asumamos que $2,3,1\pm\sqrt{-5}$ no son irreducibles.
            \[\therefore \exists a\in R: a\mid 2\]
            \[\implies\sigma(a)\mid\sigma(2)\]
            \[\sigma(a)\mid 4\]
            Pero notamos que el único divisor no trivial es $2$, pero $\forall x\in R:\sigma(r)\neq 2$. Similarmente $\forall x\in R:\sigma(x)\neq 3$, vemos que $\sigma(3)=9,\sigma(1\pm\sqrt{-5)}=-4$, por lo que $\nexists x\in R\setminus\{1,2\}:\sigma(x)\mid \sigma(2)$, y análogamente se ven los otros casos, pero esto es una contradicción. Por lo que $2,3,1\pm\sqrt{-5}$ son irreducibles.
        \end{proof}

        \item \begin{proof}
            Para simplificar la demostración, sin perder generalidad, no se tomara los asociados en cuenta.\\
            Sean $a,b\in R:a\mid b$, y sea $\sigma$ la función definida anteriormente.
            \[\therefore \sigma(a)\mid\sigma(b)\]
            Sabemos que $1<\sigma(a)<\sigma(b)$ o $\sigma(a)=\sigma(b)$, lo segundo implica que son elementos asociados, por lo que no es un caso a considerar. El primero se divide en dos casos, $a$ irreducible, o $\exists c\in R: c\mid a$, por lo mismo que antes:
            \[1<\sigma(c)<\sigma(a)\]
            Entonces notamos que $b$ solo puede tener finitos divisores (sin considerar unidades y elementos asociados), ya que la secuencia de divisores $\sigma(a_n)$ es estrictamente decreciente, pero es mayor a $1$.
        \end{proof}
    \end{enumerate}

    \subsection{13}
    If $a,b$ are integers and if $a$ divides $b$ in the ring of Gauss integers, then $a$ divides $b$ in $\set{Z}$
    \begin{proof}
        Usando $\func{\sigma}{\set{Z}[i]\setminus\{0\}}{\set{Z}}{}{}$ tal que:
        \[\sigma(a+bi)=a^2+b^2\]
        La cual sabemos que cumple lo mismo que la función denotada en el ejercicio anterior. Ahora, sean $a,b\in R:a\mid b$ enteros.
        \[\therefore \sigma(a)\mid\sigma(b)\]
        \[\sigma(a)=a^2,\sigma(b)=b^2\]
        \[\implies a^2\mid b^2\]
        \[\implies a\mid b\]
        Que es lo que queríamos demostrar.
    \end{proof}

    \section{3. Lema de Gauss}
    \subsection{1}
    Let $a,b$ be elements of a field $F$, with $a \neq 0$. Prove that the polynomial $f(x)\in F[x]$ is irreducible if and only if $f(ax+b)$ is irreducible.

    \subsection{3}
    Let $f$ be an irreducible polynomial in $\set{C}[x,y]$, and let $g$ be another polynomial. Prove that if the variety of zeros of $g$ in $\set{C}^2$ contains the variety of zeros of $f$, then $f$ divides $g$.

    \subsection{9}
    Prove that the kernel of the homomorphism $\set{Z}[x]\rightarrow \set{R}$ sending $x\mapsto 1+\sqrt{2}$ is a principal ideal, and find a generator for this ideal.

    \section{4. Factorización explicita de polinomios}
    \subsection{1}
    Prove that the following polynomials are irreducible in $\set{Q}[x]$.
    \begin{enumerate}[label=\textbf{(\alph*)}]
        \item $x^2+27x+213$

        \item $x^3+6x+12$

        \item $8x^3-6x+1$

        \item $x^3+6x^2+7$

        \item $x^5-3x^4+3$
    \end{enumerate}

    \subsection{3}
    Factor $x^3+x+1$ in $\set{F}_p[x]$, when $p=2,3,5$.

    \subsection{7}
    Factor the following polynomials into irreducible factors in $\set{Q}[x]$.
    \begin{enumerate}[label=\textbf{(\alph*)}]
        \item $x^3-3x-2$

        \item $x^3-3x+2$

        \item $x^9-6x^6+9x^3-3$
    \end{enumerate}
    \end{document}