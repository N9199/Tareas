\documentclass[11pt]{article}
    \usepackage[spanish]{babel}
    \usepackage[utf8]{inputenc}
    \usepackage[margin=1in]{geometry}          
    \usepackage{graphicx}
    \usepackage{amsthm, amsmath, amssymb}
    \usepackage{mathtools}
    \usepackage{setspace}\onehalfspacing
    \usepackage[loose,nice]{units} 
    \usepackage{enumitem}
    \usepackage{hyperref}
    \hypersetup{
        colorlinks,
        citecolor=black,
        filecolor=black,
        linkcolor=black,
        urlcolor=black
    }
    
    \setcounter{secnumdepth}{0}
    
    \title{Tarea VI}
    \author{Nicholas Mc-Donnell}
    \date{2do semestre 2017}
    
    \renewcommand{\thesection}{}
    \renewcommand{\thesubsection}{}

    \renewcommand{\d}[1]{\ensuremath{\operatorname{d}\!{#1}}}
    \renewcommand{\vec}[1]{\mathbf{#1}}
    \newcommand{\set}[1]{\mathbb{#1}}
    \newcommand{\func}[5]{#1:#2\xrightarrow[#5]{#4}#3}
    \newcommand{\contr}{\rightarrow\leftarrow}
    
    \DeclareMathOperator{\Ima}{Im}
    
    \newtheorem{thm}{Teorema}[section]
    \newtheorem{lem}[thm]{Lema}
    \newtheorem{prop}[thm]{Proposición}
    \newtheorem*{cor}{Corolario}
    
    \theoremstyle{definition}
    \newtheorem{defn}{Definición}[section]
    \newtheorem{obs}{Observación}[section]
    \newtheorem{ejm}[thm]{Ejemplo:}

    \pagenumbering{gobble}

    \begin{document}
        \maketitle
        \newpage
        
        \clearpage\null\newpage

        \pagenumbering{arabic}
        \tableofcontents
        \newpage

        \clearpage\null\newpage
    \section{2. Dominios de factorización única, Dominios de Ideales Principales y Dominios Euclidianos}
    \subsection{1}
    Prove or disprove the following.
    \begin{enumerate}[label=\textbf{(\alph*)}]
        \item The polynomial ring $\set{R}[x,y]$ in two variables is a Euclidean domain.

        \item The ring $\set{Z}[x]$ is a principal ideal domain.
    \end{enumerate}

    \subsection{3}
    Give an example showing that division with remainder need not be unique in a Euclidean domain.

    \subsection{9}
    \begin{enumerate}[label=\textbf{(\alph*)}]
        \item Prove that $2,3,1\pm\sqrt{-5}$ are irreducible elements of the ring $R=\set{Z}[\sqrt{-5}]$ and that the units of this ring are $\pm 1$.

        \item Prove that the existence of factorization is true for this ring.
    \end{enumerate}

    \subsection{13}
    If $a,b$ are integers and if $a$ divides $b$ in the ring of Gauss integers, then $a$ divides $b$ in $\set{Z}$

    \section{3. Lema de Gauss}
    \subsection{1}
    Let $a,b$ be elements of a field $F$, with $a \neq 0$. Prove that the polynomial $f(x)\in F[x]$ is irreducible if and only if $f(ax+b)$ is irreducible.

    \subsection{3}
    Let $f$ be an irreducible polynomial in $\set{C}[x,y]$, and let $g$ be another polynomial. Prove that if the variety of zeros of $g$ in $\set{C}^2$ contains the variety of zeros of $f$, then $f$ divides $g$.

    \subsection{9}
    Prove that the kernel of the homomorphism $\set{Z}[x]\rightarrow \set{R}$ sending $x\mapsto 1+\sqrt{2}$ is a principal ideal, and find a generator for this ideal.

    \section{4. Factorización explicita de polinomios}
    \subsection{1}
    Prove that the following polynomials are irreducible in $\set{Q}[x]$.
    \begin{enumerate}[label=\textbf{(\alph*)}]
        \item $x^2+27x+213$

        \item $x^3+6x+12$

        \item $8x^3-6x+1$

        \item $x^3+6x^2+7$

        \item $x^5-3x^4+3$
    \end{enumerate}

    \subsection{3}
    Factor $x^3+x+1$ in $\set{F}_p[x]$, when $p=2,3,5$.

    \subsection{7}
    Factor the following polynomials into irreducible factors in $\set{Q}[x]$.
    \begin{enumerate}[label=\textbf{(\alph*)}]
        \item $x^3-3x-2$

        \item $x^3-3x+2$

        \item $x^9-6x^6+9x^3-3$
    \end{enumerate}

    \section{5. Primos en el anillo de Enteros de Gauss}
    \subsection{1}
    Prove that every Gauss prime divides exactly one integer prime.

    \subsection{3}
    Factor the following into Gauss primes.
    \begin{enumerate}[label=\textbf{(\alph*)}]
        \item $1-3i$

        \item $10$

        \item $6+9i$
    \end{enumerate}

    \subsection{7}
    Describe the residue ring $\set{Z}[i]/(p)$ in each case.
    \begin{enumerate}[label=\textbf{(\alph*)}]
        \item $p=2$

        \item $p\equiv 1\quad(\text{modulo} 4)$

        \item $p\equiv 3 \quad(\text{modulo} 4)$
    \end{enumerate}
    \end{document}