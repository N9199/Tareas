\documentclass[11pt]{article}
\usepackage[spanish]{babel}
\usepackage[margin=1in]{geometry}          
\usepackage{graphicx}
\usepackage{amsthm, amsmath, amssymb}
\usepackage{multirow}
\usepackage{setspace}\onehalfspacing
\usepackage[loose,nice]{units}

\title{Tarea Algebra Lineal}
\author{Nicholas Mc-Donnell}

\begin{document}
\maketitle

Sea $\mathbb{F}^\infty=\{(x_n)_{n\in\mathbb{N}}|x_n\in\mathbb{F}$ $\forall n\in\mathbb{N}\}$\\
Sea  $\delta_k\in\mathbb{F}^\infty$ con $k\in\mathbb{N}$, donde\\
\[
\delta_{k,n}= \left\{\begin{array}{@{}lr@{}}
        \multirow{1}{*}{0,} & \text{si }n\neq k\\
        1, & \text{si }n=k
        \end{array}\right\}
\]

Demostrar que:
\[
\sideset{}{<\delta_k>}\bigoplus_{k\in\mathbb{N}}\neq\mathbb{F}^\infty
\]

Para esto, notemos que:
\[
\{x_n=1,\forall n\in\mathbb{N}\}\notin\sideset{}{<\delta_k>}\bigoplus_{k\in\mathbb{N}}
\]
Esto se debe a como esta definido $\oplus$, esto es, la suma directa de finitas cosas, lo que implica que solo se puede haber una cantidad finita de elementos distintos de $0$ en cada elemento del subespacio vectorial que definimos. Por ende, $(x_n)_{n\in\mathbb{N}}$ no pertenece a este subespacio vectorial.
\end{document}