\documentclass[12pt,letterpaper]{article}
\usepackage[utf8]{inputenc}
\usepackage[spanish]{babel}
\usepackage[margin=1in]{geometry}
\usepackage{graphicx}
\usepackage{amsthm, amsmath, amssymb}
\usepackage{mathtools}
\usepackage{setspace}\onehalfspacing
\usepackage[loose,nice]{units}
\usepackage{enumitem}\setlist[enumerate]{label= (\alph*)}
\usepackage{hyperref}
\usepackage{titling}

\hypersetup{
	colorlinks,
	citecolor=black,
	filecolor=black,
	linkcolor=black,
	urlcolor=black
}

\renewcommand{\d}[1]{\ensuremath{\operatorname{d}\!{#1}}}
\renewcommand{\vec}[1]{\mathbf{#1}}
\newcommand{\set}[1]{\mathbb{#1}}
\newcommand{\func}[5]{#1:#2\xrightarrow[#5]{#4}#3}
\newcommand{\contr}{\rightarrow\leftarrow}
\newcommand{\floor}[1]{\left\lfloor#1\right\rfloor}
\newcommand{\ceil}[1]{\left\lceil#1\right\rceil}
\newcommand{\abs}[1]{\left|#1\right|}
\newcommand{\paren}[1]{\left(#1\right)}
\newcommand{\mcm}{\text{mcm }}
\newcommand{\BigO}[2][]{O_{#1}\paren{#2}}
\newcommand{\ds}{\displaystyle}
\newcommand{\cis}{\text{cis }}

\renewcommand{\thesection}{}
\renewcommand{\thesubsection}{}

\DeclareMathOperator{\Ima}{Im}
\DeclareMathOperator{\rad}{rad}

\newenvironment{prob}[1]{
	{\large\raggedleft\textbf{Problema #1:}}\addcontentsline{toc}{section}{Problema #1}\par\addvspace{-\parskip}\noindent
}{}

\newenvironment{sol}[1]{\par\medskip
	\noindent \textbf{Solución problema #1:} \rmfamily}{\begin{flushright}
		$\blacksquare$
	\end{flushright}
}

\title{Tarea 2}
\author{Nicholas Mc-Donnell}
\date{2019/04/11}

\pagenumbering{gobble}

\begin{document}
\begin{minipage}{2.5cm}
    \includegraphics[width=2cm]{../../figures/logo1.jpg}
\end{minipage}
\begin{minipage}{13cm}
    \begin{flushleft}
        \raggedright{
            \noindent
            {\sc Pontificia Universidad Católica de Chile\\
                Facultad de Matemáticas\\
                Departamento de Matemática} \smallskip \\
            Primer Semestre de 2019\\
        }
    \end{flushleft}
\end{minipage}

\vspace{2ex}
{\Large \centerline{\bf \thetitle}}
{\large \centerline{Introducción a la Geometría Algebraica --- MAT 2335}}
{\normalsize \centerline{ Fecha de Entrega: \thedate}}
\vfill

\begin{flushright}
    {\large\theauthor}
\end{flushright}
\newpage
\normalsize
\pagenumbering{arabic}
\tableofcontents
\newpage

\section*{Notas}
En esta tarea se usará la notación \(\overline{a}=(a_1,...,a_n)\)\\

\begin{prob}{1.25}
    \begin{enumerate}
        \item Muestre que \(V(y-x^2)\subset\set{A}_\set{C}^2\) es irreducible; en efecto, \(I(V(y-x^2))=I(y-x^2)\)
        \item Separe \(V(y^4-x^2,y^4-x^2y^2+xy^2-x^3)\subset\set{A}_\set{C}^2\) en componentes irreducibles.
    \end{enumerate}
\end{prob}

\begin{sol}{1.25}
    \begin{enumerate}
        \item Se puede notar que si \((y-x^2)\) es un ideal primo entonces \(V(y-x^2)\) es irreducible, y si \(p(x,y)=y-x^2\) es un polinomio irreducible el ideal generado es primo. Usando criterio de Eisenstein (con \(y\) sobre \(\set{C}[y][x]\)) esto se tiene que \(p\) es irreducible, por lo que el ideal es primo y \(V(p)\) es irreducible.
        \item Se factoriza cada polinomio
              \begin{align*}
                  y^4-x^2             & =(y^2-x)(y^2+x)     \\
                  y^4-x^2y^2+xy^2-x^3 & =-(x-y)(x+y)(x+y^2)
              \end{align*}
              Con lo que se puede notar que ambos tienen el polinomio \(x+y^2\) en común, por lo que \(V(y^4-x^2,y^4-x^2y^2+xy^2-x^3)=V(y^2+x)\cup V(y^2-x,(x-y)(x+y))\). Se puede notar que \(V(y^2-x,(x-y)(x+y))\) son tres puntos \((0,0),(1,1),(1,-1)\), y \((0,0)\in V(y^2+x)\), luego \(V(x-1,y-1)=\{(1,1)\},V(x-1,y+1)=\{(1,-1)\}\), por lo que se puede separar el conjunto algebraico en los conjuntos mostrados.
    \end{enumerate}
\end{sol}

\begin{prob}{1.29}
    Muestre que \(\set{A}_k^n\) es irreducible si \(k\) es infinito.
\end{prob}

\begin{sol}{1.29}
    Sea \(\set{A}_k^n\) reducible, luego existen \(V_i\) tal que \(\set{A}_k^n=\bigcup_{i=1}^nV_i\), donde cada \(V_i\) es de la forma \(V(f_{i,1},...,f_{i,n_i})\) con los \(f_{i,j}\) no cero, notamos que \(V_i\subseteq V(f_{i,1})\), por lo que \(\set{A}_k^n\subseteq\bigcup_{i=1}^nV(f_{i,1})\), pero se sabe que \(\bigcup_{i=1}^nV(f_{i,1})=V(\prod_{i=1}^nf_{i,1})\), con lo cual se ve que \(V(\prod_{i=1}^nf_{i,1})=\set{A}_k^n\), pero se sabe\footnote{Por tarea anterior} que si \(k\) es infinito el único polinomio que es cero para todos los valores es el cero, por lo que \(k\) tiene que ser finito.
\end{sol}

\begin{prob}{1.30}
    Sea \(k=\set{R}\)
    \begin{enumerate}
        \item Muestre que \(I(V(x^2+y^2+1))=(1)\)
        \item Demuestre que todo subconjunto algebraico de \(\set{A}_\set{R}^2\) es igual a \(V(F)\) para algún \(F\in\set{R}[x,y]\)
    \end{enumerate}
\end{prob}

\begin{sol}{1.30}
    \begin{enumerate}
        \item Ya que \(\set{R}\) es un cuerpo ordenado se sabe que \(x^2\geq0\quad\forall x\in\set{R}\), dado esto se nota que \(x^2+y^2+1\geq1>0\quad\forall x,y\in\set{R}\), por lo que \(V(x^2+y^2+1)=\emptyset\). Con esto se puede concluir lo que queríamos.
        \item Sea \(A\) un subconjunto algebraico en \(\set{A}_\set{R}^2\), luego \(A\) puede ser una de cuatro cosas, un conjunto finito de puntos, una curva, el plano o la unión de las anteriores, se nota que el último caso se reduce a los otros, ya que si la unión se puede ver como los ceros de la multiplicación de los polinomios correspondientes. En el primer caso, donde \(S=\{(a_1,b_1),...,(a_n,b_n)\}\), el polinomio \(\prod_{i=1}^n((x-a_i)^2+(y-a_i)^2)\) cumple lo pedido. El segundo caso, por definición una curva es un polinomio, por lo que cumple lo pedido. Y el último caso, el polinomio \(0\) cumple lo que se quiere.
    \end{enumerate}
\end{sol}

\begin{prob}{1.31}
    \begin{enumerate}
        \item Encuentre los componentes irreducibles de \(V(y^2-xy-x^2y+x^3)\) en \(\set{A}_\set{R}^2\) y también en \(\set{A}_\set{C}^2\)
        \item Haga lo mismo para \(V(y^2-x(x^2-1))\), y para \(V(x^3+x-x^2y-y)\)%(x^2+1)(x-y)
    \end{enumerate}
\end{prob}

\begin{sol}{1.31}
    \begin{enumerate}
        \item La siguiente factorización se puede ver \(y^2-xy-x^2y+x^3=(x-y)(x^2-y)\), por lo que \(V(y^2-xy-x^2y+x^3)=V(x-y)\cup V(x^2-y)\), ambos son irreducibles por criterio de Eisenstein (usando \(y\) en \(k[y][x]\)), por lo que es la factorización en conjuntos irreducibles. Esto es independiente de \(k\), por lo que es la misma factorización para \(\set{C}\) y para \(\set{R}\).
        \item Se nota que \(y^2-x(x^2-1)\) es irreducible por criterio de Eisenstein (usando \(x\) en \(k[x][y]\)), por lo que \(V(y^2-x(x^2-1))\) es irreducible. Para el otro conjunto algebraico, se ve que \(x^3+x-x^2y-y=(x^2+1)(x-y)\), luego en \(\set{C}\) \(x^2+1=(x-i)(x+i)\), con lo que se tiene que \(V(x^3+x-x^2y-y)\) se separa en \(V(x-y)\) y \(V(x^2+1)\) en \(\set{A}_\set{R}^2\), y en \(V(x-y)\), \(V(x-i)\) y \(V(x+i)\) en \(\set{A}_\set{C}^2\).
    \end{enumerate}
\end{sol}

\begin{prob}{1.33}
    \begin{enumerate}
        \item Separe \(V(x^2+y^2-1,x^2-z^2-1)\subset\set{A}_\set{C}^3\) en componentes irreducibles.
        \item Sea \(V=\{(t,t^2,t^3)\in\set{A}_\set{C}^3:t\in\set{C}\}\). Encuentre \(I(V)\), y demuestre que es irreducible.
    \end{enumerate}
\end{prob}

\begin{sol}{1.33}
    \begin{enumerate}
        \item Sea \(V=V(x^2+y^2-1,x^2-z^2-1)\), notamos que \(V=V(x^2-z^2-1,y^2+z^2)=V(x^2-z^2-1,z-iy)\cup V(x^2-z^2-1,z+iy)\), los cuales se denominan \(V_1,V_2\) correspondientemente. Se sabe que si \(C[x,y,z]/I(V_i)\) es dominio, \(I(V_i)\) es primo, y \(V_i\) es irreducible. Notamos que \((\set{C}[x,y,z]/(z+iy))/(x^2-y^2-1)\simeq\set{C}[x,y,z]/I(V_2)\), por lo que cocientando en orden, claramente \(\set{C}[x,y,z]/(z+iy)\simeq\set{C}[x,y]\). Ahora, se cocienta \(\set{C}[x,y]/(x^2-y^2-1)\), se nota que si \(x^2-y^2-1\) es irreducible, \(\set{C}[x,y]/(x^2-y^2-1)\) es dominio. Se asume que existen \(p,q\) de grado \(1\) tal que \(x^2-y^2-1=p\cdot q\):
              \begin{align*}
                  p(x,y) & = ax+by+c \\
                  q(x,y) & = dx+ey+f
              \end{align*}
              Se ven las siguientes relaciones:
              \begin{align*}
                   & ad=1  & (ea+bd)=0 \\
                   & be=-1 & (cd+af)=0 \\
                   & cf=-1 & (bf+ec)=0
              \end{align*}
              Con lo que se nota que ninguno es cero, luego se trabajan un poco las expresiones y se consigue:
              \begin{align*}
                  a   & =d^{-1}  \\
                  b   & =-e^{-1} \\
                  e^2 & =d^2
              \end{align*}
              Lo que nos da dos casos
              \begin{itemize}
                  \item[Caso \(e=d\):] Se nota que entonces \(e(a+b)=0\), por lo que \(a=-b\), se suma \(cd+af=0\) con \(bf+ec=0\) y se consigue \(cd+ce=0\), pero eso es \(2cd=0\), una contradicción.
                  \item[Caso \(e=-d\):] Se nota que entonces \(e(a-b)=0\), siguiendo la demostración anterior, pero restando, se llega a lo mismo, otra contradicción.
              \end{itemize}
              Por lo que se tiene lo pedido.
        \item Se nota que \(V(y-x^2,z-x^3)=V\), luego \(I(V)=(y-x^2,z-x^3)\), se sabe que si \(C[x,y,z]/I(V)\) es un dominio, entonces \(I(V)\) es primo. Sea \(\varphi\) morfismo natural, luego claramente \(\Ima\varphi=\set{C}[x]\) y \(I(V)\subseteq\ker\varphi\), sea \(p\in\ker\varphi\), luego \(p(x,x^2,x^3)=0\), pero eso significaría que \(p\in I(V)\), por lo que \(I(V)=\ker\varphi\). Luego se sabe que \(\set{C}[x]\) es euclidiano, por lo que particularmente es un dominio. Entonces \(I(V)\) es primo y \(V\) es irreducible.
    \end{enumerate}
\end{sol}

\begin{prob}{1.37}
    Sea \(k\) un cuerpo cualquiera, \(F\in k[x]\) un polinomio de grado \(n>0\). Muestre que los residuos \(\overline{1},\overline{x},...,\overline{x}^{n-1}\) forman una base de \(k[x]/(F)\) sobre \(k\).
\end{prob}

\begin{sol}{1.37}
    Sea \(F(x)=\sum_{i=0}^na_ix^i\) y \(\varphi\) es morfismo natural de \(k[x]\) a \(k[x]/(F)\), luego se quiere que \(k[x]/(F)\) sea un espacio vectorial sobre \(k\) tal que \(\dim k[x]/(F) = n\). Viendo \(F\) en \(k[x]/(F)\) se puede notar que \(\overline{x}^n=-\varphi(a_n)^{-1}\sum_{i=0}^{n-1}\varphi(a_i)\overline{x}^i\), por lo que \(\overline{x}^n\) se puede escribir en la base propuesta. Claramente \(\overline{x}^{n+k}=-\varphi(a_n)^{-1}\sum_{i=0}^{n-1}\varphi(a_i)\overline{x}^{i+k}\), por lo que para \(j\geq n\) \(\overline{x}^j\) se puede escribir en base a los \(\overline{x}^i\) donde \(n-j\leq i<j\), dado esto se ve que como \(\overline{x}^n\) se puede escribir en la base propuesta, si \(j\geq n\) \(\overline{x}^j\) se puede escribir en la base propuesta. Con esto se tiene que para todo \(G\in k[x]\) \(\overline{G}\) se puede escribir en la base propuesta, como \(\varphi\) es sobreyectivo, todo elemento en \(k[x]/(F)\) se puede escribir en la base propuesta, cumpliendo lo pedido.
\end{sol}

\begin{prob}{1.45}
    Sea \(R\) un subanillo de \(S\), \(S\) un subanillo de \(T\).
    \begin{enumerate}
        \item Si \(S=\sum Rv_i,T=\sum Sw_j\), muestre que \(T=\sum Rv_iw_j\).
        \item Si \(S=R[v_1,...,v_n],T=S[w_1,...,w_m]\), muestre que \(T=R[v_1,...,v_n,w_1,...,w_m]\).
        \item Si \(R,S,T\) son cuerpos, y \(S=R(v_1,...,v_n),T=S(w_1,...,w_m)\), demuestre que \(T=R(v_1,...,v_n,w_1,...,w_m)\).
    \end{enumerate}
\end{prob}

\begin{sol}{1.45}
    \begin{enumerate}
        \item Sea \(u\in T\), luego \(u=\sum_{j=1}^n\alpha_jw_j\) donde los \(\alpha_j\in S\), como están en \(S\), se pueden escribir de la siguiente forma \(\alpha_j=\sum_{i=1}^m\beta_{i,j}v_i\), juntando ambas cosas: \(u=\sum_{j=1}^n\sum_{i=1}^m\beta_{i,j}w_jv_i\) donde \(\beta_{i,j}\in R\), por ende \(T=\sum Rv_iw_j\).
        \item Asumamos que \(T\subsetneq R[v_1,...,v_n,w_1,...,w_m]\), eso implica que existe \(a\in T\) tal que \(a\notin R[v_1,...,v_n,w_1,...,w_m]\). Como \(a\in T\), \(a\) se puede escribir en base a los \(w_i\) y elementos en \(S\), pero cada elemento en \(S\) se puede escribir en base a los \(v_j\) y elementos en \(R\), por lo que \(a\) se puede escribir en base a los \(w_i\), los \(v_j\) y elementos en \(R\), pero eso significaría que \(a\in R[v_1,...,v_n,w_1,...,w_m]\), una contradicción.
        \item Es análogo a la (b), usando la definición de extensión de cuerpo en vez de la de anillo.
    \end{enumerate}
\end{sol}

\begin{prob}{1.49}
    Sea \(k\) un cuerpo, \(L=k(x)\) el cuerpo de funciones racionales en una variable sobre \(k\).
    \begin{enumerate}
        \item Muestre que todo elemento de \(L\) que es integral sobre \(k[x]\) ya esta en \(k[x]\). (Hint: Si \(z^n+a_1z^{n-1}+...=0\), tome \(z=F/G\), con \(F,G\) coprimos. Entonces \(F^n+a_1F^{n-1}+...=0\), por lo que \(G\) divide a \(F\).)
        \item Muestre que no hay un elemento no cero \(F\in k[x]\) tal que para todo \(z\in L\), \(F^nz\) es integral sobre \(k[x]\) para algún \(n>0\).
    \end{enumerate}
\end{prob}

\begin{sol}{1.49}
    \begin{enumerate}
        \item Sea \(a\in k(x)\) tal que \(a\notin k[x]\) y \(a\) es integral sobre \(k[x]\), luego sea \(p(y)\) el polinomio mónico en \(k[x][y]\) tal que \(p(a)=0\). Se puede escribir \(a=F/G\) donde \(F,G\in k[x]\) y son coprimos. Luego \(p(F/G)=(F/G)^n+\sum_{i=0}^{n-1}\alpha_i\paren{\frac{F}{G}}^i=0\), se toma \(G^n\cdot p(F/G)=F^n+\sum_{i=0}^{n-1}\alpha_i F^iG^{n-i}=0\), se puede escribir \(F^n=-\sum_{i=0}^{n-1}\alpha_i F^iG^{n-i}\), y se nota que \(G\mid F^n\) por lo que \(G\mid F\), pero \(G,F\) son coprimos, una contradicción, con eso tenemos que no existe \(a\in k(x)\setminus k[x]\) que sea integral.
        \item Por (a), se sabe que si \(a\in L\) integral sobre \(k[x]\) entonces esta \(a\in k[x]\). Ahora, se asume que existe \(F\) y \(n>0\) tal que para todo \(z\in L\) \(F^nz\) es integral, específicamente entonces cumple para \(1/F^{n+1}\), entonces \(F^n/F^{n+1}=1/F\) es integral, pero entonces \(1/F\in k[x]\), lo que claramente es una contradicción.
    \end{enumerate}
\end{sol}

\begin{prob}{1.51}
    Sea \(k\) un cuerpo, \(F\in k[x]\) un polinomio irreducible mónico de grado \(n>0\).
    \begin{enumerate}
        \item Muestre que \(L=k[x]/(F)\) es un cuerpo, y si \(a\) es el residuo de \(x\) en \(L\), entonces \(F(a)=0\).
        \item Suponga que \(L'\) es una extensión de cuerpo de \(k\), \(y\in L'\) tal que \(F(y)=0\). Demuestre que el homorfismo de \(k[x]\) a \(L'\) que toma \(x\) a \(y\), induce un isomorfismo de \(L\) con \(k(y)\).
        \item Con \(L'\), \(y\) como en (b), suponga que \(G\in k[x]\) y \(G(y)=0\). Muestre que \(F\) divide a \(G\).
        \item Muestre que \(F=(x-a)F_1,F_1\in L[x]\)
    \end{enumerate}
\end{prob}

\begin{sol}{1.51}
    \begin{enumerate}
        \item Como \(F\) es irreducible, entonces \((F)\) es maximal, y se sabe que un anillo cocientado por un ideal maximal es un cuerpo. Sea \(a=\overline{x}\), luego \(\overline{F(x)}=0\), se puede operar \(\overline{F(x)}\) de la siguiente forma
              \begin{align*}
                  \overline{F(x)} & =\overline{\sum_{i=0}^nb_ix^i}                 \\
                                  & =\sum_{i=0}^n\overline{b_ix^i}                 \\
                                  & =\sum_{i=0}^n\overline{b_i}\overline{x^i}      \\
                                  & =\sum_{i=0}^n\overline{b_i}\cdot\overline{x}^i
              \end{align*}
              Ya que \(b_i\in k\), entonces \(\overline{b_i}=b_i\), por lo que \(\overline{F(x)}=\sum_{i=0}^nb_ia^i=0\), con lo que \(a\) es una raíz.
        \item Se denota \(\varphi\) el homorfismo, de \(k[x]\) a \(L'\) tal que \(x\mapsto y\), por primer teorema de isomorfismo \(\Ima\varphi\simeq k[x]/\ker\varphi\), se nota que si \(\Ima\varphi=k(y)\) y \(\ker\varphi=(F)\) tenemos lo pedido. Trivialmente \(\Ima\varphi\subseteq k(y)\), luego se sabe por una parte del 1.37 que la base de \(k(y)\) sobre \(k\) tiene a lo más el grado de \(F\) elementos que la generan, y más específicamente, todo elemento \(\alpha\) de \(k(y)\) se puede escribir de la siguiente forma:
              \[
                  \alpha = \sum_{i=0}^n\alpha_iy^i
              \]
              Donde \(n=\dim k(y)\), luego sea \(P(x)=\sum_{i=0}^n\alpha_ix^i\), claramente \(\varphi(P)=\alpha\), por lo que \(\alpha\in\Ima\varphi\), por lo que \(\Ima\varphi=k(y)\). Se puede ver que \(\ker\varphi\subseteq (F)\), sea \(P\in (F)\), luego \(P=\alpha\cdot F\) donde \(\alpha\in k[x]\), entonces \(\varphi(P)=\varphi(\alpha)\cdot\varphi(F)=\varphi(\alpha)\cdot0=0\), por lo que \(\ker\varphi=(F)\). Con esto se tiene lo que se quería.
        \item Ya que \(G(y)=0\), \(G\in \ker\varphi=(F)\), por lo que \(F\mid G\).
        \item Como \(a\in L\), \(F(a)=0\) y \(F\in k[x]\subset L[x]\), entonces \((x-a)\mid F\), más aún existe \(F_1\in L[x]\) tal que \(F_1(x)\cdot(x-a)=F(x)\).
    \end{enumerate}
\end{sol}

\begin{prob}{1.54}
    Sea \(R\) un dominio con \(K\) su cuerpo cociente, y sea \(L\) una extensión finita y algebraica de \(K\)
    \begin{enumerate}
        \item Para todo \(v\in L\), demuestre que existe \(a\in R\) distinto a cero tal que \(av\) es integral sobre \(R\)
        \item Muestre que hay una base \(v_1,...,v_n\) para \(L\) sobre \(K\) (como un espacio vectorial) tal que cada \(v_i\) es integral sobre \(R\).
    \end{enumerate}
\end{prob}

\begin{sol}{1.54}
    \begin{enumerate}
        \item Notemos que como \(L\) es una extensión finita, por lo que existen \(x_1,...,x_n\) tal que \(K(x_1,...,x_n)=L\). Se nota que si todos los \(x_i\) cumplen la propiedad pedida y además la suma y la multiplicación de elementos que cumplen la propiedad, tambien la cumplen, tenemos lo pedido. Se observa \(L=K(x_i)\), sea \(1,x_i,x_i^2,...,x_i^k\) tal que sean l.i. y que si se añade \(x_i^{k+1}\) son l.d., esto se logra ya que \(L\) es una extensión finita sobre \(K\). Luego \(x_i^{k+1}\) se puede escribir en la base:
              \begin{equation}
                  x_i^{k+1}=-\sum_{j=0}^ka_jx_iji
              \end{equation}
              Con esto se nota que hay un polinomio mónico \(p(x)=x^{k+1}+\sum_{j=0}^ka_jx^j\), tal que \(x_i\) es raíz. Por el problema 1.51, \(p\) es irreducible (si no lo fuera existiría un cuerpo estrictamente entre \(L\) y \(K\)). Se nota que los \(a_j\in K\), por lo que cada \(a_j=\frac{b_j}{c_j}\) con los \(b_j,c_j\in R\), sea \(c=\prod_{j=0}^kc_j\), luego \(c\cdot p\in R[x]\) y esto claramente nos lleva a concluir que \(cx_i\) es integral. Por lo que todos para todo \(x_i\) existe \(a\in R\) tal que \(ax_i\) es integral.
              Sean \(x,y\in L\) tal que cumplen que \(\exists b,c\in R\) tal que \(bx,cy\) son integrales sobre \(R\). Se recuerda el coralario de la proposición 3\footnote{Los elementos integrales forman un subanillo}, con lo que como \(bx,cy\) son integrales, \(bcxy\) es integral y \(bc(x+y)=cbx+bcy\) es integral, por lo que tenemos lo que queríamos.
        \item Por lo visto en (a), se puede notar que las bases de \(K(x_i)\) se pueden extender con elementos de las otras bases, como los \(x_i\) son integrales, las bases también lo son (son potencias de los \(x_i\))
    \end{enumerate}
\end{sol}

\end{document}