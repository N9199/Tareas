\documentclass[12pt,letterpaper]{article}
\usepackage[utf8]{inputenc}
\usepackage[spanish]{babel}
\usepackage[margin=1in]{geometry}
\usepackage{graphicx}
\usepackage{amsthm, amsmath, amssymb}
\usepackage{mathtools}
\usepackage{setspace}\onehalfspacing
\usepackage[loose,nice]{units}
\usepackage{enumitem}\setlist[enumerate]{label= (\alph*)}
\usepackage{hyperref}
\usepackage{titling}

\hypersetup{
	colorlinks,
	citecolor=black,
	filecolor=black,
	linkcolor=black,
	urlcolor=black
}

\renewcommand{\d}[1]{\ensuremath{\operatorname{d}\!{#1}}}
\renewcommand{\vec}[1]{\mathbf{#1}}
\newcommand{\set}[1]{\mathbb{#1}}
\newcommand{\func}[5]{#1:#2\xrightarrow[#5]{#4}#3}
\newcommand{\contr}{\rightarrow\leftarrow}
\newcommand{\floor}[1]{\left\lfloor#1\right\rfloor}
\newcommand{\ceil}[1]{\left\lceil#1\right\rceil}
\newcommand{\abs}[1]{\left|#1\right|}
\newcommand{\paren}[1]{\left(#1\right)}
\newcommand{\mcm}{\text{mcm }}
\newcommand{\BigO}[2][]{O_{#1}\paren{#2}}
\newcommand{\ds}{\displaystyle}
\newcommand{\cis}{\text{cis }}

\renewcommand{\thesection}{}
\renewcommand{\thesubsection}{}

\DeclareMathOperator{\Ima}{Im}
\DeclareMathOperator{\rad}{rad}

\newenvironment{prob}[1]{
	{\large\raggedleft\textbf{Problema #1:}}\addcontentsline{toc}{section}{Problema #1}\par\addvspace{-\parskip}\noindent
}{}

\newenvironment{sol}[1]{\par\medskip
	\noindent \textbf{Solución problema #1:} \rmfamily}{\begin{flushright}
		$\blacksquare$
	\end{flushright}
}

\title{Tarea 2}
\author{Nicholas Mc-Donnell}
\date{2019/04/11}

\pagenumbering{gobble}

\begin{document}
\begin{minipage}{2.5cm}
    \includegraphics[width=2cm]{../../figures/logo1.jpg}
\end{minipage}
\begin{minipage}{13cm}
    \begin{flushleft}
        \raggedright{
            \noindent
            {\sc Pontificia Universidad Católica de Chile\\
                Facultad de Matemáticas\\
                Departamento de Matemática} \smallskip \\
            Primer Semestre de 2019\\
        }
    \end{flushleft}
\end{minipage}

\vspace{2ex}
{\Large \centerline{\bf \thetitle}}
{\large \centerline{Introducción a la Geometría Algebraica --- MAT 2335}}
{\normalsize \centerline{ Fecha de Entrega: \thedate}}
\vfill

\begin{flushright}
    {\large\theauthor}
\end{flushright}
\newpage
\normalsize
\pagenumbering{arabic}
\tableofcontents
\newpage

\section*{Notas}
En esta tarea se usará la notación \(\overline{a}=(a_1,...,a_n)\)

\begin{prob}{1.25}
    \begin{enumerate}[label= (\alph*)]
        \item Muestre que \(V(y-x^2)\subset\set{A}_\set{C}^2\) es irreducible; en efecto, \(I(V(y-x^2))=I(y-x^2)\)
        \item Separe \(V(y^4-x^2,y^4-x^2y^2+xy^2-x^3)\subset\set{A}_\set{C}^2\) en componentes irreducibles.
    \end{enumerate}
\end{prob}

\begin{sol}{1.25}

\end{sol}

\begin{prob}{1.29}
    Muestre que \(\set{A}_k^n\) es irreducible si $k$ es infinito.
\end{prob}

\begin{sol}{1.29}

\end{sol}

\begin{prob}{1.30}
    Sea \(k=\set{R}\)
    \begin{enumerate}[label= (\alph*)]
        \item Muestre que \(I(V(x^2+y^2+1))=(1)\)
        \item Demuestre que todo subconjunto algebraico de \(\set{A}_\set{R}^2\) es igual a \(V(F)\) para algún \(F\in\set{R}[x,y]\)
    \end{enumerate}
\end{prob}

\begin{sol}{1.30}

\end{sol}

\begin{prob}{1.31}
    \begin{enumerate}[label= (\alph*)]
        \item Encuentre los componentes irreducibles de \(V(y^2-xy-x^2y+x^3)\) en \(\set{A}_\set{R}^2\) y también en \(\set{A}_\set{C}^2\)
        \item Haga lo mismo para \(V(y^2-x(x^2-1))\), y para \(V(x^3+x-x^2y-y)\)
    \end{enumerate}
\end{prob}

\begin{sol}{1.31}

\end{sol}

\begin{prob}{1.33}
    \begin{enumerate}[label= (\alph*)]
        \item Separe \(V(x^2+y^2-1,x^2-z^2-1)\subset\set{A}_\set{C}^3\) en componentes irreducibles.
        \item Sea \(V=\{(t,t^2,t^3)\in\set{A}_\set{C}^3:t\in\set{C}\}\). Encuentre \(I(V)\), y demuestre que es irreducible.
    \end{enumerate}
\end{prob}

\begin{sol}{1.33}

\end{sol}

\begin{prob}{1.37}
    Sea \(k\) un cuerpo cualquiera, \(F\in k[x]\) un polinomio de grado \(n>0\). Muestre que los residuos \(\overline{1},\overline{x},...,\overline{x}^{n-1}\) forman una base de \(k[x]/(F)\) sobre \(k\).
\end{prob}

\begin{sol}{1.37}

\end{sol}

\begin{prob}{1.45}
    Sea \(R\) un subanillo de \(S\), \(S\) un subanillo de \(T\).
    \begin{enumerate}
        \item Si \(S=\sum Rv_i,T=\sum Sw_j\), muestre que \(T=\sum Rv_iw_j\).
        \item Si \(S=R[v_1,...,v_n],T=S[w_1,...,w_m]\), muestre que \(T=R[v_1,...,v_n,w_1,...,w_m]\).
        \item Si \(R,S,T\) son cuerpos, y \(S=R(v_1,...,v_n),T=S(w_1,...,w_m)\), demuestre que \(T=R(v_1,...,v_n,w_1,...,w_m)\).
    \end{enumerate}
\end{prob}

\begin{sol}{1.45}

\end{sol}

\begin{prob}{1.49}
    Sea \(k\) un cuerpo, \(L=k(x)\) el cuerpo de funciones racionales en una variable sobre \(k\).
    \begin{enumerate}
        \item Muestre que todo elemento de \(L\) que es integral sobre \(k[x]\) ya esta en \(k[x]\). (Hint: Si \(z^n+a_1z^{n-1}+...=0\), tome \(z=F/G\), con \(F,G\) coprimos. Entonces \(F^n+a_1F^{n-1}+...=0\), por lo que \(G\) divide a \(F\).)
        \item Muestre que hay un elemento no cero \(F\in k[x]\) tal que para todo \(z\in L\), \(F^nz\) es integral sobre \(k[x]\) para algún \(n>0\).
    \end{enumerate}
\end{prob}

\begin{sol}{1.49}

\end{sol}

\begin{prob}{1.51}
    Sea \(k\) un cuerpo, \(F\in k[x]\) un polinomio irreducible mónico de grado \(n>0\).
    \begin{enumerate}[label= (\alph*)]
        \item Muestre que \(L=k[x]/(F)\) es un cuerpo, y si \(a\) es el residuo de \(x\) en \(L\), entonces \(F(a)=0\).
        \item Suponga que \(L'\) es una extensión de cuerpo de \(k\), \(y\in L'\) tal que \(F(y)=0\). Demuestre que el homorfismo de \(k[x]\) a \(L'\) que toma \(x\) a \(y\), induce un isomorfismo de \(L\) con \(k(y)\).
        \item Con \(L'\), \(y\) como en (b), suponga que \(G\in k[x]\) y \(G(y)=0\). Muestre que \(F\) divide a \(G\).
        \item Muestre que \(F=(x-a)F_1,F_1\in L[x]\)
    \end{enumerate}
\end{prob}

\begin{sol}{1.51}

\end{sol}

\begin{prob}{1.54}
    Sea \(R\) un dominio con \(K\) su cuerpo cociente, y sea \(L\) una extensión finita y algebraica de \(K\)
    \begin{enumerate}
        \item Para todo \(v\in L\), demuestre que existe \(a\in R\) distinto a cero tal que \(av\) es integral sobre \(R\)
        \item Muestre que hay una base \(v_1,...,v_n\) para \(L\) sobre \(K\) (como un espacio vectorial) tal que cada \(v_i\) es integral sobre \(R\).
    \end{enumerate}
\end{prob}

\begin{sol}{1.54}

\end{sol}

\end{document}