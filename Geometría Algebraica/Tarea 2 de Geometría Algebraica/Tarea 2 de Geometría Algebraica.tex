\documentclass[12pt,letterpaper]{article}
\usepackage[utf8]{inputenc}
\usepackage[spanish]{babel}
\usepackage[margin=1in]{geometry}
\usepackage{graphicx}
\usepackage{amsthm, amsmath, amssymb}
\usepackage{mathtools}
\usepackage{setspace}\onehalfspacing
\usepackage[loose,nice]{units}
\usepackage{enumitem}\setlist[enumerate]{label= (\alph*)}
\usepackage{hyperref}
\usepackage{titling}

\hypersetup{
	colorlinks,
	citecolor=black,
	filecolor=black,
	linkcolor=black,
	urlcolor=black
}

\renewcommand{\d}[1]{\ensuremath{\operatorname{d}\!{#1}}}
\renewcommand{\vec}[1]{\mathbf{#1}}
\newcommand{\set}[1]{\mathbb{#1}}
\newcommand{\func}[5]{#1:#2\xrightarrow[#5]{#4}#3}
\newcommand{\contr}{\rightarrow\leftarrow}
\newcommand{\floor}[1]{\left\lfloor#1\right\rfloor}
\newcommand{\ceil}[1]{\left\lceil#1\right\rceil}
\newcommand{\abs}[1]{\left|#1\right|}
\newcommand{\paren}[1]{\left(#1\right)}
\newcommand{\mcm}{\text{mcm }}
\newcommand{\BigO}[2][]{O_{#1}\paren{#2}}
\newcommand{\ds}{\displaystyle}
\newcommand{\cis}{\text{cis }}

\renewcommand{\thesection}{}
\renewcommand{\thesubsection}{}

\DeclareMathOperator{\Ima}{Im}
\DeclareMathOperator{\rad}{rad}

\newenvironment{prob}[1]{
	{\large\raggedleft\textbf{Problema #1:}}\addcontentsline{toc}{section}{Problema #1}\par\addvspace{-\parskip}\noindent
}{}

\newenvironment{sol}[1]{\par\medskip
	\noindent \textbf{Solución problema #1:} \rmfamily}{\begin{flushright}
		$\blacksquare$
	\end{flushright}
}

\title{Tarea 2}
\author{Nicholas Mc-Donnell}
\date{2019/04/11}

\pagenumbering{gobble}

\begin{document}
\begin{minipage}{2.5cm}
    \includegraphics[width=2cm]{../../figures/logo1.jpg}
\end{minipage}
\begin{minipage}{13cm}
    \begin{flushleft}
        \raggedright
        {
            \noindent
            {\sc Pontificia Universidad Católica de Chile\\
                Facultad de Matemáticas\\
                Departamento de Matemática} \smallskip \\
            Primer Semestre de 2019\\
        }
    \end{flushleft}
\end{minipage}

\vspace{2ex}
{\Large \centerline{\bf \thetitle}
{\large \centerline{Introducción a la Geometría Algebraica - MAT 2335}}
\centerline{\normalsize Fecha de Entrega: \thedate}
\vfill

\begin{flushright}
    \large\theauthor
\end{flushright}
\newpage
\pagenumbering{arabic}
\tableofcontents
\newpage

\section*{Notas}
En esta tarea se usará la notación $\overline{a}=(a_1,...,a_n)$

\begin{prob}{1.25}

\end{prob}

\begin{sol}{1.25}

\end{sol}

\begin{prob}{1.29}

\end{prob}

\begin{sol}{1.29}

\end{sol}

\begin{prob}{1.30}

\end{prob}

\begin{sol}{1.30}

\end{sol}

\begin{prob}{1.31}

\end{prob}

\begin{sol}{1.31}

\end{sol}

\begin{prob}{1.33}

\end{prob}

\begin{sol}{1.33}

\end{sol}

\begin{prob}{1.37}

\end{prob}

\begin{sol}{1.37}

\end{sol}

\begin{prob}{1.45}

\end{prob}

\begin{sol}{1.45}

\end{sol}

\begin{prob}{1.49}

\end{prob}

\begin{sol}{1.49}

\end{sol}

\begin{prob}{1.51}

\end{prob}

\begin{sol}{1.51}

\end{sol}

\begin{prob}{1.54}

\end{prob}

\begin{sol}{1.54}

\end{sol}

\end{document}