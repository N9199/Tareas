\documentclass{homework}

\title{Tarea 1}
\date{2020-08-27}
\gdate{2do Semestre 2020}
\author{Nicholas Mc-Donnell}
\course{Teoría de Autómatas y Lenguajes Formales --- IIC2223}


% Comment for final compile
\ifx\condition\undefined
\def\condition{2}
\fi

\ifx\condition\undefined
\immediate\write18{ pdflatex -synctex=1 --jobname="P1\jobname" "\gdef\string\condition{0} \string\input\space\jobname"} 
\immediate\write18{ pdflatex -synctex=1 --jobname="P2\jobname" "\gdef\string\condition{1} \string\input\space\jobname"} 

\immediate\write18{rm *.aux *.log *.out}

\expandafter\stop
\fi

\ifcase\condition
\includecomment{p1}
\excludecomment{p2}
\or
\excludecomment{p1}
\includecomment{p2}
\or
\includecomment{p1}
\includecomment{p2}
\fi


\begin{document}
\maketitle
\newpage
\pagenumbering{arabic}
\begin{p1}
    
    \begin{prob}
        Sea \(\Sigma=\{\#,a\}\). Para \(i\geq1\) se define \(a^i=a\stackrel{i-\text{veces}}{\cdots}a\). Para \(n\geq2\) se define el Lenguaje \(L_n\subseteq\Sigma^*\) de todas las palabras de la forma:
        \[\#a^{i_1}\#a^{i_2}\dots\#a^{i_k}\]
        para algún \(k\geq0\) tal que \(1\leq i_j\leq n\) para todo \(1\leq j\leq k\). Por ejemplo, para \(n=3\) se tiene que \(\#aa\#a\) y \(\#a\#aaa\#aa\) pertenecen a \(L_3\), pero \(\#aaaa\) no pertenece a \(L_3\). Notar que cuando \(k=0\) se tiene que \(w=\varepsilon\), y por lo tanto siempre se cumple que \(\varepsilon\in L_n\).
        \begin{enumerate}
            \item Para un \(n\) arbitrario, muestre como construir un autómata finito determinista \(\mathcal{A}_n\) tal que \(\mathcal{L}(\mathcal{A}_n)=L_n\). Explique por qué su construcción cumple con lo pedido.
            \item Para un \(n\) arbitrario, muestre como construir un autómata finito no-determinista \(\mathcal{B}_n\) con \(n+1\) estados tal que \(\mathcal{L}(\mathcal{B}_n)=L_n\). Explique por qué su construcción cumple con lo pedido.
        \end{enumerate}\
    \end{prob}
    
    \begin{sol}
        \begin{enumerate}
            \item Sea \(Q=\{q_1,q_2,...,q_{n+1},q_{n+2}\}\), \(F=\{q_1,...,q_n\}\) y \(q_0=q_n\). Ahora, definimos \(\delta\) de la siguiente manera:
                  \begin{align*}
                      \delta(q_i,a)  & =\begin{cases}
                          q_{i+1} & \text{ si }1\leq i<n           \\
                          q_{1}   & \text{ si }i=n+1               \\
                          q_{n+2} & \text{ si }i=n\text{ o } i=n+2 \\
                      \end{cases} \\
                      \delta(q_i,\#) & =\begin{cases}
                          q_{n+1} & \text{ si }1\leq i\leq n     \\
                          q_{n+2} & \text{ si }n+1\leq i\leq n+2 \\
                      \end{cases} \\
                  \end{align*}
                  Con lo anterior se define \(\mathcal{A}_n\). Por claridad, se denota \(q_{n+2}\) como un estado `basura', \(q_{n+1}\) como un `comienzo'\footnote{En el sentido de que representa un string que comienza con \(\#\)} y \(q_i\) con \(1\leq i\leq n\) como un estado de \(i\) `a'. Ahora, sea \(w\in L_n\), se nota que \(w=\#a^i\cdot z\) con \(z\in L_n\), donde \(\cdot\) es la operación de concatenación. Dado lo anterior, se nota que es suficiente demostrar que \(\varepsilon\) es aceptada por \(\mathcal{A}_n\) y que dado una palabra \(w=\#a^i\cdot z\) en \(L_n\), con \(z\in L_n\) aceptada por \(\mathcal{A}_n\), es aceptada por \(\mathcal{A}_n\). Lo anterior se puede expresar en la siguiente inducción sobre \(k\). Para el caso base, si \(k=0\), se tiene la palabra vacía \(\varepsilon\), como es la ejecución vacía se queda en \(q_0\) el cual pertenece a \(F\), por lo que la palabra es aceptada. Ahora, sea \(w=\#a^{i_{k+1}}\cdot z\) donde \(z=\#a^{i_k}\dots\#a^{i_1}\) y los \(1\leq i_j\leq n\), se tiene que \(z\) es aceptada por \(\mathcal{A}_n\) por hipótesis inductiva. Luego, se procesa \(w\), comenzando en \(q_0\) se tiene la secuencia de ejecución es \(q_n\xrightarrow{\#} q_{n+1}\xrightarrow{a} q_1\) ahora se tiene \(i_{k+1}\) `a', por lo que se tiene que de \(q_1\xrightarrow{a^{i_{k+1}-1}}q_{i_{k+1}}\) donde \(i_{k+1}\leq n\), con lo que se empieza la ejecución de \(z\) en el estado \(q_{i_{k+1}}\), y notamos que \(\delta(q_j,\#)=q_n\) para \(1\leq j\leq n\), lo cual es cierto en nuestro caso, por lo que se usa la hipótesis inductiva, y se tiene que para \(z\) hay una ejecución que la acepta. Juntando esto con la ejecución de \(\#a^{i_{k+1}}\) se tiene una ejecución que acepta a \(w\), con lo que se cumple la hipótesis inductiva para \(k+1\).
    
                  Con lo anterior se tiene que \(L_n\subseteq\mathcal{L}(\mathcal{A}_n)\). Para demostrar la otra contención, sea \(w\in\Sigma^*\setminus L_n\), como \(w\notin L_n\), se tiene que \(w=z_1\cdot \#\#\cdot z_2\), \(w=z_1\cdot a^{n+1}\cdot z_2\), \(w=z_1\cdot \#\) o \(w=a\cdot z_1\) donde \(z_1,z_2\in\Sigma^*\). 
                  \begin{enumerate}[label=\roman*.]
                      \item Para el primer caso, s.p.d.g. se asume que hay una ejecución de \(z_1\) tal que no termine en \(q_{n+2}\). Dado eso, se tiene que el estado en el que se empieza a evaluar \(\#\#\cdot z_2\) es \(q_i\) con \(1\leq i\leq n+1\). Si \(i=n+1\) se tiene que \(\delta(q_{n+1},\#)=q_{n+2}\), por lo que se entra al estado `basura', con lo que \(w\) no es aceptada. En cambio, si \(i\leq n\) se tiene que \(\delta(q_i,\#)=q_{n+1}\), pero ahora volvemos a la situación anterior, ya que \(\delta(q_{n+1},\#)=q_{n+2}\), por lo que \(w\) no es aceptada.
                      \item Al igual que el caso anterior, se asume que existe una ejecución de \(z_1\) tal que no termine en \(q_{n+2}\). Por lo que el estado donde se empieza a evaluar \(a^{n+1}\cdot z_2\) es \(q_i\) con \(1\leq i\leq n+1\). Si \(i\leq n\), se tiene que al llegar a \(a^{n+1-(n-i)}\cdot z_2\) se está en el estado \(q_n\), pero el siguiente carácter es \(a\), y \(\delta(q_n,a)=q_{n+2}\). En el otro caso, \(i=n+1\) con lo que \(\delta(q_{n+1},a)=q_1\) y queda \(a^n\cdot z_2\) para procesar, similarmente al caso recién mencionado se llega a \(a\cdot z_2\) en el estado \(q_n\), pero \(\delta(q_n,a)=q_{n+2}\).
                      \item Como se menciono en los casos anteriores se asume que existe una ejecución de \(z_1\) que no termina en \(q_{n+2}\). Ahora, si \(i=n+1\) se tiene que \(\delta(q_{n+1},\#)=q_{n+2}\), en cambio si \(i\leq n\) se tiene que \(\delta(q_i,\#)=q_{n+1}\) y ninguno es un estado de aceptación, por lo que \(w\) no es aceptada.
                      \item Como \(q_0=q_n\) y \(\delta(q_n,a)=q_{n+2}\) se tiene que \(w\) no es aceptada.
                  \end{enumerate}
                  Como \(w\notin\mathcal{L}(\mathcal{A}_n)\), se tiene que \(L_n=\mathcal{L}(\mathcal{A}_n)\).
                  \item Para construir \(\mathcal{B}_n\) se toma \(\mathcal{A}_n\) y se quita el estado basura y las transiciones asociadas al mismo. Se nota que este NFA tiene \(n+1\) estados y que cuando se llega a una transición no definida la palabra no es aceptada, se puede usar la misma explicación de la parte anterior, con la diferencia de que en vez de hacer la transición a \(q_{n+2}\) y esperar a que se termine de evaluar la palabra, inmediatamente se ve que la palabra no es aceptada.
        \end{enumerate}
    \end{sol}
\end{p1}
\begin{p2}
    \begin{prob}[2]
        \begin{enumerate}
            \item Demuestre que para todo lenguaje regular \(L\) con \(\varepsilon\notin L\), existe un autómata finito no-determinista \(\mathcal{A}=(Q,\Sigma,\Delta, I, F)\) tal que \(L=\mathcal{L}(\mathcal{A})\), \(\abs{I}=1\) y \(\abs{F}=1\).
            \item Demuestre que existe un lenguaje regular \(L\) con \(\varepsilon\notin L\), tal que para todo autómata finito determinista \(\mathcal{A}=(Q,\Sigma,\delta, q_0, F)\) con \(L=\mathcal{L}(\mathcal{A})\) se cumple que \(\abs{F}\geq 2\).
        \end{enumerate}
    \end{prob}
    
    \begin{sol}[2]
        \begin{enumerate}
            \item Dado un lenguaje regular \(L\) tal que \(\varepsilon\notin L\). Por definición se tiene que existe un DFA \(\mathcal{A}\) tal que \(\mathcal{L}(\mathcal{A})=L\), como todo DFA es un NFA, se tiene que existe un NFA \(\mathcal{A}\) tal que \(\mathcal{L}(\mathcal{A})=L\). Ahora, como \(\varepsilon\notin L\) se sabe que \(F\cap I=\emptyset\). Dado eso, se construye \(\mathcal{A}'\) de la siguiente forma, usando \(\mathcal{A}\) como base se agregan dos estados, \(q_\alpha,q_\Omega\), el único estado inicial y el estado final respectivamente. Ahora, para cada estado inicial de \(\mathcal{A}\) se ven las transiciones que salen de ellos y se agregan a las transiciones de \(q_\alpha\), i.e. para cada \(q_i\in I_\mathcal{A}\), \(q_j\in Q_\mathcal{A}\) y \(a_k\in\Sigma\) tal que \(\Delta_{\mathcal{A}}(q_i,a_k,q_j)\) se tiene que \(\Delta_{\mathcal{A}'}(q_\alpha,a_k,q_j)\), pero con el \(q_j\in Q_{\mathcal{A}'}\) correspondiente. Similarmente, se construye lo mismo para los \(q_i\in F_\mathcal{A}\), \(q_j\in Q_\mathcal{A}\) y \(a_k\in\Sigma\), i.e. en \(\mathcal{A}'\) se tiene \(\Delta_{\mathcal{A}'}(q_j,a_k,q_\Omega)\). Ahora, para demostrar que \(\mathcal{L}(\mathcal{A}')=L\) se hace la doble contención con \(\mathcal{L}(\mathcal{A})\). Entonces, dado \(w\in\mathcal{L}(\mathcal{A})\) se tiene que existe una ejecución \(\rho\) por \(\mathcal{A}\) tal que \(w\) es aceptada, \(\rho\) se ve la de siguiente manera \(p_0\xrightarrow{a_1}p_1\xrightarrow{a_2}\cdots\xrightarrow{a_{n-1}}p_{n-1}\xrightarrow{a_n}p_n\), donde \(p_0\in I_\mathcal{A}\) y \(p_n\in F_\mathcal{A}\), entonces se tiene que \(\rho':q_\alpha\xrightarrow{a_1}p_1\xrightarrow{a_2}\cdots\xrightarrow{a_{n-1}}p_{n-1}\xrightarrow{a_n}q_\Omega\) es una ejecución de \(\mathcal{A}'\) sobre \(w\), ya que \(\Delta_{\mathcal{A}}(p_0,a_1,p_1)\) se tiene si y solo si \(\Delta_{\mathcal{A}'}(q_\alpha,a_1,p_1)\), similarmente \(\Delta_{\mathcal{A}}(p_{n-1},a_n,p_n)\) se tiene si y solo si \(\Delta_{\mathcal{A}'}(p_{n-1},a_n,q_\Omega)\). Como \(q_\Omega\in F\), se tiene que \(w\in\mathcal{L}(\mathcal{A}')\), por lo que \(\mathcal{L}(\mathcal{A})\subseteq\mathcal{L}(\mathcal{A}')\)\footnote{Como \(F_\mathcal{A}\cap I_\mathcal{A}=\emptyset\) todo ejecución de \(\mathcal{A}\) es de largo al menos uno y pasa por al menos dos estados, por lo que \(\rho'\) siempre está bien definida.}. Pero el argumento se puede hacer al revés, o sea, dado una ejecución \(\rho'\) de \(\mathcal{A}'\) que acepta \(w\in\mathcal{L}(\mathcal{A}')\) se tiene que \(\rho':q_\alpha\xrightarrow{a_1}p_1\xrightarrow{a_2}\cdots\xrightarrow{a_{n-1}}p_{n-1}\xrightarrow{a_n}q_\Omega\), ahora por construcción existe \(p_0\in I_\mathcal{A}\) tal que \(\Delta_\mathcal{A}(p_0,a_1,p_1)\), y se tiene que existe \(p_n\in F_\mathcal{A}\) tal que \(\Delta_\mathcal{A}(p_{n-1},a_n,p_n)\), por lo que \(\rho:p_0\xrightarrow{a_1}p_1\xrightarrow{a_2}\cdots\xrightarrow{a_{n-1}}p_{n-1}\xrightarrow{a_n}p_n\) es una ejecución de \(\mathcal{A}\) que acepta \(w\). Por lo que se tiene que \(\mathcal{L}(\mathcal{A}')=\mathcal{L}(\mathcal{A})=L\). Más aún se tiene que \(\abs{I_{\mathcal{A}'}}=\abs{F_{\mathcal{A}'}}=1\).
            \item Sea \(\Sigma=\{a\}\) y \(L=\{a,aa\}\), sea \(\mathcal{A}\) un DFA tal que \(\mathcal{L}(\mathcal{A})=L\). Luego, existe una ejecución \(\rho_a\) de \(\mathcal{A}\) sobre \(a\) que la acepta. Como \(\abs{a}=1\) entonces \(\rho_a:q_0\xrightarrow{a}p_1\) donde \(q_0\) es el estado inicial y \(p_1\in F\). Similarmente, existe una ejecución \(\rho_{aa}\) que acepta \(aa\), vemos que \(rho_{aa}:q_0\xrightarrow{a}p_1\xrightarrow{a}p_2\), donde el \(p_1\) es el mismo estado que en la ejecución \(\rho_a\)\footnote{Si fuera distinto, \(\delta\) no sería función, y por ende \(\mathcal{A}\) no sería DFA.} y \(p_2\in F\). Luego, se tienen dos opciones \(p_1\neq p_2\) y \(p_1=p_2\) con la primera se tiene que \(\abs{F}\geq2\), consiguiendo lo pedido. Por lo tanto se observará el segundo caso, se nota que \(\delta(p_1,a)=p_1\), ahora sea \(\rho_{aaa}\) la ejecución de \(\mathcal{A}\) sobre la palabra \(aaa\), se tiene que \(\rho_{aaa}:q_0\xrightarrow{a}p_1\xrightarrow{a}p_1\xrightarrow{a}p_1\), y como \(p_1\in F\) se tiene que \(aaa\) es aceptada por \(\mathcal{A}\), pero se recuerda que \(aaa\notin L\) y \(\mathcal{L}(\mathcal{A})=L\), lo que es una contradicción. Se tiene entonces que \(L\) cumple lo pedido.
        \end{enumerate}
    \end{sol}
\end{p2}


\end{document}