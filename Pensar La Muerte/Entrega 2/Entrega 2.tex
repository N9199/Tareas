\documentclass{homework}

\title{Trabajo N$^{\circ}$2}
\date{2020-05-05}
\gdate{1er Semestre 2020}
\author{Nicholas Mc-Donnell}
\course{Para Pensar la Muerte - TTF058}

\geometry{margin=2.5cm}
\linespread{1.15}

\begin{document}
\maketitle

% ¿Por qué la muerte nos produce angustia?
La problemática a trabajar aparece de la siguiente pregunta, ¿Por qué la muerte produce angustia? Contextualizando está pregunta en los filósofos vistos, se tienen las siguientes preguntas ¿Como la narración personal nos lleva a la angustia? ¿Como la inquietud nos lleva a la angustia? Y por último, la pregunta original pero desde Heidegger.

Como dice Ricoeur, cada persona construye una narrativa a través de la autocomprensión a distancia, y es a través de estás narrativas que se genera la angustia. Se pueden consideraran dos posibles narrativas en especifico, la narrativa generada por como se recuerda a los muertos, y la narrativa generada por qué es la muerte. En el primer caso se ve el ejemplo de Alfred Nobel quien se enfrento a la narrativa que construyo la sociedad sobre su él, la del mercader de la muerte, por una confusión respecto a la muerte de su hermano, y como al enfrentarse a está narrativa, este sufrió una profunda angustia al tener que reconocer la narrativa que había sido creada sobre él, más aún este hizo lo posible para poder cambiar esta narrativa, específicamente a través de la creación del premio Nobel. Más generalmente, todo humano siente intriga sobre la narrativa que tiene el resto de la humanidad sobre su propia existencia, y al considerar una posible narrativa negativa, uno llega fácilmente a la angustia, dado que a diferencia de Nobel es difícil cambiar la narrativa que la sociedad tiene de uno, sin poder saber cual es esta misma y sin tener los medios para cambiarla (en el caso de Nobel dinero). Esto último abarca a casi toda persona, y es reconocer esto lo cual nos lleva a la angustia, la angustia de no poder controlar la narrativa que se queda para los vivos, de controlar nuestra imagén después de la muerte. Habiendo visto esta narrativa, se puede considerar otra narrativa que nos puede acomplejar, la correspondiente a la muerte, esta narrativa corresponde a como el individuo ve la muerte y como la sociedad ve la muerte. Para ejemplificar esto, se ve la diferencia entre una visión judeo-cristiana del Cielo contra una visión no religiosa en la cual hay nada, en la primera se ve puede tener una esperanza de que la muerte traiga una recompensa por una vida ``bien'' vivida, pero el castigo el Infierno en el otro caso, en contraste, la nada corresponde a la situación independiente de como se viva la vida. Ahora, son los casos negativos, o que sea no positivos, lo cuales nos llevan a la angustia, al final el enfrentarse a una narración de sufrimiento eterno, o simplemente la finalización de la narración nos lleva a una profunda angustia, ya que de nuevo nos encontramos con situaciones mayormente fuera de nuestro control.

Continuando, es esta falta de control también que la cual nos deja inquietos, que nos lleva a intentar entender (y fallar) la experiencia del otro, para que uno como individuo pueda satisfacer esta inquietud generada por el no conocimiento del otro, y al mismo tiempo por el entendimiento de que el otro es distinto, pero que esta inquietud nunca podrá tener respuesta, por como dice Levinas ``La muerte es un viaje sin regreso, pregunta sin datos, puro signo de interrogación''. Y el darnos cuenta de que no podemos resolver esta interrogante nos genera una profunda angustia, ya que solo nos queda enfrentarnos a la muerte, enfrentarnos al otro, y con esto enfrentarnos a la nada.

Es aquí que donde Heidegger nos recuerda que cosa nos caracteriza como humanos, como \textit{dasein}, al ser, al tener que enfrentar el existir, y por ende el fin de esta existencia. Pero es este enfrentamiento el cual nos lleva a darnos cuenta de una horrible situación, la muerte es algo insuperable, algo tan ajeno a nosotros que el enfrentarse a la posibilidad de esta nos afecta profundamente. Es esta afección sobre la cual basamos nuestra existencia, y sobre la cual basamos nuestra responsabilidad con existir. Visto de otra forma, nuestra existencia (y por ende su fin) nos provoca una profunda angustia que nos acompaña desde el instante que tenemos consciencia de la propia existencia, todo esto solo por el hecho de que no nos podemos enfrentar a la existencia y por ende somos entes arrojados a la misma.

Con esto se ve que hay variadas formas de las que uno puede llegar a la angustia, sea a través de enfrentarse con las narraciones de la muerte, las narraciones de la sociedad sobre uno mismo después de la muerte, el otro (en la muerte), o nuestra existencia (y su fin). Más específicamente, se que de distintas formas uno llega a la angustia, la angustia de no tener control sobre la narración, la angustia de no poder saber sobre el otro y la angustia de no poder sobrepasar el fin de nuestra existencia (y al mismo tiempo nuestra misma existencia), y todas estas cosas vienen de enfrentarse con la muerte, específicamente nuestra muerte, y sus posibles consecuencias. O sea, es a través nuestra interacción con la muerte, más específicamente, a través de distintas facetas de la misma, como llegamos a la angustia.

\end{document}