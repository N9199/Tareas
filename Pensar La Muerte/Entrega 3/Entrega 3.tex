\documentclass{homework}

\title{Trabajo N$^{\circ}$3}
\date{2020-07-03}
\gdate{1er Semestre 2020}
\author{Nicholas Mc-Donnell}
\course{Para Pensar la Muerte - TTF058}

\geometry{margin=2.5cm}
\linespread{1.15}

\begin{document}
\maketitle
%
% El proceso atemporal de resucitar, y tener que forzosamente interactuar con gente, porque esas personas quieren que uno lo perdone, la existencia de ese proceso causa angustia.
% 
%
%
% ¿Por qué la muerte nos produce angustia?
La problemática a trabajar aparece de la siguiente pregunta, ¿por qué la muerte produce angustia? Para tratar esta interrogante desde la perspectiva de la teología católica se contrastarán los siguientes conceptos: vivir la eternidad y como el cristianismo se basa en narraciones. Comenzaremos analizando los conceptos bajo una óptica cristiana, para luego contrastar con las visiones de los filósofos vistos anteriormente.

Una parte importante de las creencias cristianas tienen que ver con la resurrección y, a su vez, una parte importante del proceso de resurrección es vivir la eternidad. Esto corresponde a la parte de la resurrección donde cada persona se encuentra con todos aquellos con quienes interactuó en su vida, comunicando a cada cual lo que se desea, para así dar la oportunidad de ser perdonados antes de continuar con el proceso de resurrección. Esta parte de dicho proceso es inherentemente problemática, ya que cada persona está forzada a entender lo que quiera comunicar la otra, independiente de sus propios deseos. Es este factor en el cual se centrará el siguiente análisis a través de un ejemplo: si una persona sobrevive la experiencia de sufrir abuso por parte de un familiar, entonces esta persona no querrá volver a enfrentarse a quien abusó de ella. Pese a que muchas veces esta persona llega a entender las razones de fondo y todo el proceso por el cual pasó quien abusó de ella para llegar a cometer tal acto, la parte afectada no va a querer enfrentarse a quien cometió el abuso. Esto puede tener múltiples razones de fondo, pero ya que no es parte importante del argumento solo se mencionarán algunas posibles causas. Entre ellas está el hecho de que la persona que abusó nunca haya sentido remordimiento por sus acciones, o que esta persona sintiera que causar daño a otras personas es justificado por el simple hecho que son otras personas. Volviendo al tema principal, si la persona que sufrió del abuso cree en la fe cristiana, específicamente en la resurrección, y aún más específicamente en vivir la eternidad, esta persona tendrá que enfrentarse con el horror de tener que estar forzada a interactuar con la persona que abusó de ella, más aun va a tener que comprenderla. Este es un proceso que desde la visión limitada de un ser que solo tiene experiencias terrenales suena horroroso, tener que pasar una eternidad\footnote{La eternidad corresponde a la idea más cercana a la atemporalidad que un ser terrenal puede llegar a comprender.} comprendiendo la persona que, en el mejor casa, solo causo un dolor inmenso. Enfrentarse a esto, y no tener un escape, lleva a una profunda angustia, y enfrentarse a esto esta inherentemente relacionado con enfrentarse a la muerte, ya que es el instante anterior a este proceso atemporal. Este caso en especifico tiene similitudes con lo que Heidegger presenta respecto a como el ser humano se enfrenta a la muerte, y la angustia que genera enfrentarse con este hecho. Pero a diferencia del cristianismo, Heidegger propone que no hay algo después de la muerte, por lo que no sé generaría la angustia que aparece por enfrentarse a la persona que abusó de ella, ya que no se llega a esa instancia después de la muerte. Esto no quita que aún se pueda generar angustia al tener que enfrentarse a la mortalidad, ya que el mismo hecho de ser mortal genera angustia.

Esto nos lleva de forma natural a pensar en como las narraciones sobre la muerte afectan nuestra relación con ella misma. Las narraciones en el contexto cristiano corresponden a los evangelios del nuevo testamento, las cuales, en general, fueron escritos por terceros que escucharon de los hechos de la salvación a traves de otras fuentes, y además escriben estos evangelios considerando que el público de los mismos está inmerso en un contexto donde las historias del antiguo testamento son conocimiento común. Es esto último lo que nos lleva a tener que ver los evangelios como narraciones, ya que el individuo contemporáneo no tiene este contexto y, por ende, solo puede interpretar los evangelios. Aquí la idea de interpretación es particularmente importante porque, al no tener una interacción directa con lo narrado por tales evangelios, debemos basar nuestra comprensión de lo narrado a partir de un texto. Al respecto, Ricoeur trata de manera especial las narraciones bajo tales circunstancias. Son dichas narraciones las que permiten que podamos interactuar con la muerte y, dentro del contexto cristiano que compete, con la resurrección. En este sentido, esto nos lleva naturalmente a que se genere una angustia ya que esta distancia e incertidumbre respecto a nuestra interpretación se basa en el ya mencionado hecho de que no tenemos acceso directo a la salvación, sino que solo podemos acceder a una idea de ella mediante los textos que se nos han hecho presentes desde los inicios de la era cristiana. Es decir, como no podemos vivir aquello a lo que hacemos constante referencia, la incertidumbre que se genera debido a la interpretación (que, por supuesto, no es necesariamente única) es la responsable de que la angustia tratada en el caso anteriormente relatado se replique en este nuevo contexto.

A modo de conclusión, la angustia es transversal a la noción de resurrección y muerte, sin importar el contexto en el cual nos estamos ubicando. Es decir, tanto en la interpretación cristiana dentro de bajo cierta noción religiosa de lo que es la resurrección, como en los ámbitos más cotidianos y ajenos a la interpretación bíblica, se encuentra presente una idea de angustia que refleja nuestra relación con la muerte. Es aquí donde la salvación cristiana y la idea de resurrección juega un papel principal, pues surge como reacción a la angustia; como una forma a través de la cual podemos superar el terror que genera la muerte mediante el evangelio y su interpretación.
\end{document}