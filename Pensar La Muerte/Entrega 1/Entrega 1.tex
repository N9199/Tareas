\documentclass{homework}

\title{Entrega 1}
\date{2020-03-26}
\gdate{1er Semestre 2020}
\author{Nicholas Mc-Donnell}
\course{Para Pensar la Muerte - TTF058}


\begin{document}
\maketitle
\newpage
\pagenumbering{arabic}

% ¿Por qué la muerte nos produce angustia?
La problemática a trabajar aparece de la siguiente pregunta, ¿Por qué la muerte produce angustia? Esta pregunta naturalmente nos lleva a preguntarnos varias otras cosas, entre ellas ¿Cómo nuestra relación con la vida nos lleva a la angustia? y ¿De qué forma nuestra experiencias previas con la muerte nos llevan a esta angustia?

Para el análisis de la problemática se usarán como base estas dos preguntas, y se responderán bajo una visión que no asumirá la existencia de algún sentido de la vida universal, pero se presupondrá que todo ser humano ha encontrado, o está en la búsqueda de, un sentido de vida individual. Todo esto será acompañado por mis experiencias en la búsqueda de sentido en la vida y mis experiencias con la muerte.

Considerando la segunda pregunta en relación al sentido de vida, se ve de forma clara la individualización de la muerte. Esto se puede ver desde la famosa frase de Nietzche ``Dios ha muerto'', esta es la muerte de un sentido global en la cultura occidental, y esta pérdida fuerza la búsqueda de un sentido de vida, lo que individualiza la relación con la muerte. Y es esta individualización, la cual muchas veces nos lleva a ver negativamente a la muerte, o más aún, nos lleva a una profunda angustia. Una posible avenida a explorar de como la individualización y la búsqueda de sentido nos lleva a una visión negativa de la muerte es la de la valorarización de la vida, versus la valorarización de la muerte. Lo anterior se ve de forma clara en la religión cristiana, la cual habla de un mundo después de la muerte, el Cielo, el cual se valoriza por su cualidad de eterno. En contraste, alguien ateo, quien no cree en una vida eterna después de la muerte valoriza la vida, por su cualidad de ser única, y esto llevaría a ver la muerte como un fin de está experiencia, más aún el fin de su existencia. Es aquí donde se ve como la valorarización cambia nuestra relación con la muerte, pero aún así, quedaría la pregunta ¿Por qué un creyente en la vida después de la muerte aún así puede sentir angustia? A lo que queda una posible respuesta a está última pregunta, la incertidumbre, algo que causa angustia, ansiedad, y muchas otras emociones, y el caso de la muerte no es una excepción.

A diferencia de la segunda pregunta, la primera pregunta es multifacética, esta se caracteriza por como nuestra experiencia con la muerte ha cambiado, sea a través de la mediatización de la muerte o la biologización de la misma. Ambas son facetas del cambio de nuestras experiencias con la muerte, y cada una nos lleva a crear una idea propia de la significancia de la muerte, sea una vista analítica como la genera la biologización, o una desensibilización a la brutalidad de parte de la mediatización. Para ver estás facetas más concretamente, es necesario primero considerar como nuestra visión de la muerte es generada por como nos relacionamos con la vida, y esto se puede ejemplificar viendo gente con depresión y viendo gente con buena salud que además tienen un estado natural de felicidad. Una persona con depresión siente tanto dolor, tanta tristeza, que puede llegar a un punto donde la muerte, el fin de su propia vida, se ve como algo preferible, y esto se debe a que esta persona ve la vida como un constante sufrimiento el cual sobrepasa a la persona, es así como esta ve la muerte como un escape. En contraste, una persona que disfruta de su vida ve a la muerte como un fin de esto, y por ende la ve negativamente.



\end{document}