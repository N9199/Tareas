\documentclass{homework}

\title{Tarea 1}
\date{2020-09-26}
\gdate{1er Semestre 2020}
\author{Nicholas Mc-Donnell}
\course{Geometría Diferencial - MAT}


\begin{document}
\maketitle
\newpage
\pagenumbering{arabic}


\begin{prob}
    Un disco de radio 1 en el plano \(xy\) rueda sin deslizarse a lo largo del eje \(x\). La figura que describe un punto de la circunferencia del disco se llama cicloide.
\begin{enumerate}
    \item Encuentre una curva parametrizada cuya traza sea la cicloide, y determine sus puntos críticos.
    \item Calcule la longitud del arco de la cicloide correspondiente a una vuelta del disco.
\end{enumerate}
\end{prob}

\begin{sol}
    
\end{sol}


\begin{prob}
    Sea \(\alpha:(-1,\infty)\rightarrow\set{R}^2\) dada por
    \begin{equation*}
        \alpha(t)=\paren{\frac{3at}{1+t^3},\frac{3at^2}{1+t^3}}
    \end{equation*}
    demuestre que:
    \begin{enumerate}
        \item En \(t=0\), \(\alpha\) es tangente al eje \(x\).
        \item Cuando \(t\rightarrow 0\), \(\alpha(t)\rightarrow(0,0)\) y \(\alpha'(t)\rightarrow(0,0)\)
        \item Cuando \(t\rightarrow-1\), \(\alpha\) y su tangente tienden a la recta \(x+y+a=0\).
        \item El arco con \(t\in(0,\infty)\) es simétrico con respecto a la recta \(y=x\).
    \end{enumerate}
    La figura que se obtiene completando la traza para que sea simétrica con respecto a la recta \(y=x\) en todo punto se denomina el \textit{folium de Descartes}.
\end{prob}

\begin{sol}
    
\end{sol}


\begin{prob}
    (Líneas rectas son las más cortas.) Sea \(\alpha:[a,b]\rightarrow \set{R}^3\) una curva parametrizada con \(\alpha(a)=p\) y \(\alpha(b)=q\).
    \begin{enumerate}
        \item Demuestre que para cualquier vector unitario \(v\) se cumple
        \begin{equation*}
            (q-p)\cdot v=\int_a^b\alpha'(t)\cdot v\d{t}\lq\int_a^b\norm{\alpha'(t)}\d{t}
        \end{equation*}
        \item Use lo anterior para demostrar que
        \begin{equation*}
            \norm{\alpha(b)-\alpha(a)}\leq\int_a^b\norm{\alpha'(t)}\d{t}
        \end{equation*}
    \end{enumerate}
\end{prob}

\begin{sol}
    
\end{sol}


\begin{prob}
    Demuestre que si todos los planos normales a una curva pasan por un punto fijo entonces la curva está contenida en una esfera.
\end{prob}

\begin{sol}
    
\end{sol}


\begin{prob}
    Sea \(\alpha:I\rightarrow\set{R}^3\) una curva parametrizada regular (no necesariamente arcoparametrizada) y sean \(s=s(t)\) su longitud de arco y \(t=t(s)\) la inversa de este. Denotamos \(()'\) a las derivadas respecto a \(t\). Demuestre que:
    \begin{enumerate}
        \item \begin{equation*}
            \frac{\d{t}}{\d{s}}=\frac1{\norm{\alpha'}}\text{ y }\frac{\d{^2t}}{\d{s^2}}=-\frac{a'\cdot\alpha''}{\norm{a'}^4}
        \end{equation*}
        \item La curvatura de \(\alpha\) en \(t\) es
        \begin{equation*}
            \kappa(t)=\frac{\norm{\alpha'\wedge\alpha''}}{\norm{a'}^3}
        \end{equation*}
        \item La torsión de \(\alpha\) en \(t\) es
        \begin{equation*}
            \tau(t)=-\frac{(\alpha'\wedge\alpha'')\cdot\alpha'''}{\norm{\alpha'\wedge\alpha''}^2}
        \end{equation*}
    \end{enumerate}
\end{prob}

\begin{sol}
    
\end{sol}


\begin{prob}
    Sea \(\alpha:I\rightarrow\set{R}^3\) una curva arcoparametrizada regular con \(k(s)\neq0\) en todo \(I\). Demuestre que:
    \begin{enumerate}
        \item El plano osculador es el límite de los planos que pasan por \(\alpha(s),\alpha(s+h_1)\) y \(\alpha(s+h_2)\) cuando \(h_1,h_2\rightarrow0\).
        \item El límite de los círculos que pasan por \(\alpha(s),\alpha(s+h_1)\) y \(\alpha(s+h_2)\) cuando \(h_1,h_2\rightarrow0\) es un círculo en el plano osculador con centro en la recta normal y radio el radio curvatura de \(\alpha,r=1/\kappa(s)\). Este círculo se conoce como el círculo osculador de \(\alpha\) en \(s\).
    \end{enumerate}
\end{prob}

\begin{sol}
    
\end{sol}


\end{document}