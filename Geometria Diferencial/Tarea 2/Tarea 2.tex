\documentclass{homework}

\title{Tarea 1}
\date{2020-04-07}
\gdate{1er Semestre 2020}
\author{Nicholas Mc-Donnell}
\course{Geometría Diferencial - MATMAT2305}


\begin{document}
\maketitle
\newpage
\pagenumbering{arabic}


\begin{prob}
    (Proyección estereográfica) Una forma de obtener coordenadas para la esfera \(x^2+y^2+(z-1)^2=1\) es usando la \textit{proyección esterográfica} \(\pi:S^2\setminus\{N\}\rightarrow\set{R}^2\) que lleva cada punto \(p=(x,y,z)\), excepto el polo norte \(N=(0,0,2)\), a la intersección de la recta de \(N\) a \(p\) con el plano \(xy\). Sea \((u,v)=\pi(x,y,z)\)
    \begin{enumerate}
        \item Demuestre que \(\pi^{-1}:\set{R}^2\rightarrow S^2\) es un parche coordenado para la esfera, y que está dado por
        \begin{equation*}
            \pi^{-1}(u,v)=\paren{\frac{4u}{u^2+v^2+4}, \frac{4v}{u^2+v^2+4}, \frac{2(u^2+v^2)}{u^2+v^2+4}}
        \end{equation*}
        \item Demuestre que usando estas coordenadas se puede cubrir la esfera con dos parches coordenados.
    \end{enumerate}
\end{prob}

\begin{sol}

\end{sol}


\begin{prob}
    Sea \(C\) una ``figura 8'' en el plano \(xy\) y \(S\) la superficie cilíndrica sobre \(C\), eso es: \(S=\{(x,y,z)\in\set{R}^2:z\in\set{R}\wedge (x,y)\in C\}\). ¿Es \(S\) una superficie regular? (demuestrelo)
\end{prob}

\begin{sol}

\end{sol}


\begin{prob}
    Considere la función \(f(x,y,z)=x^2+y^2-z^2\). Determine para que valores de \(c\in\set{R}\) la ecuación \(f(x,y,z)=c\) es una superficie regular, describa y esboce cada caso. Cuando no es superficie regular, de una razón geométrica y una forma de solucionarlo.
\end{prob}

\begin{sol}
    Se nota que para \(c\neq0\) el punto \(\vec{0}\) no es solución, más aún \(\vec{0}\) es solución si y solo si \(c=0\). Viendo \(\nabla f\)\footnote{\(\nabla f=(2x,2y,-z2)\)}, se nota que \(\nabla f=\vec{0}\) solo en \(\vec{0}\) y \(\vec{0}\) solo pertenece a la superficie si \(c=0\), por lo que la superficie es regular ssi \(c\neq0\).
\end{sol}


\begin{prob}
    Sea \(S^2\) la esfera unitaria \(x^2+y^2+z^2=1\) y \(A:S^2\rightarrow S^2\) el \textit{mapa antipodal} dado por \(A(x,y,z)=(-x,-y,-z)\). Demuestre que \(A\) es un difeomorfismo.
\end{prob}

\begin{sol}
    Se nota que \(A\circ A=Id_{S^2}\), por lo que \(A\) es biyectiva. Es claro que es continua, ya que \(A\) es una restricción de la transformación lineal \(T:\set{R}^3\rightarrow\set{R}^3\) dada por \(T(\vec{x})=-\vec{x}\). Ahora, se nota que \(S^2\) se puede parametrizar localmente como \(\phi(u,v)=(\cos(u)\sin(v),\sin(u)\sin(v),\cos(v))\) con \((u,v)\in[0,2\pi)\times[0,\pi]\), con esto se nota que \(\phi^{-1}\circ A\circ\phi(u,v)=(u,\pi-v)\), lo cual es claramente diferenciable, por lo que \(A\) tambien es diferenciable, más aún es un difeomorfismo.
\end{sol}


\begin{prob}
    Podemos identificar el plano \(xy\) en \(\set{R}^3\) con el plano complejo \(\set{C}\) identificando \((x,y,0)\sim x+iy=\zeta\). Sean \(P(\zeta)=a\zeta^n\) y \(Q(\zeta)=\zeta+b,a,b\in\set{C}\) dos polinomios complejos. Sea \(\pi:S^2\rightarrow\set{R}^2\) la proyección estereográfica del Problema 1. Demuestre que los mapas \(F:S^2\rightarrow S^2\) y \(G:S^2\rightarrow S^2\) dados por
    \begin{equation*}
        F(p)=\begin{cases}
            \pi^{-1}\circ P\circ \pi &\text{si }p\neq N\\
            N &\text{si }p=N
        \end{cases}\quad\text{y}\quad G(p)=\begin{cases}
            \pi^{-1}\circ Q\circ \pi &\text{si }p\neq N\\
            N &\text{si }p=N
        \end{cases}
    \end{equation*}
    son diferenciables, y describa sus acciones en la esfera.
\end{prob}

\begin{sol}

\end{sol}


\begin{prob}
    Sea \(\alpha:(a,b)\rightarrow\set{R}^3\) una curva parametrizada regular que no pasa por el origen. Sea \(\Sigma\) el conjunto compuesto por los puntos de los segmentos desde el origen que pasan por puntos de \(\alpha\).
    \begin{enumerate}
        \item Encuentre una superficie parametrizada son traza \(\Sigma\).
        \item Encuentre los puntos donde no es una superficie parametrizada regular.
        \item ¿Qué tenemos que remover de \(\Sigma\) para obtener una superficie regular?
    \end{enumerate}
\end{prob}

\begin{sol}
    \begin{enumerate}
        \item Sea \(\gamma:(a,b)\rightarrow S^2\) definida como \(\gamma(x)=\frac{\alpha(x)}{\norm{\alpha(x)}}\), sea \(\Gamma\) la imagén de \(\gamma\), se nota que \(\Gamma\) puede ser una de las siguientes 3, un singleton \(\{P\}\),  una curva abierta parametrizada por \(\gamma\), o una curva cerrada parametrizada por \(\gamma^*:[0,1]\rightarrow S^2\). Veremos cada caso por separado:
        \begin{enumerate}[label=\roman*.]
            \item  En el primer caso se nota que esto implica que \(\Sigma\) es un segmento o un rayo, por lo que \(\Sigma\) no es una superficie, es una curva, y dependiendo de si un rayo o un segmento, es parametrizable tomando la función que hace el siguiente mapeo \(x\mapsto xP\)\footnote{El \(P\) del singleton \(\Gamma\)}, donde el dominio de la función es \([0,\sup_{x\in(a,b)}\norm{\alpha(x)}]\) si el supremo existe o \([0,\infty)\) si no existe.
            \item En el segundo caso, se define \(\beta:(a,b)\times[0,1]\rightarrow\set{R}^3\) con \(\beta(x,y)=y\alpha(x)\), y se nota que \(\beta\) corresponde a una parametrización de \(\Sigma\).
            \item Para el último caso es necesario ver dos subcasos, el primero es cuando \(\sup_{x\in(a,b)}\norm{\alpha(x)}\) existe, y el segundo es cuando no existe. En el primer subcaso se nota que es posible ver la curva parametrizada por \(\alpha\) como la unión disjunta numerable de curvas minimales que al ser proyectadas sobre \(S^2\) tienen como imagén a \(\Gamma\), y por ende son parametrizables por un conjunto de funciones \(\{\rho_i:[0,1)\rightarrow\set{R}^3\}\)\footnote{Salvo a los más una de estas curvas la cual tendría como imagén \(\Gamma\) sin un punto, está además tendría como dominio \((0,1)\)}, de este conjunto se puede tomar \(\rho_j\) tal que \(\sup_{x\in[0,1)}\norm{\rho_j(x)}\) sea maximal. Con lo anterior es posible parametrizar \(\Sigma\) con la siguiente función \(\beta^*:[0,1)\times[0,1]\rightarrow\set{R}^3\) donde \(\beta^*(x,y)=y\rho_j(x)\). En el segundo subcaso es claro que \(\beta^*:[0,1)\times[0,\infty)\) donde \(\beta^*(x,y)=y\gamma^*(x)\) parametriza \(\Sigma\).
        \end{enumerate}
        \item Dados los casos mencionados anteriormente, se nota que en el primer caso no hay punto donde sea una superficie parametrizada regular, por lo que todos los puntos corresponden. Para el segundo caso es suficiente hacer un análisis de las auto intersecciones de \(\gamma\), recordar la parametrización \(\beta\) y ver los puntos del dominio que son de la forma \((x,y)\) con \(x\) siendo un punto de auto intersección de \(\gamma\). Similarmente para el tercer caso es suficiente hacer el mismo análisis que para el segundo caso, pero sobre \(\gamma^*\) y recordando \(\beta^*\).
        \item Depende de los casos, pero salvo el primero al remover los puntos mencionados anteriormente \(\Sigma\) se puede ver como la unión de superficies regulares, por lo que se puede elegir alguna de ellas y remover el resto.
    \end{enumerate}
\end{sol}

\end{document}