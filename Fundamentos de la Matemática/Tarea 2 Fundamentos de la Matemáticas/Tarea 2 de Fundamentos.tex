\documentclass[12pt,letterpaper]{article}
\usepackage[T1]{fontenc}
\usepackage[spanish]{babel}
\usepackage[margin=1in]{geometry}
\usepackage{amsthm, amsmath, amssymb}
\usepackage{mathtools}
\usepackage{setspace}\onehalfspacing
\usepackage[loose,nice]{units}
\usepackage{enumitem}
\usepackage{float}

\usepackage{hyperref}
\usepackage{url}
\usepackage{color,graphicx}
\usepackage{fullpage}
\usepackage{multicol}
\usepackage{tabularx}
\usepackage[natbibapa]{apacite}
\usepackage{titling}

\usepackage{xargs}

\graphicspath{{../figures/}}

\hypersetup{
	colorlinks,
	citecolor=black,
	filecolor=black,
	linkcolor=black,
	urlcolor=black
}

\renewcommand{\d}[1]{\ensuremath{\operatorname{d}\!{#1}}}
\renewcommand{\vec}[1]{\mathbf{#1}}
\newcommand{\set}[1]{\mathbb{#1}}
\newcommand{\func}[5]{#1:#2\xrightarrow[#5]{#4}#3}
\newcommand{\contr}{\rightarrow\leftarrow}
\newcommand{\floor}[1]{\left\lfloor#1\right\rfloor}
\newcommand{\ceil}[1]{\left\lceil#1\right\rceil}
\newcommand{\abs}[1]{\left|#1\right|}
\newcommand{\angled}[1]{\left\langle#1\right\rangle}
\newcommand{\paren}[1]{\left(#1\right)}
\newcommand{\mcm}{\text{mcm }}
\newcommand{\BigO}[2][]{O_{#1}\paren{#2}}
\newcommand{\ds}{\displaystyle}
\newcommand{\cis}{\text{cis }}
\newcommand{\dom}{\text{Dom }}

\newcommand{\nope}{\(\contr\)}

\DeclareMathOperator{\Ima}{Im}

\newcounter{prob}[section]
\newcounter{sol}[section]

\newenvironment{prob}[2][]{\refstepcounter{prob}
	{\large\raggedleft\textbf{Problema \ifx&#1&\theprob\else#1\fi:}}\addcontentsline{toc}{section}{Problema \ifx&#1&\theprob\else#1\fi}\par\addvspace{-\parskip}\noindent
}{}

\newenvironment{sol}[2][]{\refstepcounter{sol}\par\medskip
	\noindent \textbf{Solución problema \ifx&#1&\thesol\else#1\fi:} \rmfamily\\
	}{\begin{flushright}
		\(\blacksquare\)
	\end{flushright}
}

\title{Tarea 1}
\author{Nicholas Mc-Donnell, Maximiliano Norbu}

\begin{document}
\begin{minipage}{2.5cm}
    \includegraphics[width=2cm]{../../figures/logo1.jpg}
\end{minipage}
\begin{minipage}{13cm}
    \begin{flushleft}
        \raggedright
        {
            \noindent
            {\sc Pontificia Universidad Católica de Chile\\
                Facultad de Matemáticas\\
                Departamento de Matemática} \smallskip \\
            Primer Semestre de 2019\\
        }
    \end{flushleft}
\end{minipage}

\vspace{2ex}
{\Large \centerline{\bf Tarea 2}}
{\large \centerline{Fundamentos de la Matemática - MAT 2405}}
\centerline{Fecha de Entrega: 2019/04/24}

\begin{flushright}
    Integrantes del grupo:\\
    Nicholas Mc-Donnell, Maximiliano Norbu
\end{flushright}

\section*{Problemas}

\begin{prob}[8pts]
    Sea \(\varphi\) una \(\mathcal{L}\)-fórmula con una variable libre \(x\). Sea \(\mathfrak{M}\) una \(\mathcal{L}\)-estructura. Muestre que \(\mathfrak{M}\models\forall x\varphi\) si y sólo si para toda \(\mathfrak{M}\)-asignación \(i:\{x\}\rightarrow M\) se cumple que \((\mathfrak{M},i)\models\varphi\).
\end{prob}

\begin{sol}
    Debemos probar que, dado $b \in M$: \begin{enumerate}
        \item $\mathfrak{M}'_b \models \varphi (\hat{x}|x)$
        \item $(\mathfrak{M},i_b)\models\varphi$, donde $i_b$: \{$x$\} $\rightarrow M$ es la asignación $i_b(x)=b$
    \end{enumerate}
    son equivalentes.\\
    Para esto basta demostrar el caso de términos y luego (inductivamente) extenderlo a todas las fórmulas posibles.\\
    Para un término t cualquiera, como t tiene solo la variable libre x (enunciado):\begin{enumerate}
        \item t($\hat{x}|x$) no tiene variables libres. Su interpretación en $\mathfrak{M}'_b$ (donde $\mathfrak{M}'_b$ es una $\mathcal{L}'$-estructura y donde $\mathcal{L}'$ es una extensión del lenguaje $\mathcal{L}$ que solo agrega una constante b la cual interpreto de manera natural) está bien definida, pues $t^{\mathfrak{M}'_b} \in M$
        \item Podemos usar $i_b$ para interpretar $x$ y $t^{(\mathfrak{M},i_b)}\in M$

    \end{enumerate}
    Es importante notar que $t^{(\mathfrak{M},i_b)} = t^{\mathfrak{M}'_b}$, o sea, son el mismo elemento en $M$.\\
    Por esto, para el caso de términos, se cumple lo propuesto.\\
    Ahora extenderemos este caso a las fórmulas atómicas. \\
    Si $\varphi$ fuera una fórmula atómica $Rt_1,...,t_n$ ($R$ siendo una relación n-aria y $t_j$ siendo términos con solo una variable libre $x$), entonces dado $b$ $\in M$, de las definiciones obtenemos: \begin{enumerate}
        \item $\mathfrak{M}'_b \models \varphi (\hat{x}|x)$ si y sólo si $(t_1^{\mathfrak{M}'_b},...,t_n^{\mathfrak{M}'_b}) \in R^{\mathfrak{M}'_b}$
        \item $(\mathfrak{M},i_b) \models$ si y solo si $(t_1^{(\mathfrak{M},i_b)},...,t_n^{(\mathfrak{M},i_b})) \in R^{\mathfrak{M}}$
    \end{enumerate}
    Aquí es importante notar que $R^{\mathfrak{M}'_b} = R^{\mathfrak{M}}$. También notar que las interpretaciones de los términos ($t_j$) son las mismas en ambos casos (viendo término a término) por el caso de términos visto anteriormente. Por estas dos cosas se tiene que $\mathfrak{M}'_b \models \varphi (\hat{x}|x)$ si y solo si $(\mathfrak{M},i_b) \models \varphi$.\\
    Ahora, si $\psi$ fuera fórmula y $\varphi = \neg \psi$, claramente $Free(\psi)=Free(\neg \psi)=Free(\varphi)=x$, $t^{\mathfrak{M}'_b}=t^{(\mathfrak{M},i_b)}=b \in M$.\\
    Con esto tenemos: \begin{enumerate}
        \item $\mathfrak{M}'_b \models \varphi (\hat{x}|x) = \neg \psi (\hat{x}|x)$ si y solo si $\mathfrak{M}'_b \nvDash \psi (\hat{x}|x)$
        \item $(\mathfrak{M},i_b) \models \varphi = \neg \psi$ si y solo si $(\mathfrak{M},i_b) \nvDash \psi$
    \end{enumerate}
    Por el paso inductivo tenemos $\mathfrak{M}'_b \models \psi (\hat{x}|x)$ si y solo si $(\mathfrak{M},i_b) \models \psi$ por lo que:\\
    $\mathfrak{M}'_b \models \varphi (\hat{x}|x) = \neg \psi (\hat{x}|x)$ si y solo si $(\mathfrak{M},i_b) \models \varphi = \neg \psi$.\\
    Ahora, sean $\psi$ y $\sigma$ fórmulas con unica variable libre $x$ y sea $\varphi = \psi * \sigma$ con $*$ conectivo binario, claramente $Free(\varphi)=x$. $t^{\mathfrak{M}'_b}=t^{(\mathfrak{M},i_b)}=b \in M$. Con esto tenemos: \begin{enumerate}
        \item $\mathfrak{M}'_b \models \varphi (\hat{x}|x) =  \psi * \sigma (\hat{x}|x)$ si y solo si $\mathfrak{M}'_b \models \psi(\hat{x}|x) * \mathfrak{M}'_b \models \sigma (\hat{x}|x)$
        \item $(\mathfrak{M},i_b) \models \varphi = \psi * \sigma$ si y solo si $(\mathfrak{M},i_b) \models \psi *(\mathfrak{M},i_b) \models \sigma$
    \end{enumerate}
    Por ejemplo, si $\mathfrak{M}'_b \models \varphi (\hat{x}|x) =  \psi \vee \sigma (\hat{x}|x)$ si y solo si $\mathfrak{M}'_b \models \psi(\hat{x}|x)$ o $\mathfrak{M}'_b \models \sigma (\hat{x}|x)$.\\
    También $(\mathfrak{M},i_b) \models \varphi = \psi \vee \sigma$ si y solo si $(\mathfrak{M},i_b) \models \psi$ o $(\mathfrak{M},i_b) \models \sigma$.\\
    O sea, $\mathfrak{M}'_b \models \varphi (\hat{x}|x) =  \psi \vee \sigma (\hat{x}|x)$ si y solo si $(\mathfrak{M},i_b) \models \varphi = \psi \vee \sigma$.
    Por paso inductivo tenemos $\mathfrak{M}'_b \models \varphi (\hat{x}|x) =  \psi * \sigma (\hat{x}|x)$ si y solo $(\mathfrak{M},i_b) \models \varphi = \psi * \sigma$ se cumple pa todos los conectores binarios.\\
    Ahora, sea $\psi$ fórmula, $y$ variable y $\varphi = \mathbb{Q}y\psi$ con $\mathbb{Q}$ cuantificador. Hay dos casos. Caso 1: si $y \notin Free(\psi)$:\\
    Tenemos: \begin{enumerate}
        \item $\mathfrak{M}'_b \models \phi(\hat{x}|x) = \mathbb{Q}y\psi (\hat{x}|x)$ si y solo si $\mathfrak{M}'_b \models \psi(\hat{x}|x)$
        \item $(\mathfrak{M},i_b) \models \phi = \mathbb{Q}y\psi$ si y solo si $(\mathfrak{M},i_b) \models \psi$
    \end{enumerate}
    Entonces $\mathfrak{M}'_b \models \phi(\hat{x}|x)$ si y solo si $(\mathfrak{M},i_b) \models \phi$\\
    Caso 2: si $y \in Free(\psi)$. \\
    Sea $\hat{y}^{\mathfrak{M}''_{bb'}}:=b' \in M$, tenemos $\mathfrak{M}'_b(\hat{y}|y) \models \mathbb{Q}y\psi(\hat{y}|y)$ si y solo si $\mathbb{Q}b' \in M$, $\mathfrak{M}''_{bb'} \models \psi$.\\
    Luego, $\hat{y}^{\mathfrak{M}'_{b'}}:=b' \in M$, tenemos que $(\mathfrak{M},i_b) \models \mathbb{Q}y\psi$ si y solo si $(\mathfrak{M}'_{b'},i_b) \models \psi$\\
    Por hipotesis tenemos que $\mathfrak{M}'_b \models \psi$ si y solo si $(\mathfrak{M},i_b) \models \psi$. Además tenemos que dado $b' \in M$ tenemos que $\mathfrak{M}''_{bb'}$ es la misma que $\mathfrak{M}'_b$ con la siguiente modificación:\\
    $\hat{y}^{\mathfrak{M}''_{bb'}} = \hat{y}^{\mathfrak{M}'_{b'}} = b' \in M$. Esto demuestra que $\mathfrak{M}'_b \models \mathbb{Q}y\psi$ si y solo si $(\mathfrak{M},i_b) \models \mathbb{Q}y\psi$.\\
    Con todo esto, se cumple para todas las fórmulas y obtenemos lo pedido.

\end{sol}

\begin{prob}
    \
    \begin{enumerate}[label=(\alph*)]
        \item (4pts) Con el lenguaje \(\mathcal{L}=\{\dot+,\dot=,E\}\), y la \(\mathcal{L}\)-estructura \(\mathfrak{{M}}=(\set{Z}/p\set{Z},+,=,f)\) donde el símbolo de función unaria \(E\) se interpreta como la función \(f(x)=x^2\) y los otros símbolos se interpretan de la manera usual, muestre que el conjunto \(A=\{\overline{1}\}\subseteq\set{Z}/p\set{Z}\) es definible.
        \item (8pts) Con el lenguaje \(\mathcal{L}=\{\dot+,\dot=\}\) y la \(\mathcal{L}\)-estructura \(\mathfrak{M}=(\set{N}_0,+,=)\) donde los símbolos del lenguaje se interpretan de la manera usual, mostrar que todo subconjunto finito de \(\set{N}_0\) es definible.
    \end{enumerate}
\end{prob}

\begin{sol}
    \begin{enumerate}[label=(\alph*)]
        \item Sea \(\varphi\) la siguiente \(\mathcal{L}\)-fórmula con variable libre \(x\):
              \[(f(x)=x)\wedge(\forall y\neg(x+y=y))\]
              Se puede notar que si \(x\) cumple \(f(x)=x\), entonces es su propio cuadrado en \(\set{Z}/p\set{Z}\), se recuerda que es cuerpo\footnote{\cite{artin2011algebra}}, por lo que no hay divisores de \(0\), entonces \(x^2-x=x(x-\overline{1})=0\) y como es un cuerpo se sabe que \(x=\overline{1}\) ó \(x=\overline{0}\). Luego \(\overline{1}+y\neq y\), para cualquier \(y\), pero \(\overline{0}+y=y\), para cualquier \(y\), por lo que solo \(\overline{1}\) satisface \(\varphi\). Con lo que \(A\) es definible.
        \item Se nota que si existe \(\varphi_n\) \(\mathcal{L}\)-fórmula tal que solo \(n\) lo satisface, entonces un conjunto finito \(A=\{a_1,...,a_k\}\) es definible por la siguiente \(\mathcal{L}\)-fórmula:
              \[
                  \bigvee_{i=1}^k\varphi_{a_i}
              \]
              Se puede notar que \(\varphi_0\) sería la \(\mathcal{L}\)-fórmula  con variable libre \(b\) \(\forall x (x+b=x)\). Luego, se considera la siguiente \(\mathcal{L}\)-fórmula con variable libre \(a\)
              \[
                  \forall x,y ((x+y=a)\implies((\neg(x=y))\wedge(((a=x)\wedge\varphi_0(y|b))\vee((a=y)\wedge\varphi_0(x|b)))))
              \]
              Se va a notar que esta es \(\varphi_1\), ya que solo \(1\) cumple que la única forma de escribirlo en forma de suma es tomándose a si mismo y sumándole el \(0\), también se consideran ambas posibilidades con el orden\footnote{La suma es conmutativa.}. %Explicar de nuevo
              Con estas \(\mathcal{L}\)-fórmulas se tiene lo suficiente para construir \(\varphi_n\) con variable libre \(x\):
              \begin{align*}
                  \exists y ((\underbrace{((\dots(y+y)+\dots)+y)}_{\text{``\(n\)'' \(y\)}}=x)\wedge \varphi_1(y|a))
              \end{align*}
              Esto es suficiente ya que para cada \(\varphi_n\) el \(n\) es fijo, y \(n\) es único número natural que cumple que es la suma de \(n\) unos. Ya construido \(\varphi_n\) por lo mencionado al comienzo, se tiene que todo subconjunto finito \(A\) de \(\set{N}_0\) es definible.
    \end{enumerate}
\end{sol}

\begin{prob}[Bonus]
    Sea \(A=\{p_1,p_2,...\}\) el conjunto de todas las letras proposicionales. Muestre que hay a lo más un conjunto consistente maximal que contiene el conjunta \(A\).
\end{prob}

\begin{sol}
    Sean \(\Delta_1\) y \(\Delta_2\) dos conjuntos consistentes maximales que contienen \(A\). Luego, pasa una de los opciones \(\Delta_1\neq\Delta_2\) ó \(\Delta_1=\Delta_2\). Viendo el primer caso, existe \(\varphi\in\Delta_1\setminus\Delta_2\), luego ya que \(\Delta_2\) es consistente maximal y \(\varphi\notin\Delta_2\), \(\neg\varphi\in\Delta_2\). Sea \(B\subset A\) las letras proposicionales en \(\varphi\), luego \(B\cup\{\varphi\}\subset\Delta_1\) y \(B\cup\{\neg\varphi\}\subset\Delta_2\). Ahora, claramente \(B\cup\{\varphi\}\vdash\varphi\) y por correctitud \(B\cup\{\varphi\}\models\varphi\), similarmente para \(B\cup\{\neg\varphi\}\) entonces existen valuaciones \(V_1,V_2\) que satisfacen cada uno correspondientemente, entonces particularmente satisfacen \(B\) y como \(\varphi\) esta compuesta por los elementos de \(B\) se cumple que \(V_1(\varphi)=V_2(\varphi)\), pero si esto es una contradicción. Con esto se tiene que \(\Delta_1=\Delta_2\) por lo que solo hay un conjunto consistente maximal que contiene \(A\).
\end{sol}

\bibliographystyle{apacite}
\bibliography{Tarea}

\end{document}