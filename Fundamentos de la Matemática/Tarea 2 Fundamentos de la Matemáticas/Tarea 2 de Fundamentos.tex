\documentclass[12pt,letterpaper]{article}
\usepackage[T1]{fontenc}
\usepackage[spanish]{babel}
\usepackage[margin=1in]{geometry}
\usepackage{amsthm, amsmath, amssymb}
\usepackage{mathtools}
\usepackage{setspace}\onehalfspacing
\usepackage[loose,nice]{units}
\usepackage{enumitem}
\usepackage{float}

\usepackage{hyperref}
\usepackage{url}
\usepackage{color,graphicx}
\usepackage{fullpage}
\usepackage{multicol}
\usepackage{tabularx}
\usepackage[natbibapa]{apacite}
\usepackage{titling}

\usepackage{xargs}

\graphicspath{{../figures/}}

\hypersetup{
	colorlinks,
	citecolor=black,
	filecolor=black,
	linkcolor=black,
	urlcolor=black
}

\renewcommand{\d}[1]{\ensuremath{\operatorname{d}\!{#1}}}
\renewcommand{\vec}[1]{\mathbf{#1}}
\newcommand{\set}[1]{\mathbb{#1}}
\newcommand{\func}[5]{#1:#2\xrightarrow[#5]{#4}#3}
\newcommand{\contr}{\rightarrow\leftarrow}
\newcommand{\floor}[1]{\left\lfloor#1\right\rfloor}
\newcommand{\ceil}[1]{\left\lceil#1\right\rceil}
\newcommand{\abs}[1]{\left|#1\right|}
\newcommand{\angled}[1]{\left\langle#1\right\rangle}
\newcommand{\paren}[1]{\left(#1\right)}
\newcommand{\mcm}{\text{mcm }}
\newcommand{\BigO}[2][]{O_{#1}\paren{#2}}
\newcommand{\ds}{\displaystyle}
\newcommand{\cis}{\text{cis }}
\newcommand{\dom}{\text{Dom }}

\newcommand{\nope}{\(\contr\)}

\DeclareMathOperator{\Ima}{Im}

\newcounter{prob}[section]
\newcounter{sol}[section]

\newenvironment{prob}[2][]{\refstepcounter{prob}
	{\large\raggedleft\textbf{Problema \ifx&#1&\theprob\else#1\fi:}}\addcontentsline{toc}{section}{Problema \ifx&#1&\theprob\else#1\fi}\par\addvspace{-\parskip}\noindent
}{}

\newenvironment{sol}[2][]{\refstepcounter{sol}\par\medskip
	\noindent \textbf{Solución problema \ifx&#1&\thesol\else#1\fi:} \rmfamily\\
	}{\begin{flushright}
		\(\blacksquare\)
	\end{flushright}
}

\title{Tarea 1}
\author{Nicholas Mc-Donnell, Maximiliano Norbu}

\begin{document}
\begin{minipage}{2.5cm}
    \includegraphics[width=2cm]{../../figures/logo1.jpg}
\end{minipage}
\begin{minipage}{13cm}
    \begin{flushleft}
        \raggedright
        {
            \noindent
            {\sc Pontificia Universidad Católica de Chile\\
                Facultad de Matemáticas\\
                Departamento de Matemática} \smallskip \\
            Primer Semestre de 2019\\
        }
    \end{flushleft}
\end{minipage}

\vspace{2ex}
{\Large \centerline{\bf Tarea 2}}
{\large \centerline{Fundamentos de la Matemática - MAT 2405}}
\centerline{Fecha de Entrega: 2019/04/24}

\begin{flushright}
    Integrantes del grupo:\\
    Nicholas Mc-Donnell, Maximiliano Norbu
\end{flushright}

\section*{Problemas}

\begin{prob}[8pts]
    Sea \(\varphi\) una \(\mathcal{L}\)-fórmula con una variable libre \(x\). Sea \(\mathfrak{M}\) una \(\mathcal{L}\)-estructura. Muestre que \(\mathfrak{M}\models\forall x\varphi\) si y sólo si para toda \(\mathfrak{M}\)-asignación \(i:\{x\}\rightarrow M\) se cumple que \((\mathfrak{M},i)\models\varphi\).
\end{prob}

\begin{sol}
    
\end{sol}

\begin{prob}
    \
    \begin{enumerate}[label=(\alph*)]
        \item (4pts) Con el lenguaje \(\mathcal{L}=\{\dot+,\dot=,E\}\), y la \(\mathcal{L}\)-estructura \(\mathfrak{{M}}=(\set{Z}/p\set{Z},+,=,f)\) donde el símbolo de función unaria \(E\) se interpreta como la función \(f(x)=x^2\) y los otros símbolos se interpretan de la manera usual, muestre que el conjunto \(A=\{\overline{1}\}\subseteq\set{Z}/p\set{Z}\) es definible.
        \item (8pts) Con el lenguaje \(\mathcal{L}=\{\dot+,\dot=\}\) y la \(\mathcal{L}\)-estructura \(\mathfrak{M}=(\set{N}_0,+,=)\) donde los símbolos del lenguaje se interpretan de la manera usual, mostrar que todo subconjunto finito de \(\set{N}_0\) es definible.
    \end{enumerate}
\end{prob}

\begin{sol}
    
\end{sol}

\begin{prob}[Bonus]
    Sea \(A=\{p_1,p_2,...\}\) el conjunto de todas las letras proposicionales. Muestre que hay a lo más un conjunto consistente maximal que contiene el conjunta \(A\).
\end{prob}

\begin{sol}
    
\end{sol}

\end{document}