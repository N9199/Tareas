\documentclass[12pt,letterpaper]{article}
\usepackage[T1]{fontenc}
\usepackage[spanish]{babel}
\usepackage[margin=1in]{geometry}
\usepackage{amsthm, amsmath, amssymb}
\usepackage{mathtools}
\usepackage{setspace}\onehalfspacing
\usepackage[loose,nice]{units}
\usepackage{enumitem}
\usepackage{float}

\usepackage{hyperref}
\usepackage{url}
\usepackage{color,graphicx}
\usepackage{fullpage}
\usepackage{multicol}
\usepackage{tabularx}
\usepackage[natbibapa]{apacite}
\usepackage{titling}

\usepackage{xargs}

\graphicspath{{../figures/}}

\hypersetup{
	colorlinks,
	citecolor=black,
	filecolor=black,
	linkcolor=black,
	urlcolor=black
}

\renewcommand{\d}[1]{\ensuremath{\operatorname{d}\!{#1}}}
\renewcommand{\vec}[1]{\mathbf{#1}}
\newcommand{\set}[1]{\mathbb{#1}}
\newcommand{\func}[5]{#1:#2\xrightarrow[#5]{#4}#3}
\newcommand{\contr}{\rightarrow\leftarrow}
\newcommand{\floor}[1]{\left\lfloor#1\right\rfloor}
\newcommand{\ceil}[1]{\left\lceil#1\right\rceil}
\newcommand{\abs}[1]{\left|#1\right|}
\newcommand{\angled}[1]{\left\langle#1\right\rangle}
\newcommand{\paren}[1]{\left(#1\right)}
\newcommand{\mcm}{\text{mcm }}
\newcommand{\BigO}[2][]{O_{#1}\paren{#2}}
\newcommand{\ds}{\displaystyle}
\newcommand{\cis}{\text{cis }}
\newcommand{\dom}{\text{Dom }}

\newcommand{\nope}{\(\contr\)}

\DeclareMathOperator{\Ima}{Im}

\newcounter{prob}[section]
\newcounter{sol}[section]

\newenvironment{prob}[2][]{\refstepcounter{prob}
	{\large\raggedleft\textbf{Problema \ifx&#1&\theprob\else#1\fi:}}\addcontentsline{toc}{section}{Problema \ifx&#1&\theprob\else#1\fi}\par\addvspace{-\parskip}\noindent
}{}

\newenvironment{sol}[2][]{\refstepcounter{sol}\par\medskip
	\noindent \textbf{Solución problema \ifx&#1&\thesol\else#1\fi:} \rmfamily\\
	}{\begin{flushright}
		\(\blacksquare\)
	\end{flushright}
}

\title{Tarea 3}
\author{Nicholas Mc-Donnell, Maximiliano Norbu}
\date{2019/06/05}

\pagenumbering{gobble}

\begin{document}
\begin{minipage}{2.5cm}
    \includegraphics[width=2cm]{../../figures/logo1.jpg}
\end{minipage}
\begin{minipage}{13cm}
    \begin{flushleft}
        \raggedright{
            \noindent
            {\sc Pontificia Universidad Católica de Chile\\
                Facultad de Matemáticas\\
                Departamento de Matemática} \smallskip \\
            Primer Semestre de 2019\\
        }
    \end{flushleft}
\end{minipage}

\vspace{2ex}
{\Large \centerline{\bf \thetitle}}
{\large \centerline{Fundamentos de la Matemática --- MAT 2405}}
{\normalsize \centerline{ Fecha de Entrega: \thedate}}
\vfill

\begin{flushright}
    {\large Integrantes del grupo:\\
        \theauthor}
\end{flushright}
\newpage
\normalsize
\pagenumbering{arabic}
\tableofcontents
\newpage

\begin{prob}
    -Dado una relación antisimétrica \(R\neq\emptyset\), muestre que \(R\cap R^{-1}\) es una función.
\end{prob}

\begin{sol}
    -Se puede asumir que \(R\cap R^{-1}\neq\emptyset\), si no, es función por vacuidad. Sea \(\angled{x,y},\angled{x,z}\in R\cap R^{-1}\), entonces \(\angled{y,x},\angled{z,x}\in R\cap R^{-1}\), por esto, también \(\angled{y,x},\angled{z,x}\in R\) como \(R\)  antisimétrica,  \(x=y,x=z\), por lo que \(y=z\). En conclusión, o bien \(R\cap R^{-1}\) una relación vacía o bien \(R\cap R^{-1}\) la relación identidad, en ambos casos, \(R\cap R^{-1}\) es función.
\end{sol}

\begin{prob}
    -Sea \(A\) un conjunto, y sea \(F=\{\angled{x,\angled{x,x}}:x\in A\}\), muestre que \(F\) es función biyectiva entre \(A\) y \(I_A=\{\angled{y,y}:y\in A\}\).
\end{prob}

\begin{sol}
    -Claramente \(F\) es función, se recuerdan las definiciones de inyectividad y de sobreyectividad:
    \[
        \forall x\forall y\forall z\forall w\paren{\paren{\paren{\angled{x,y}\in F}\wedge\paren{\angled{z,w}\in F}}\implies\paren{\paren{x=z}\iff\paren{z=w}}}
    \]
    \[
        \forall y\paren{\paren{y\in I_A}\implies\paren{\exists x\paren{\angled{x,y}\in F}}}
    \]
    Para la primera, si \(\angled{x,y}\) o \(\angled{z,w}\), no pertenecen a \(F\), no hay problema, ya que la función no está definida en esos casos, entonces no hay problema. Si ambas pertenecen a \(F\) 
\end{sol}

\begin{prob}
    -Muestre que dado un conjunto \(A\), un elemento \(a\in A\) y una función
    \[
        f:A\times\omega\rightarrow A,
    \]
    entonces existe una única función \(h:\omega\rightarrow A\) tal que \(h(0)=a\) y que cumple \(h(n^+)=f(h(n),n)\).
\end{prob}

\begin{sol}
    -
\end{sol}

\begin{prob}[Bonus]
    -Sea \(\varphi(x)\) una fórmula del lenguaje de la teoría de conjuntos con única variable libre \(x\). Suponga que \(\emptyset\) verifica \(\varphi(x)\) y que para cada \(a\in\omega\) si \(a\) verifica \(\varphi(x)\), entonces \(a^+\) también verifica \(\varphi(x)\). Demuestre que todo \(b\in\omega\) verifica \(\varphi(x)\).
\end{prob}

\begin{sol}[Bonus]
    -Se nota que solo es necesario demostrar que el siguiente conjunto es inductivo:
    \[
        A=\{a\in\omega:\varphi(a)\}
    \]
    Se nota que \(\emptyset\in A\), ya que \(\emptyset\in\omega\) y \(\emptyset\) verifica \(\varphi(x)\). Luego, se sabe que \(\forall a\paren{a\in A\implies a^+\in A}\), ya que si \(a\in A\), se tiene que \(a\) verifica \(\varphi(x)\), por lo que \(a^+\) también verifica \(\varphi(x)\), pero entonces \(a^+\in A\). Con esto se tiene que \(A\) es un conjunto inductivo, por lo que \(\omega\subseteq A\), y por definición de \(A\) se tiene \(A\subseteq\omega\), entonces \(A=\omega\). Con lo que todo \(b\in\omega\) verifica \(\varphi(x)\).
\end{sol}

\end{document}