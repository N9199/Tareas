\documentclass{homework}
\usepackage{multicol}
\usepackage[dvipsnames]{xcolor}


\title{Tarea 1}
\date{2019-08-30}
\gdate{2do Semestre 2019}
\author{Nicholas Mc-Donnell}
\course{Ecuaciones Diferenciales Ordinarias - MAT2500}


\begin{document}
\maketitle
\thanks{Maximiliano Norbu, Agustín Oyarce, Paulina Vega, Darwin Sanhueza, Francisco Monardes, Luciano Sciaraffia}
\pagenumbering{roman}
\newpage
\tableofcontents
\newpage
\pagenumbering{arabic}

\begin{prob}
    Sean \(\phi_1(t),\phi_2(t),\dots,\phi_{n-1}(t)\) soluciones del sistema homogéneo \(\dot{x}=A(t)x\), y sea \(x\) una solución del sistema no homogéneo \(\dot{x}=A(t)x+g(t)\), donde \(A(t)\) y \(g(t)\) son continuas sobre el intervalo \(I\). Pruebe que \(Z=\det\paren{\phi_1,\dots,\phi_{n-1},x}\) satisface la \textit{Formula no homogénea de Abel}
    \[\dot{Z}=\tr(A(t))Z+\det\paren{\phi_1,\dots,\phi_{n-1},g}\]
\end{prob}

\begin{sol}
    Dado la definición de \(Z\), y la multilinealidad del determinante se tiene lo siguiente:
    \begin{align*}
        \dot{Z}&=\det\paren{\dot{\phi_1},\dots,\phi_{n-1},x}+\dots+\det\paren{\phi_1,\dots,\phi_{n-1},\dot{x}}\\
        &=\det\paren{A(t)\phi_1,\dots,\phi_{n-1},x}+\dots+\det\paren{\phi_1,\dots,\phi_{n-1},A(t)x+g(t)}\\
        &=\det\paren{A(t)\phi_1,\dots,\phi_{n-1},x}+\dots+\det\paren{\phi_1,\dots,\phi_{n-1},A(t)x}+\det\paren{\phi_1,\dots,\phi_{n-1},g(t)}
    \end{align*}
    Por propiedades del determinante se tiene la siguiente identidad para cualquier conjunto de vectores \(v_1,\dots,v_n\):
    \[\det\paren{A(t)v_1,\dots,v_n}+\dots+\det\paren{v_1,\dots,A(t)v_n}=c\cdot\det\paren{v_1,\dots,v_n}\footnotemark\]\footnotetext{El \(c\) no depende del conjunto de vectores}
    Ahora, tomando la base canónica, claramente se ve que \(c=\tr(A(t))\), ya que \(A(t)e_i=A_{i,i}(t)\). Con esto se llega a lo siguiente:
    \[\dot{Z}=\tr(A(t))z+\det\paren{\phi_1,\dots,\phi_{n-1},g}\]
\end{sol}

\begin{prob}
    Sea \(A\) la matriz constante asociada a la EDO homogénea de orden \(n\):
    \[x^{(n)}+q_{n-1}x^{(n-1)}+\dots+q_1\dot{x}+q_0x=0\]
    (tal que el sistema equivalente de primer orden es \(\dot{y}=Ay\), donde \(y=(x,\dot{x},\dots,x^{(n-1)})^T\)).
    \begin{enumerate}[label=(\alph*)]
        \item Demuestre que el polinomio característico de \(A\) es
        \[\chi(z):=\det(zI-A)=z^n+q_{n-1}z^{n-1}+\dots+q_1z+q_0\]
        \item Pruebe que la multiplicidad geométrica de cada valor propio de \(A\) es 1, i.e. cada valor propio de \(A\) es asociado con sólo un bloque de Jordan.
        \item Demuestre que la ecuación, o equivalentemente, el sistema \(\dot{y}=Ay\) es estable si y sólo si todos los valores propios tienen parte real no positiva, y todos los valores propios imaginarios son simples.
    \end{enumerate}
\end{prob}

\begin{sol}
    
\end{sol}

\begin{prob}
    Considere el sistema lineal homogéneo \(\dot{x}=A_bx\), donde \(\ds A_b=\begin{pmatrix}
        b & 3\\-3 & 2
    \end{pmatrix}\).
    \begin{enumerate}[label=(\alph*)]
        \item Encuentre la solución general del sistema cuando \(b=-4\) y \textit{dibuje} el retrato de fase.
        \item Determine los valores de \(b\) para los cuales el origen es, respectivamente, una fuente, una fuente espiral, un sumidero, un sumidero espiral, una silla y un centro.
    \end{enumerate}
\end{prob}

\begin{sol}
    \begin{enumerate}[label=(\alph*)]
        \item Se calcula el polinomio característico \( p_{A_{-4}}(\lambda)=(\lambda+1)^2\), con lo que se tiene que la matriz de Jordan es la siguiente:
        \[J=\begin{pmatrix}
            -1&1\\0&-1
        \end{pmatrix}\]
        Como \(A_{-4}\) es una matriz constante se tiene que la solución general del sistema es la siguiente:
        \[x(t)=S\exp(tJ)S^{-1}x_0\]
        Donde \(x_0\) es la condición inicial y donde \(S\) es la matriz tal que \(A_{-4}=SJS^{-1}\). Luego por lo que se vio en el Teschl se tiene que \(\exp(J)=\begin{pmatrix}
            1/e&1/e\\0&1/e
        \end{pmatrix}\), y calculando \(S=\begin{pmatrix}
            1&-\frac13\\1&0
        \end{pmatrix}\) se tiene que la solución general es de la siguiente manera:
        \begin{equation*}
            x(t)=\exp(t)\begin{pmatrix}
                -\frac2e&\frac3e\\-\frac3e&\frac4e
            \end{pmatrix}
        \end{equation*}
        \item Se ve el polinomio característico \(p_{A_b}(t)=\lambda^2-\lambda(b+2)+9+2b\), se nota que solo si \(b=-2\) se tiene que los valores propios son completamente imaginarios, por lo que el origen es un centro. Para los otros casos se verá cuando los valores propios son completamente reales y cuando son complejos, para esto se verá el signo del discriminante del polinomio característico:
        \begin{equation*}
            \Delta=(b+2)^2-4\cdot(9+2b)=(b+4)(b-8)
        \end{equation*}
        Por lo que para \(b\in(-4,8)\) se tiene que \(\Delta<0\), y en otro caso \(\Delta\geq0\), además se quiere ver cuando la parte real es positiva o negativa, lo cual tiene dos casos, si \(\sqrt{\Delta}\) es imaginario o si es real. Comenzando por el primer caso, se nota que se depende del signo de \(b+2\). En el segundo caso, se nota que si \(b\geq 8\) ambos valores propios son siempre positivos\footnote{\(b+2>\sqrt{\Delta}\)}
    \end{enumerate}
\end{sol}

\begin{prob}
    Encuentre la solución general de
    \begin{enumerate}[label=(\alph*)]
        \item \(\dot{x}=Ax\), donde \(\ds A=\begin{pmatrix}
            5&2&4\\0&1&0\\-8&-1&-7
        \end{pmatrix}\);
        \item \(\ddot{x}+x=2\sin(2t)\)
        \item \(\ddot{x}-2\dot{x}=-x+t-1+2\exp(t)\)
        \item \(t\ddot{x}-2(t+1)\dot{x}+(t+2)x=0\), \(\phi_1(t)=\exp(t)\)
    \end{enumerate}
\end{prob}

\begin{sol}
    \begin{enumerate}[label=(\alph*)]
        \item 
    \end{enumerate}
\end{sol}

\begin{prob}
    Considere el sistema lineal homogéneo \(\dot{x}=A(t)x\), para \(t>0\), donde
    \[A(t)=\begin{pmatrix}
        3/t&-1\\2/t^2&-1/t
    \end{pmatrix}\]
    \begin{enumerate}[label=(\alph*)]
        \item Verifique que
        \[x_1(t)=\begin{pmatrix}
            t^2\\t
        \end{pmatrix}\]
        resuelve el sistema para \(t>0\).
        \item Sea \(x_2(t)\) otra solución, tal que el Wronskiano \(W(t):=\det\paren{x_1,x_2}\) satisface \(W(1)=1\). Encuentre \(W(t)\).
        \item Use el de conocimiento de \(W(t)\) para determinar una posible solución \(x_2\).
        \item Resuelva el problema de valor inicial
        \[\dot{x}=A(t)x+\begin{pmatrix}
            0\\-2t
        \end{pmatrix},\quad t>0\quad\text{ con }\quad x(1)=\begin{pmatrix}
            1\\1
        \end{pmatrix}\]
    \end{enumerate}
\end{prob}

\begin{sol}
    
\end{sol}

\begin{prob}
    \begin{enumerate}[label=(\alph*)]
        \item Dado el sistema \(\dot{x}=A(t)x\), donde \(A(t)\) es continua y periódica con periodo \(T\). Demuestre que la transformación \(y(t)=P(t,t_0)^{-1}x(t)\) traduce el sistema a uno con coeficientes constantes:
        \[\dot{y}=Q(t_0)y\]
        Donde \(\Pi(t,t_0)=P(t,t_0)\exp((t-t_0)Q(t_0))\), 
        \item Considere la EDO no homogénea
        \[\dot{x}=A(t)x+g(t)\]
        donde \(A(t)\) y \(g(t)\) son periódicas de periodo \(T\). Muestre que esta EDO tiene una solución periódica única de periodo \(T\) si y solo si \(1\) no es un valor propio de la matriz de monodronía \(M(t_0)\).
    \end{enumerate}
\end{prob}

\begin{sol}
    
\end{sol}

\begin{prob}
    Considere el sistema homogéneo \(\dot{x}=A(t)x\), donde
    \[A(t)=\begin{pmatrix}
        a(t)&b(t)\\c(t)&d(t)
    \end{pmatrix}\]
    y \(a,b,c,d\) son funciones reales continuas con periodo \(1\). Suponga que \(a(t)>0\) y \(d(t)>0\). Demuestre que para todo entero \(k>2\) el sistema no puede tener una solución periódica \(x(t)\) con periodo mínimo igual a \(k\).
\end{prob}

\begin{sol}
    Usando el cambio de coordenadas del problema 6(a) se nota que el periodo de una solución depende del periodo de \(P(t,t_0)\), pero por el corolario 3.16 y por el teorema de Floquet, se tiene que el periodo de \(P(t,t_0)\) es el mismo o el doble que el de \(A(t)\), por lo que el sistema no puede tener una solución periódica con periodo mayor a \(2\).
\end{sol}

\begin{prob}
    Demuestre que la ecuación lineal
    \[\ddot{x}+(1+\exp(-t))x=0\]
    es estable, es decir, que todas sus soluciones permanecen acotadas para \(t\geq0\)
\end{prob}

\begin{sol}
    Se nota que la EDO se puede escribir de esta forma \((\ddot{x}+x)+\exp(-t)x=0\), más específicamente denotando \(x=\begin{pmatrix}
        x\\y
    \end{pmatrix}\) se tiene lo siguiente:
    \begin{equation*}
        \dot{x}=\begin{pmatrix}
            0&1\\-1&0
        \end{pmatrix}x+\begin{pmatrix}
            0&0\\\exp(-t)&0
        \end{pmatrix}x
    \end{equation*}
    Tomando \(A(t)\) como la primera matrix y \(B(t)\) como la segunda, se nota que \(\norm{B(t)}=\exp(-t)\) y que los valores propios de \(A(t)\) son \(\pm i\), como \(A(t)\in M_{2\times 2}\) todas las multiplicidades de los valores propios son iguales, específicamente son \(1\). Notando que \(\int_0^\infty\norm{B(t)}\d{t}=1<\infty\), por el corolario 3.24 se tiene que toda solución del sistema es estable, por lo que se tiene lo pedido.
\end{sol}

\end{document}