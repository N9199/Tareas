\documentclass{homework}
\usepackage{multicol}
\usepackage[dvipsnames]{xcolor}


\title{Tarea 1}
\date{2019-08-30}
\gdate{2do Semestre 2019}
\author{Nicholas Mc-Donnell}
\course{Ecuaciones Diferenciales Ordinarias - MAT2500}


\begin{document}
\maketitle
\thanks{Maximiliano Norbu, Agustín Oyarce, Paulina Vega, Luciano Sciaraffia}
\pagenumbering{roman}
\newpage
\tableofcontents
\newpage
\pagenumbering{arabic}

\begin{prob}
    Sean \(\phi_1(t),\phi_2(t),\dots,\phi_{n-1}(t)\) soluciones del sistema homogéneo \(\dot{x}=A(t)x\), y sea \(x\) una solución del sistema no homogéneo \(\dot{x}=A(t)x+g(t)\), donde \(A(t)\) y \(g(t)\) son continuas sobre el intervalo \(I\). Pruebe que \(Z=\det\paren{\phi_1,\phi_2,\dots,\phi_{n-1},x}\) satisface la \textit{Formula no homogénea de Abel}
    \[\dot{Z}=\tr(A(t))Z+\det\paren{\phi_1,\phi_2,\dots,\phi_{n-1},g}\]
\end{prob}

\begin{sol}
    
\end{sol}

\begin{prob}
    Sea \(A\) la matriz constante asociada a la EDO homogénea de orden \(n\):
    \[x^{(n)}+q_{n-1}x^{(n-1)}+\dots+q_1\dot{x}+q_0x=0\]
    (tal que el sistema equivalente de primer orden es \(\dot{y}=Ay\), donde \(y=(x,\dot{x},\dots,x^{(n-1)})^T\)).
    \begin{enumerate}[label=(\alph*)]
        \item Demuestre que el polinomio característico de \(A\) es
        \[\chi(z):=\det(zI-A)=z^n+q_{n-1}z^{n-1}+\dots+q_1z+q_0\]
        \item Pruebe que la multiplicidad geométrica de cada valor propio de \(A\) es 1, i.e. cada valor propio de \(A\) es asociado con sólo un bloque de Jordan.
        \item Demuestre que la ecuación, o equivalentemente, el sistema \(\dot{y}=Ay\) es estable si y sólo si todos los valores propios tienen parte real no positiva, y todos los valores propios imaginarios son simples.
    \end{enumerate}
\end{prob}

\begin{sol}
    
\end{sol}

\begin{prob}
    Considere el sistema lineal homogéneo \(\dot{x}=A_bx\), donde \(\ds A_b=\begin{pmatrix}
        b & 3\\-3 & 2
    \end{pmatrix}\).
    \begin{enumerate}[label=(\alph*)]
        \item Encuentre la solución general del sistema cuando \(b=-4\) y \textit{dibuje} el retrato de fase.
        \item Determine los valores de \(b\) para los cuales el origen es, respectivamente, una fuente, una fuente espiral, un sumidero, un sumidero espiral, una silla y un centro.
    \end{enumerate}
\end{prob}

\begin{sol}
    
\end{sol}

\begin{prob}
    Encuentre la solución general de
    \begin{enumerate}[label=(\alph*)]
        \item \(\dot{x}=Ax\), donde \(\ds A=\begin{pmatrix}
            5&2&4\\0&1&0\\-8&-1&-7
        \end{pmatrix}\);
        \item \(\ddot{x}+x=2\sin(2t)\)
        \item \(\ddot{x}-2\dot{x}=-x+t-1+2\exp(t)\)
        \item \(t\ddot{x}-2(t+1)\dot{x}+(t+2)x=0\), \(\phi_1(t)=\exp(t)\)
    \end{enumerate}
\end{prob}

\begin{sol}
    
\end{sol}

\begin{prob}
    Considere el sistema lineal homogéneo \(\dot{x}=A(t)x\), para \(t>0\), donde
    \[A(t)=\begin{pmatrix}
        3/t&-1\\2/t^2&-1/t
    \end{pmatrix}\]
    \begin{enumerate}[label=(\alph*)]
        \item Verifique que
        \[x_1(t)=\begin{pmatrix}
            t^2\\t
        \end{pmatrix}\]
        resuelve el sistema para \(t>0\).
        \item Sea \(x_2(t)\) otra solución, tal que el Wronskiano \(W(t):=\det\paren{x_1,x_2}\) satisface \(W(1)=1\). Encuentre \(W(t)\).
        \item Use el de conocimiento de \(W(t)\) para determinar una posible solución \(x_2\).
        \item Resuelva el problema de valor inicial
        \[\dot{x}=A(t)x+\begin{pmatrix}
            0\\-2t
        \end{pmatrix},\quad t>0\quad\text{ con }\quad x(1)=\begin{pmatrix}
            1\\1
        \end{pmatrix}\]
    \end{enumerate}
\end{prob}

\begin{sol}
    
\end{sol}

\begin{prob}
    \begin{enumerate}[label=(\alph*)]
        \item Dado el sistema \(\dot{x}=A(t)x\), donde \(A(t)\) es continua y periódica con periodo \(T\). Demuestre que la transformación \(y(t)=P(t,t_0)^{-1}x(t)\) traduce el sistema a uno con coeficientes constantes:
        \[\dot{y}=Q(t_0)y\]
        Donde \(\Pi(t,t_0)=P(t,t_0)\exp((t-t_0)Q(t_0))\), 
        \item Considere la EDO no homogénea
        \[\dot{x}=A(t)x+g(t)\]
        donde \(A(t)\) y \(g(t)\) son periódicas de periodo \(T\). Muestre que esta EDO tiene una solución periódica única de periodo \(T\) si y solo si \(1\) no es un valor propio de la matriz de monodronía \(M(t_0)\).
    \end{enumerate}
\end{prob}

\begin{sol}
    
\end{sol}

\begin{prob}
    Considere el sistema homogéneo \(\dot{x}=A(t)x\), donde
    \[A(t)=\begin{pmatrix}
        a(t)&b(t)\\c(t)&d(t)
    \end{pmatrix}\]
    y \(a,b,c,d\) son funciones reales continuas con periodo \(1\). Suponga que \(a(t)>0\) y \(d(t)>0\). Demuestre que para todo entero \(k>2\) el sistema no puede tener una solución periódica \(x(t)\) con periodo mínimo igual a \(k\).
\end{prob}

\begin{sol}
    
\end{sol}

\begin{prob}
    Demuestre que la ecuación lineal
    \[\ddot{x}+(1+\exp(-t))x=0\]
    es estable, es decir, que todas sus soluciones permanecen acotadas para \(t\geq0\)
\end{prob}

\begin{sol}
    
\end{sol}

\end{document}