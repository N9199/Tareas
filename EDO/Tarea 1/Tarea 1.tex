\documentclass{homework}
\usepackage{multicol}

\title{Tarea 1}
\date{2019-08-30}
\gdate{2do Semestre 2019}
\author{Nicholas Mc-Donnell}
\course{Ecuaciones Diferenciales Ordinarias - MAT2500}


\begin{document}
\maketitle
\thanks{Maximiliano Norbu, Agustín Oyarce, Camilo Sánchez, Benjamín Cortez, Felipe Guzmán}
\pagenumbering{roman}
\newpage
\tableofcontents
\newpage
\pagenumbering{arabic}
\begin{prob}
    Transforme las siguientes EDOs en sistemas autónomos de primer orden:
    \begin{enumerate}
        \item \(\ddot{x}+t\sin(\dot{x})=x\).
        \item \(\ddot{x}=-y,\ddot{y}=x\).
    \end{enumerate}
\end{prob}

\begin{sol}
    Se toman los siguientes sistemas autónomos de primer orden, y se nota que son equivalentes a los correspondientes:
    \begin{enumerate}
        \item \(\dot{t}=1\), \(z=\dot{x}\), \(\dot{z}+t\sin(z)=x\)
        \item \(w=\dot{x}\), \(y=-\dot{w}\), \(x=\dot{z}\), \(z=\dot{y}\)
    \end{enumerate}
\end{sol}

\begin{prob}
    Encuentre soluciones a las siguientes EDOs:
    \begin{enumerate}
        \item \(\dot{x}=x(1-x)\)
        \item \(\dot{x}=\sin(t)\exp(x)\)
    \end{enumerate}
\end{prob}

\begin{sol}
    \begin{enumerate}
        \item Notemos que \(x(t)\equiv0\) y \(x(t)\equiv1\) son soluciones, por lo que se puede asumir que \(x\neq1\) y \(x\neq0\) localmente. Usando un poco de álgebra se llega a la siguiente ecuación:
        \begin{equation*}
            \frac{\dot{x}}{x(1-x)}=1
        \end{equation*}
        La cual se puede integrar, quedando lo siguiente:
        \begin{equation*}
            \int_{x(t_0)}^{x(t)}\frac{\d{x}}{x(1-x)}=\int_{t_0}^t\d{s}
        \end{equation*}
        Se nota que \(\frac1{x(1-x)}=\frac1x+\frac1{1-x}\), por lo que la ecuación anterior se ve de la siguiente forma:
        \begin{equation*}
            \ln(x(t))-\ln(x(t_0))-\ln(1-x(t))+\ln(1-x(t_0))=t-t_0
        \end{equation*}
        Con lo que se ve que
        \begin{equation*}
            \frac{x(t)}{x(t_0)}\cdot\frac{1-x(t_0)}{1-x(t)}=\exp(t-t_0)
        \end{equation*}
        Y por un poco de álgebra se tiene lo siguiente:
        \begin{equation*}
            x(t)=\dfrac{\frac{x(t_0)}{1-x(t_0)}\exp(t-t_0)}{1+\frac{x(t_0)}{1-x(t_0)}\exp(t-t_0)}
        \end{equation*}
        La cual es una solución local.
        \item Se nota que la EDO es equivalente a la siguiente:
        \begin{equation*}
            \dot{x}\exp(-x)=\sin(t)
        \end{equation*}
        Por lo que se puede integrar, consiguiendo lo siguiente:
        \begin{equation*}
            \int_{x(t_0)}^{x(t)}\exp(-x)\d{x}=\int_{t_0}^t\sin(s)\d{s}
        \end{equation*}
        Solucionando las integrales y haciendo un poco de álgebra se tiene que:
        \begin{equation*}
            x(t)=-\ln(\cos(t)-\cos(t_0)+\exp(-x(t_0)))
        \end{equation*}
        Lo cual nos da una solución local.
    \end{enumerate}
\end{sol}

\begin{prob}
    Encuentre soluciones a las siguientes EDOs
    \begin{enumerate}
        \item \(\dot{x}=\frac{3x-2t}{t}\)
        \item \(y'=y^2-\frac{y}{x}-\frac1{x^2}\)
        \item \(y'=\frac{y}{x}-\tan({\frac{y}{x}})\)
    \end{enumerate}
\end{prob}

\begin{sol}
    \begin{enumerate}
        \item La EDO correspondiente se puede escribir de la siguiente manera\footnote{Recordando que \(t\neq0\)}:
        \begin{equation*}
            \dot{x}-\frac{3x}t=-2
        \end{equation*}
        Tomando el factor integrante \(\exp\paren{\int_{t_0}^t-3/t\d{t}}\), se llega a la siguiente igualdad:
        \begin{equation*}
            \paren{x\exp\paren{\int_{t_0}^t-3/s\d{s}}}'=-2\exp\paren{\int_{t_0}^t-3/s\d{s}}
        \end{equation*}
        Desarrollando el factor integrante e integrando en ambos lados se consigue lo siguiente:
        \begin{equation*}
            x(t)\cdot\paren{\frac{t_0}{t}}^3-x(t_0)=2t_0^3\paren{\frac1{t^4}-\frac1{t_0^4}}
        \end{equation*}
        Con lo que se ve que las soluciones son de la siguiente forma:
        \begin{equation*}
            x(t)=x(t_0)\cdot\paren{\frac{t}{t_0}}^3+2\paren{\frac1t-\frac{t^3}{t_0^4}}
        \end{equation*}
        Consiguiendo lo pedido.
        \item Para esta EDO se nota que la sustitución de Ricatti funciona, por lo que se necesita una solución particular. Notamos que \(y(x)=\frac1x\) es una solución particular\footnote{\(y'=-\frac1{x^2}=\frac1{x^2}-\frac1x\cdot\frac1x-\frac1{x^2}=y^2-\frac{y}x-\frac1{x^2}\)}, por lo que se usa la sustitución \(u=\frac1{y-\frac1x}\) y algo de álgebra para llegar a la siguiente EDO:
        \begin{equation*}
            u'+\frac{u}x=-1
        \end{equation*}
        La cual se puede solucionar multiplicando por el factor integrante:
        \begin{equation*}
            \paren{u\exp\paren{\int_{x_0}^x\frac1s\d{s}}}'=-\exp\paren{\int_{x_0}^x\frac1s\d{s}}
        \end{equation*}
        Ahora, integrando de nuevo y solucionando el factor integrante se llega a lo siguiente:
        \begin{equation*}
            u(x)\cdot\frac{x}{x_0}-u(x_0)=\frac{x}{x_0}-1
        \end{equation*}
        Con lo que tenemos la forma general de \(u(x)\):
        \begin{equation*}
            u(x)=\frac{x_0}xu(x_0)-\frac{x_0}x+1
        \end{equation*}
        Deshaciendo la sustitución se llega a lo siguiente:
        \begin{equation*}
            y(x)=\frac1x+\frac1{\frac{x_0}xu(x_0)-\frac{x_0}x+1}
        \end{equation*}
        Lo que nos da la forma general de una solución.
        \item Se ve que si se usa la sustitución \(u=\frac{y}{x}\), esto nos simplifica la EDO a lo siguiente:
        \begin{equation*}
            u'x+u=u-\tan(u)
        \end{equation*}
        Si es que \(u=0\), la función idénticamente cero es solución, por lo que se verán soluciones localmente no cero. Con esto se puede resolver la EDO escribiéndola de la siguiente forma:
        \begin{equation*}
            \frac{u'}{\tan(u)}=-\frac1x
        \end{equation*}
        Esto se puede integrar, y reordenar algebraicamente para conseguir esto:
        \begin{equation*}
            u(x)=\arcsin\paren{\frac{x_0}x\sin(u(x_0))}
        \end{equation*}
        Deshaciendo la sustitución, se consigue lo siguiente:
        \begin{equation*}
            y(x)=x\arcsin\paren{\frac{x_0}x\sin\paren{\frac{y(x_0)}{x_0}}}
        \end{equation*}
        Donde \(x\neq0\) e \(y(x_0)\neq0\), con lo que tenemos una solución local.
    \end{enumerate}
\end{sol}

\begin{prob}
    Sean \(\tau>0\) y \(\gamma>0\) constantes. Considere
    \[\dot{x}=\gamma\sqrt{\abs{x}}-\tau x,\quad x(0)=x_0\]
    \begin{enumerate}
        \item Resuelva el problema. (\textit{Sugerencia}: La EDO es de tipo Bernoulli)
        \item Analice la unicidad de la solución, y determine el intervalo máximo de definición. Si hay falla de unicidad, explique porqué esto no contradice el teorema de Picard-Lindelöf.
        \item Analice el comportamiento a largo plazo, \(t\rightarrow\infty\), cuando \(x_0>0\).
    \end{enumerate}
\end{prob}

\begin{sol}
    \begin{enumerate}
        \item Se nota que la EDO es autónoma, por lo que al solucionar el problema localmente para \(t=0\), se soluciona localmente para cualquier \(t_0\). Dado esto que la función \(x(t)\equiv0\) es solución si \(x_0=0\). También se nota que \(f\) es lipschitz con respecto a \(x\) para todo \(x\neq0\), y al ser autónoma es uniformemente continua con respecto \(t\), por lo que dado una condición inicial \(x_0\neq0\) se tiene solución única, por el teorema de Picard-Lindelöf. Sea \(x_0>0\) entonces localmente se puede hacer la sustitución \(x=y^2\), lo que nos da la siguiente EDO:
        \begin{equation*}
            \dot{y}=\frac\gamma2-y\frac\tau2 
        \end{equation*}
        Una EDO separable, por lo que integrando directamente y reordenando se llega a:
        \begin{equation*}
            y(t)=\frac\gamma\tau-\paren{\frac\gamma\tau-x_0^2}\exp\paren{-\frac{\tau\cdot t}2}
        \end{equation*}
        Se recuerda la sustitución que se uso, y se deshace, pero se considera que \(x(0)=x_0>0\)\footnote{Al usar \(x=y^2\) se pierde la noción de positividad, por lo que al deshacer la sustitución, se necesita considerarla.}. Con eso se llega a la presente solución:
        \begin{equation*}
            x(t)=\sqrt{\frac\gamma\tau-\paren{\frac\gamma\tau-x_0^2}\exp\paren{-\frac{\tau\cdot t}2}}
        \end{equation*}
    \end{enumerate}
\end{sol}
\begin{prob}
    Sea \(J\subseteq\set{R}\) un intervalo abierto. Suponga que el problema de valor inicial
    \[\begin{cases}
            \dot{x}=f(t,x) & (t,x)\in J\times\set{R},\quad f\in C(J\times\set{R}) \\
            x(t_0)=x_0
        \end{cases}\]
    tiene una solución \(C^1\) definida localmente en tiempo para todos datos iniciales \((t_0,x_0)\in J\times\set{R}\). Demuestre que si el intervalo máximo de definición de una solución \(x(t)\) es \((T_-,T_+)\subseteq J\), entonces \(\lim_{x\downarrow T_-}\abs{x(t)}=\infty\) y \(\lim_{x\uparrow T_+}\abs{x(t)}=\infty\). (\textit{Sugerencia}: Argumente por contradicción. Primero demuestre que si hay dos sucesiones \(\{a_i\}\) y \(\{b_j\}\) con \(a_i\uparrow T_+\) y \(b_j\uparrow T_+\) tales que \(x(a_i)\rightarrow x_1\) y \(x(b_j)\rightarrow x_2\), entonces \(x_1=x_2\). Úselo para probar que \(x(t)\) se puede extender como una solución después del momento \(T_+\)).
\end{prob}

\begin{sol}

\end{sol}

\begin{prob}
    Considere el problema de valor inicial
    \[\begin{cases}
            \dot{x}=x^3-\exp(t^2)x^2 & (t,x)\in[0,\infty)\times\set{R} \\
            x(0)=\xi
        \end{cases}\]\
    \begin{enumerate}
        \item Identifique y dibuje la 0-isoclina de la ecuación.
        \item Demuestre que para \(\xi\in[0,1]\), la solución \(x(t)\) está definida para todos \(t\geq0\), y que \(\lim_{t\rightarrow\infty}x(t)=0\).
        \item Pruebe que cuando \(\xi\geq K\) es suficientemente grande, \(x(t)\) explota en tiempo finito. (\textit{Sugerencia}: Construya una subsolución de la forma \(y(t)=\exp(t^2)g(t)\) que explota en tiempo finito).
    \end{enumerate}
\end{prob}

\begin{sol}

\end{sol}

\begin{prob}
    Sea \(C\subseteq X\) un subconjunto cerrado del espacio de Banach \(X\). Suponga que para la función \(K:C\rightarrow C\), su \(n\)-ésima iteración \(K^n:C\rightarrow C\) es una contracción. Demuestre que \(K\) tiene un único punto fijo en \(C\).
\end{prob}

\begin{sol}
    Por pto. fijo de Banach, se tiene que \(K^n\) tiene un pto. fijo único, el cual se denotará como \(\overline{x}\), luego
    \begin{align*}
        K^n(K(\overline{x}))&=K^{n+1}(\overline{x})\\
        &=K(K^n(\overline{x}))\\
        &=K(\overline{x})\quad\text{ya que \(\overline{x}\) es pto. fijo de \(K^n\)}
    \end{align*}
    Entonces \(K(\overline{x})\) es pto. fijo de \(K^n\), pero este es único, por lo que \(K(\overline{x})=\overline{x}\). Lo que significa que \(K\) tiene un pto. fijo.
\end{sol}

\begin{prob}
    (La Desigualdad de Gronwall): Suponga que \(\psi(t)\) satisface
    \[\psi(t)\leq\alpha(t)+\int_0^t\beta(s)\psi(s)\d{s},\quad t\in[0,T]\]
    con \(\alpha(t)\in\set{R}\) y \(\beta(t)\geq0\). Entonces
    \[\psi(t)\leq\alpha(t)+\int_0^t\alpha(s)\beta(s)\exp\paren{\int_s^t\beta(r)\d{r}}\d{s},\quad t\in[0,T]\]
    Es más, si además \(\alpha(s)\leq\alpha(t)\) para \(s\leq t\), entonces
    \[\psi(t)\leq\alpha(t)\exp\paren{\int_0^t\beta(s)\d{s}},\quad t\in[0,T].\]\\
    Demuestre la última desigualdad.
\end{prob}

\begin{sol}

\end{sol}
\end{document}