\documentclass[11pt]{article}
    \usepackage[spanish]{babel}
    \usepackage[utf8]{inputenc}
    \usepackage[margin=1in]{geometry}          
    \usepackage{graphicx}
    \usepackage{amsthm, amsmath, amssymb}
    \usepackage{mathtools}
    \usepackage{setspace}\onehalfspacing
    \usepackage[loose,nice]{units} 
    \usepackage{enumitem}
    \usepackage{hyperref}
    \hypersetup{
        colorlinks,
        citecolor=black,
        filecolor=black,
        linkcolor=black,
        urlcolor=black
    }
    
    \setcounter{secnumdepth}{0}
    
    \title{Tarea V}
    \author{Nicholas Mc-Donnell}
    \date{2do semestre 2017}
    
    \renewcommand{\thesection}{}
    \renewcommand{\thesubsection}{}

    \renewcommand{\d}[1]{\ensuremath{\operatorname{d}\!{#1}}}
    \renewcommand{\vec}[1]{\mathbf{#1}}
    \newcommand{\set}[1]{\mathbb{#1}}
    \newcommand{\func}[5]{#1:#2\xrightarrow[#5]{#4}#3}
    \newcommand{\contr}{\rightarrow\leftarrow}
    
    \DeclareMathOperator{\Ima}{Im}
    
    \newtheorem{thm}{Teorema}[section]
    \newtheorem{lem}[thm]{Lema}
    \newtheorem{prop}[thm]{Proposición}
    \newtheorem*{cor}{Corolario}
    
    \theoremstyle{definition}
    \newtheorem{defn}{Definición}[section]
    \newtheorem{obs}{Observación}[section]
    \newtheorem{ejm}[thm]{Ejemplo:}

    \pagenumbering{gobble}

    \begin{document}
        \maketitle
        \newpage

        \pagenumbering{arabic}
        \tableofcontents
        \newpage
        \section{Capítulo 10}
        \subsection{10.5}
        \subsubsection{1}
        Describe the ring obtained from $\set{Z}$ by adjoining an element$\alpha$ satisfying the two relations $2\alpha-6=0$ and $\alpha-10=0$
        \begin{proof}
            Primero recordemos que $\set{Z}[\alpha]\simeq\set{Z}[x]/(x-10,2x-6)$, y con esto veremos algunas propiedades del anillo.
            \[x\equiv 10\quad2(x-3)\equiv0\]
            \[\implies 10\equiv 3\implies 7\equiv 0\]
            Por lo que podemos ver que usando el primer teorema de isomorfismos, se concluye que $\set{Z}[\alpha]\simeq\set{Z}_7$
        \end{proof}

        \subsubsection{7}
        Analyze the ring obtained from $\set{Z}$ by adjoining an element $\alpha$ which satisfies the pair of relations $\alpha^3+\alpha^2+1=0$ and $\alpha^2+\alpha=0$
        \begin{proof}
            Lo primero que notamos es que $\set{Z}[\alpha]\simeq\set{Z}[x]/(x^3+x^2+1,x^2+x)$. Notamos que el ideal $(x^3+x^2+1,x^2+x)$ contiene el 1. Esto implica que $\set{Z}[\alpha]\simeq\set{Z}_0$
        \end{proof}

        \subsubsection{9}
        Describe the ring obtained fro $\set{Z}/12\set{Z}$ by adjoining an inverse of $2$
        \begin{proof}
            Se sabe que adjuntar un inverso a un anillo es equivalente a cocientar de la siguiente forma:
            \[R[a]=R[x]/(2x-1)\]
            Donde $a$ es el inverso del elemento en cuestión. Para este caso en especifico es el inverso de $2$.
            \[\implies\set{Z}_{12}[a]=\set{Z}_{12}[x]/(2x-1)\]
            Usando las propiedades del anillo:
            \[12x\equiv 0\]
            \[x\equiv a\]
            \[\therefore 12a\equiv 0\]
            Pero notamos lo siguiente:
            \[12=6\cdot 2\implies 6\equiv 0\]
            Mas aun:
            \[3\equiv 0\]
            Observamos que $2\cdot 2=4=3+1\equiv 1$
            \[\implies 2\equiv a\]
            Esto nos lleva a que concluir $\set{Z}_{12}[a]\simeq\set{Z}_3$, usando el primer teorema de isomorfismos.
        \end{proof}

        \subsection{10.6 Dominio de Enteros y Cuerpos Fraccionarios}
        \subsubsection{1}
        Prove that the subring of an integral domain is an integral domain.
        \begin{proof}
            Sea $R'\subset R$ anillo, y R dominio.
            \[a,b\in R': ab=0\]
            \[\therefore a,b\in R\implies(a=0\vee b=0)\]
            Lo que implica que $R'$ es dominio.
        \end{proof}

        \subsubsection{3}
        Let $R$ be an integral domain. Prove that the polynomial ring $R[x]$ is an integral domain.
        \begin{proof}
            Demostraremos esto por medio de inducción.\\
            Sean $a,b\in R[x]: ab=0$
            \[\therefore a=\sum^n_{i=0}\alpha_ix^i\quad \forall i:\alpha_i\in R\]
            \[\therefore b=\sum^k_{j=0}\beta_ix^j\quad \forall j:\beta_j\in R\]
            Luego gr$(a)=n$, gr$(b)$=k.\\
            \underline{Caso Base:}\\
            \underline{$n=0,k=0$}
            \[a=\alpha_0,b=\beta_0\]
            \[\implies \alpha_0\beta_0=0\]
            \[\alpha_0,\beta_0\in R\implies \alpha_0=0\vee\beta_0=0\]
            \underline{Caso Inductivo sobre $n$:}\\
            \underline{$n=l,k=0$}
            \[\sum^l_{i=0}\alpha_i\beta_0x^i=0\implies (\forall i\leq l: \alpha_i=0\vee \beta_0=0)\iff (a=0\vee b=0)\]
            \underline{$n=l+1,k=0$}
            \[\sum^{l+1}_{i=0}\alpha_i\beta_0x^i=\sum^l_{i=0}\alpha_i\beta_0x^i+\alpha_{l+1}\beta_0x^{l+1}=0\]
            Pero sabemos que lo primero implica $\forall i\leq l: \alpha_i=0\vee \beta_0=0$. Lo que nos deja con lo siguiente:
            \[\alpha_{l+1}\beta_0x^{l+1}=0\]
            \[\implies \alpha_{l+1}\beta_0=0\]
            Recordamos que ambos coeficientes pertenecen a $R$, por lo que $\alpha_{l+1}=0\vee \beta_0=0$. Vemos que si $\beta_0\neq0\implies \forall i\leq l+1:\alpha_i=0$.
            \[\implies \forall n: a\beta_0=0\iff a=0\vee \beta_0=0\]
            \underline{Caso Inductivo sobre k:}\\
            \underline{$n=m,k=l$}
            \[ab=0\implies(\forall i\leq n:\alpha_i=0\vee\forall j\leq k:\beta_j=0) \iff (a=0\vee b=0)\]
            \underline{$n=m,k=l+1$}
            \[\left(\sum^n_{i=0}\alpha_ix^i\right)\cdot\left(\sum^{l+1}_{j=0}\beta_jx^j\right)=\left(\sum^n_{i=0}\alpha_ix^i\right)\cdot\left(\sum^l_{j=0}\beta_jx^j\right)+\sum^n_{i=0}\alpha_i\beta_{l+1}x^{i+l+1}=0\]
            Sabemos que lo primero implica que $\forall i\leq n:\alpha_i=0\vee\forall j\leq l:\beta_j=0$. Y notamos que lo segundo es una caso similar y equivalente a la inducción sobre $n$. También vemos que si $\beta_{l+1}\neq 0\implies \forall i\leq n:\alpha_i=0$.
            \[\implies(\forall i\leq n:\alpha_i=0\vee\forall j\leq k:\beta_j=0)\iff a=0\vee b=0\]
            Que es lo que queríamos demostrar.
        \end{proof}

        \subsubsection{5}
        Is there an integral domain containing exactly 10 elements?
        \begin{proof}
            Hay dos grupos de orden 10, el dihedral y $\set{Z}_{10}$, notamos que solo $\set{Z}_{10}$ es un grupo abeliano. Vemos los siguientes elementos de $\set{Z}_{10}$:
            \[2\cdot 5=10=0\]
            \[\implies \textrm{No hay dominio de orden 10}\qedhere\]
        \end{proof}

        \subsection{10.7 Ideales Máximos}
        \subsubsection{1}
        Prove that the maximal ideals of the ring of integers are the principal ideals generated by prime integers.

        \subsubsection{3}
        Prove that the ideal $(x+y^2,y+x^2+2xy^2+y^4)$ in $\set{C}[x,y]$ is a maximal ideal

        \subsubsection{7}
        Prove that the ring $\set{F}_2[x]/(x^3+x+1)$ is a field, but that $\set{F}_3[x]/(x^3+x+1)$ is not a field.

        \subsection{10.8 Geometría Algebraica}
        \subsubsection{1}
        Determine the following points of intersection of two complex plane curves in each of the following:
        \begin{enumerate}[label=\textbf{(\alph*)}]
            \item $y^2-x^3+x^2=1, x+y=1$

            \item $x^2+xy+y^2=1, x^2+2y^2=1$

            \item $y^2=x^3,xy=1$

            \item $x+y+y^2=0,x-y+y^2=0$

            \item $x+y^2=0,y+x^2+2xy^2+y^4=0$
        \end{enumerate}

        \subsubsection{5}
        Let $f_1,...,f_r;g_1,...,g_r\in\set{C}[x_1,...,x_n]$, and let $U,V$ be the zeros of $\{f_1,...,f_r\},\{g_1,...,g_r\}$ respectively. Prove that if $U$ and $V$ do not meet, then $(f_1,...,f_r;g_1,...,g_r)$ is the unit ideal.

        \subsubsection{7}
        Prove that the variety defined by a set $\{f_1,...,f_r\}$ of polynomials depends only on the ideal $(f_1,...,f_r)$ that they generate.

        \section{Capítulo 11}
        \subsection{11.1 Factorización de Enteros y Polinomios}
        \subsubsection{3}
        Prove that if $d$ is the greatest common divisor of $a_1,...,a_n$ then the greatest common divisor of $a_1/d,...,a_n/d$ is $1$.
        \begin{proof}
            Asumamos que el gcd de $a_1/d,...,a_n/d$ es $k$, donde $k\neq 1$
            \[\therefore a_i/d=km_i\forall i\]
            \[\implies a_i=kdm_i\forall i\]
            \[\implies gcd(a_1,...,a_n)=kd=d\]
            \[\contr\]
            Esto finaliza la demostración.
        \end{proof}

        \subsubsection{5}
        \begin{enumerate}[label=\textbf{(\alph*)}]
            \item Let $a,b$ be integers with $a\neq 0$, and write $b=aq+r$, where $0\leq r\leq |a|$. Prove that the two greatest common divisors $(a,b)$ and $(a,r)$ are equal.

            \item Describe and algorithm, based on (a), for computing the greatest common divisor.

            \item Use your algorithm to compute the greatest common divisor of the following:
            \begin{enumerate}[label=(\alph*)]
                \item 1456, 235

                \item 123456789, 135792468
            \end{enumerate}
        \end{enumerate}
        \subsubsection{8}
        Factor the following polynomials into irreducible factors in $\set{F}_p[x]$
        \begin{enumerate}[label=\textbf{(\alph*)}]
            \item $x^3+x+1,p=2$

            \item $x^2-3x-3,p=5$

            \item $x^2+1,p=7$
        \end{enumerate} 
    \end{document}