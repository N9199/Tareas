\documentclass[11pt]{article}
    \usepackage[spanish]{babel}
    \usepackage[utf8]{inputenc}
    \usepackage[margin=1in]{geometry}          
    \usepackage{graphicx}
    \usepackage{amsthm, amsmath, amssymb}
    \usepackage{mathtools}
    \usepackage{setspace}\onehalfspacing
    \usepackage[loose,nice]{units} 
    \usepackage{enumitem}
    \usepackage{hyperref}
    \hypersetup{
        colorlinks,
        citecolor=black,
        filecolor=black,
        linkcolor=black,
        urlcolor=black
    }
    
    \setcounter{secnumdepth}{0}
    
    \title{Tarea V}
    \author{Nicholas Mc-Donnell}
    \date{2do semestre 2017}
    
    \renewcommand{\thesection}{}
    \renewcommand{\thesubsection}{}

    \renewcommand{\d}[1]{\ensuremath{\operatorname{d}\!{#1}}}
    \renewcommand{\vec}[1]{\mathbf{#1}}
    \newcommand{\set}[1]{\mathbb{#1}}
    \newcommand{\func}[5]{#1:#2\xrightarrow[#5]{#4}#3}
    \newcommand{\contr}{\rightarrow\leftarrow}
    
    \DeclareMathOperator{\Ima}{Im}
    
    \newtheorem{thm}{Teorema}[section]
    \newtheorem{lem}[thm]{Lema}
    \newtheorem{prop}[thm]{Proposición}
    \newtheorem*{cor}{Corolario}
    
    \theoremstyle{definition}
    \newtheorem{defn}{Definición}[section]
    \newtheorem{obs}{Observación}[section]
    \newtheorem{ejm}[thm]{Ejemplo:}

    \pagenumbering{gobble}

    \begin{document}
        \maketitle
        \newpage

        \pagenumbering{arabic}
        \tableofcontents
        \newpage
        \section{Capítulo 10}
        \subsection{10.5}
        \subsubsection{1}
        Describe the ring obtained from $\set{Z}$ by adjoining an element$\alpha$ satisfying the two relations $2\alpha-6=0$ and $\alpha-10=0$

        \subsubsection{7}
        Analyze the ring obtained from $\set{Z}$ by adjoining an element $\alpha$ which satisfies the pair of relations $\alpha^3+\alpha^2+1=0$ and $\alpha^2+\alpha=0$

        \subsubsection{9}
        Describe the ring obtained fro $\set{Z}/12\set{Z}$ by adjoining an inverse of $2$
        \begin{proof}
            Se sabe que adjuntar un inverso a un anillo es equivalente a cocientar de la siguiente forma:
            \[R'=R[x]/(ax-1)\]
            Donde $a$ es el inverso del elemento en cuestión. Para este caso en especifico es el inverso de $2$
        \end{proof}

        \subsection{10.6 Dominio de Enteros y Cuerpos Fraccionarios}
        \subsubsection{1}
        Prove that the subring of an integral domain is an integral domain.

        \subsubsection{3}
        Let $R$ be an integral domain. Prove that the polynomial ring $R[x]$ is an integral domain.

        \subsubsection{5}
        Is there an integral domain containing exactly 10 elements?
        \begin{proof}
            Hay dos grupos de orden 10, el dihedral y $\set{Z}_{10}$, notamos que solo $\set{Z}_{10}$ es un grupo abeliano. Vemos los siguientes elementos de $\set{Z}_{10}$:
            \[2\cdot 5=10=0\]
            \[\implies \textrm{No hay dominio de orden 10}\]
        \end{proof}

        \subsection{10.7 Ideales Máximos}
        \subsubsection{1}
        Prove that the maximal ideals of the ring of integers are the principal ideals generated by prime integers.

        \subsubsection{3}
        Prove that the ideal $()$ in $\set{C}[x,y]$ is a maximal ideal

        \subsubsection{7}
        Prove that the ring $\set{F}_2/$ is a field, but that $\set{F}_3/$ is not a field.

        \subsection{10.8 Geometría Algebraica}
        \subsubsection{1}
        Determine the following points of intersection of two complex plane curves in each of the following:
        \begin{enumerate}[label=\textbf{(\alph*)}]
            \item

            \item

            \item

            \item

            \item
        \end{enumerate}

        \subsubsection{5}
        Let $f_1,...,f_r;g_1,...,g_r\in\set{C}[x_1,...,x_n]$, and let $U,V$ be the zeros of $\{f_1,...,f_r\},\{g_1,...,g_r\}$ respectively. Prove that if $U$ and $V$ do not meet, then $(f_1,...,f_r;g_1,...,g_r)$ is the unit ideal.

        \subsubsection{7}
        Prove that the variety defined by a set $\{f_1,...,f_r\}$ of polynomials depends only on the ideal $(f_1,...,f_r)$ that they generate.

        \section{Capítulo 11}
        \subsection{11.1 Factorización de Enteros y Polinomios}
        \subsubsection{3}
        Prove that if $d$ is the greatest common divisor of $a_1,...,a_n$ then the greatest common divisor of $a_1/d,...,a_n/d$ is $1$.

        \subsubsection{5}

        \subsubsection{8}
        Factor the following polynomials into irreducible factors in $\set{F}_p[x]$
    \end{document}