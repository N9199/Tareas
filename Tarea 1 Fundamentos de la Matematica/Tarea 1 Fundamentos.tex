\documentclass[12pt,letterpaper]{article}
\usepackage[T1]{fontenc}
\usepackage[spanish]{babel}
\usepackage[margin=1in]{geometry}
\usepackage{amsthm, amsmath, amssymb}
\usepackage{mathtools}
\usepackage{setspace}\onehalfspacing
\usepackage[loose,nice]{units}
\usepackage{enumitem}
\usepackage{float}

\usepackage{hyperref}
\usepackage{url}
\usepackage{color,graphicx}
\usepackage{fullpage}
\usepackage{multicol}
\usepackage{tabularx}
\usepackage[natbibapa]{apacite}
\usepackage{titling}

\usepackage{xargs}

\graphicspath{{../figures/}}

\hypersetup{
	colorlinks,
	citecolor=black,
	filecolor=black,
	linkcolor=black,
	urlcolor=black
}

\renewcommand{\d}[1]{\ensuremath{\operatorname{d}\!{#1}}}
\renewcommand{\vec}[1]{\mathbf{#1}}
\newcommand{\set}[1]{\mathbb{#1}}
\newcommand{\func}[5]{#1:#2\xrightarrow[#5]{#4}#3}
\newcommand{\contr}{\rightarrow\leftarrow}
\newcommand{\floor}[1]{\left\lfloor#1\right\rfloor}
\newcommand{\ceil}[1]{\left\lceil#1\right\rceil}
\newcommand{\abs}[1]{\left|#1\right|}
\newcommand{\angled}[1]{\left\langle#1\right\rangle}
\newcommand{\paren}[1]{\left(#1\right)}
\newcommand{\mcm}{\text{mcm }}
\newcommand{\BigO}[2][]{O_{#1}\paren{#2}}
\newcommand{\ds}{\displaystyle}
\newcommand{\cis}{\text{cis }}
\newcommand{\dom}{\text{Dom }}

\newcommand{\nope}{\(\contr\)}

\DeclareMathOperator{\Ima}{Im}

\newcounter{prob}[section]
\newcounter{sol}[section]

\newenvironment{prob}[2][]{\refstepcounter{prob}
	{\large\raggedleft\textbf{Problema \ifx&#1&\theprob\else#1\fi:}}\addcontentsline{toc}{section}{Problema \ifx&#1&\theprob\else#1\fi}\par\addvspace{-\parskip}\noindent
}{}

\newenvironment{sol}[2][]{\refstepcounter{sol}\par\medskip
	\noindent \textbf{Solución problema \ifx&#1&\thesol\else#1\fi:} \rmfamily\\
	}{\begin{flushright}
		\(\blacksquare\)
	\end{flushright}
}

\title{Tarea 1}
\author{Nicholas Mc-Donnell, Maximiliano Norbu}

\begin{document}
\begin{minipage}{2.5cm}
    \includegraphics[width=2cm]{../figures/logo1.jpg}
\end{minipage}
\begin{minipage}{13cm}
    \begin{flushleft}
        \raggedright
        {
            \noindent
            {\sc Pontificia Universidad Católica de Chile\\
                Facultad de Matemáticas\\
                Departamento de Matemática} \smallskip \\
            Primer Semestre de 2019\\
        }
    \end{flushleft}
\end{minipage}

\vspace{2ex}
{\Large \centerline{\bf Tarea 1}}
{\large \centerline{Fundamentos de la Matemática - MAT 2405}}
\centerline{Fecha de Entrega: 2019/03/27}

\begin{flushright}
    Integrantes del grupo:\\
    Nicholas Mc-Donnell, Maximiliano Norbu
\end{flushright}

\section*{Problemas}

\begin{prob}[15 pts]
    \
    \begin{enumerate}[label=(\alph*)]
        \item (5 pts) Dadas oraciones $\alpha$ y $\beta$, muestre que $(\alpha \iff \beta)$ y $((\alpha \implies \beta) \wedge (\beta \implies \alpha))$ son lógicamente equivalentes.
        \item (10 pts) Demuestre por inducción en oraciones que toda oración es lógicamente equivalente a alguna oración que no tiene el símbolo $\iff$.
    \end{enumerate}
\end{prob}

\begin{sol}
    \begin{enumerate}[label=(\alph*)]
        \item Viendo la siguiente tabla con todas las valuaciones posibles se nota que son lógicamente equivalentes pues ambas siempre tienen el mismo valor de verdad.\\
              \begin{tabular}{|c|c|c|c|c|c|}
                  \hline
                  $\alpha$ & $\beta$ & $\alpha\implies\beta$ & $\beta\implies\alpha$ & $\alpha\iff\beta$ & $((\alpha \implies \beta) \wedge (\beta \implies \alpha))$ \\
                  \hline
                  \hline
                  V        & V       & V                     & V                     & V                 & V                                                          \\
                  V        & F       & F                     & V                     & F                 & F                                                          \\
                  F        & V       & V                     & F                     & F                 & F                                                          \\
                  F        & F       & V                     & V                     & V                 & V                                                          \\
                  \hline
              \end{tabular}
        \item 
    \end{enumerate}
\end{sol}

\begin{prob}[10 pts]
    \
    \begin{enumerate}[label=(\alph*)]
        \item (5 pts) Demuestre que si $\Sigma$ es un conjunto no vacío de oraciones que cumple ambos $\Sigma \models \varphi$ y $\Sigma \models \neg \varphi$, entonces $\Sigma$ no es satisfacible.
        \item (5 pts) ¿Es el conjunto vacío $\phi$ satisfacible? ¿Se cumplen $\phi \models \varphi$ y/o $\phi \models \neg \varphi$?
    \end{enumerate}
\end{prob}

\begin{sol}
    \begin{enumerate}[label=(\alph*)]
        \item Se asume que $\Sigma$ es satisfacible, entonces existe una valuación $\mathcal{V}$ tal que toda oración en $\Sigma$ sea verdad. Pero si $\varphi$ es verdad, entonces $\neg\varphi$ es falso, ahora $\Sigma\models\neg\varphi$, por lo que $\neg\varphi$ es verdad, pero una oración no puede ser verdadera y falsa ya que una valuación solo puede dar un valor para cada oración. Con esto se concluye que $\Sigma$ no es satisfacible.
        \item Por definición, ya que $\emptyset$ no tiene oraciones se cumple que para toda valuación \underline{todas} las oraciones de $\emptyset$ son verdad. Ahora si $\emptyset\models\varphi$, significa que para toda valuación
    \end{enumerate}
\end{sol}

\begin{prob}[5 pts]
    Sea $\alpha$ oración. Encuentre una derivación $\neg \neg \alpha$ a partir del conjunto $\Delta=\{\alpha\}$ utilizando los axiomas y regla de deducción vistas en clase.
\end{prob}

\begin{sol}
    \begin{align*}
        \varphi_1 & = \alpha                                                                                                                                                                           \\
        \varphi_2 & = (\neg \alpha \implies ((c \implies \neg \alpha )\implies \neg \alpha))                                                                                                    & (A1) \\
        \varphi_3 & = ((\neg \alpha \implies ((c \implies \neg \alpha)\implies \neg \alpha))\implies ((\neg \alpha \implies (c \implies \neg \alpha))\implies(\neg \alpha \implies \neg \alpha))) & (A2) \\
        \varphi_4 & = ((\neg \alpha \implies (c \implies \neg \alpha))\implies (\neg \alpha \implies \neg \alpha))                                                                                & (MP) \\
        \varphi_5 & = (\neg \alpha \implies (c \implies \neg \alpha))                                                                                                                           & (A1) \\
        \varphi_6 & = (\neg \alpha \implies \neg \alpha)                                                                                                                                        & (MP) \\
        \varphi_7 & = ((\neg \alpha \implies \neg \alpha) \implies (\alpha \implies \neg \neg \alpha))                                                                                            & (A9) \\
        \varphi_8 & = (\alpha \implies \neg \neg \alpha)                                                                                                                                        & (MP) \\
        \varphi_9 & = \neg \neg \alpha                                                                                                                                                          & (MP)
    \end{align*}
\end{sol}

\begin{prob}
    Bonus
    \\
    Encuentre una derivación de la oración $\beta$  partir del conjunto $\{\neg \neg \beta\}$ utilizando los axiomas y regla de deducción vistas en clase.
\end{prob}

\begin{sol}
    Dado $\neg\neg\alpha$
    \begin{align*}
        \varphi_1    & = \neg\neg\alpha                                                                                                                                 \\
        \varphi_2    & = (\neg\neg\alpha\implies((x\implies x)\implies \neg\neg\alpha))                                                                           & (A1)  \\
        \varphi_3    & = ((x\implies x)\implies\neg\neg\alpha)                                                                                                    & (MP)  \\
        \varphi_4    & = (((x\implies x)\implies\neg\neg\alpha)\implies(\neg\alpha\implies\neg(x\implies x)))                                                     & (A9)  \\
        \varphi_5    & = (\neg\alpha\implies\neg(x\implies x))                                                                                                    & (MP)  \\
        \varphi_6    & = (\neg(x\implies x)\implies\alpha)                                                                                                        & (A10) \\
        \varphi_7    & = ((\neg(x\implies x)\implies\alpha)\implies(\neg\alpha\implies(\neg(x\implies x)\implies\alpha)))                                         & (A1)  \\
        \varphi_8    & = (\neg\alpha\implies(\neg(x\implies x)\implies\alpha))                                                                                    & (MP)  \\
        \varphi_9    & = ((\neg\alpha\implies(\neg(x\implies x)\implies\alpha))\implies((\neg\alpha\implies\neg(x\implies x))\implies(\neg\alpha\implies\alpha))) & (A2)  \\
        \varphi_{10} & = ((\neg\alpha\implies\neg(x\implies x))\implies(\neg\alpha\implies\alpha))                                                                & (MP)  \\
        \varphi_{11} & = (\neg\alpha\implies\alpha)                                                                                                               & (MP)  \\
        \varphi_{12} & = (\alpha\vee\neg\alpha)                                                                                                                   & (A11) \\
        \varphi_{13} & = ((\alpha\implies\alpha)\implies((\neg\alpha\implies\alpha)\implies((\alpha\vee\neg\alpha)\implies\alpha)))                             & (A5)  \\
        \varphi_{14} & = (\alpha\implies((c\implies\alpha)\implies\alpha))                                                                                        & (A1)  \\
        \varphi_{15} & = ((\alpha\implies((c\implies\alpha)\implies\alpha))\implies((\alpha\implies(c\implies\alpha))\implies(\alpha\implies\alpha)))             & (A2)  \\
        \varphi_{16} & = ((\alpha\implies(c\implies\alpha))\implies(\alpha\implies\alpha))                                                                        & (MP)  \\
        \varphi_{17} & = (\alpha\implies(c\implies\alpha))                                                                                                      & (A1)  \\
        \varphi_{18} & = (\alpha\implies\alpha)                                                                                                                   & (MP)  \\
        \varphi_{19} & = ((\neg\alpha\implies\alpha)\implies((\alpha\vee\neg\alpha)\implies\alpha))                                                             & (MP)  \\
        \varphi_{20} & = ((\alpha\vee\neg\alpha)\implies\alpha)                                                                                                   & (MP)  \\
        \varphi_{21} & = \alpha                                                                                                                                 & (MP)
    \end{align*}
\end{sol}
\end{document}