\documentclass{homework}
\usepackage{multicol}
\usepackage[dvipsnames]{xcolor}


\title{Tarea 1}
\date{2019-09-06}
\gdate{2do Semestre 2019}
\author{Nicholas Mc-Donnell}
\course{Variable Compleja - MAT2705}

\begin{document}
\maketitle
\pagenumbering{roman}
\newpage
\tableofcontents
\newpage
\pagenumbering{arabic}
\begin{prob}
    \begin{enumerate}[label=(\alph*)]
        \item Grafique la imagen bajo proyección estereográfica de los siguientes conjuntos
              \begin{enumerate}[label=\roman*.]
                  \item El hemisferio inferior: \(\{(x,y,z)\in\set{S}^2:z<0\}\).
                  \item \(\{(x,y,z)\in\set{S}^2:\frac34\leq z\leq 1\}\).
                  \item Un circulo de la forma \(\{(\sqrt{1-z_0^2}\cos\theta,\sqrt{1-z_0^2}\sin\theta,z_0)\in\set{S}^2:\theta\in[0,2\pi)\}\) con \(z_0\) fijo.
                  \item Un circulo de la forma \(\{(\sqrt{1-z^2}\cos\theta_0,\sqrt{1-z^2}\sin\theta_0,z)\in\set{S}^2:z\in[-1,1]\}\) con \(\theta_0\) fijo.
              \end{enumerate}
        \item Demuestre que la inversión \(\frac1x\) es equivalente a una rotación de la esfera en \(\pi\) radianes alrededor del eje \(x\).
    \end{enumerate}
\end{prob}

\begin{sol}

\end{sol}

\begin{prob}
    Encuentre mapeos conformes entre las siguientes regiones:
    \begin{enumerate}[label=(\alph*)]
        \item \(\{z\in\set{C}:\abs{z}<1\}\) y \(\{z\in\set{C}:\abs{z}>1\}\).
        \item \(\{r\exp(i\theta)\in\set{C}:\theta\in(0,\frac\pi{n}),r\in\set{R}\}\) y \(\set{C}\), con \(n\in\set{N}\setminus\{0\}\).
        \item \(\{z\in\set{C}:\abs{z}>1\}\) y \(\{z\in\set{C}:\Im(z)>0\}\).
        \item \(\{z\in\set{C}:\Re(z)>0\}\) y \(\{z\in\set{C}:\Im(z)\in(-\frac\pi2,\frac\pi2)\}\).
        \item \(\{z\in\set{C}:\abs{z}>1\}\) y \(\{z\in\set{C}:\Im(z)\in(a,b)\}\)
    \end{enumerate}
\end{prob}

\begin{sol}
    \begin{enumerate}
        \item Se recuerda que la inversión es un mapeo conforme, y que mapea las regiones pedidas.
        \item Se recuerda que las funciones analíticas son mapeos conformes, se toma \(f(x)=x^{n+1}\) se nota que cumple lo pedido.
        \item 
    \end{enumerate}
\end{sol}

\begin{prob}
    Sea \(h:[0,1]\rightarrow\set{C}\) continua y se define en \(\set{C}\setminus[0,1]\) la función
    \[H(z)=\int_0^1\frac{h(t)}{t-z}\d{t}.\]
    Demuestre que \(H\) es analítica y calcule por definición su derivada.
\end{prob}

\begin{sol}
    Se nota que si para cada \(z_0\in\set{C}\setminus[0,1]\), se tiene que
    \begin{equation}
        \lim_{z\rightarrow z_0}\abs{\frac{H(z)-H(z_0)}{z-z_0}-\int_0^1\frac{h(t)}{(t-z_0)^2}\d{t}}=0\label{limHz}
    \end{equation}
    Entonces, \(H(z)\) es analítica. Luego, se nota que dado \(z_0\in\set{C}\setminus[0,1]\), se tiene \(0<\inf_{t\in[0,1]}\abs{t-z_0}=\gamma\). Dado esto, se tiene que si \(\abs{z-z_0}<\gamma/2\) entonces \(\abs{z-t}>\gamma/2\) para \(t\in[0,1]\). Ahora desarrollando la siguiente expresión:
    \begin{align*}
        \abs{\frac{H(z)-H(z_0)}{z-z_0}-\int_0^1\frac{h(t)}{(t-z_0)^2}\d{t}}&=\abs{\int_0^1\frac{h(t)(t-z-t+z_0)}{(z-z_0)(t-z)(t-z_0)}-\int_0^1\frac{h(t)}{(t-z_0)^2}\d{t}}\\
        &=\abs{\int_0^1\frac{h(t)}{t-z_0}\cdot\paren{\frac1{t-z}-\frac1{t-z_0}}\d{t}}\\
        &=\abs{\int_0^1\frac{h(t)}{t-z_0}\cdot\paren{\frac{z_0-z}{(t-z)(t-z_0)}}\d{t}}\\
        &\leq\int_0^1\abs{\frac{h(t)}{(t-z_0)^2}\cdot\frac1{t-z}\cdot(z-z_0)}\d{t}\\
        &\leq\int_0^1\abs{\frac{h(t)}{(t-z_0)^2}\cdot\frac2\gamma}\d{t}\cdot\abs{z-z_0}
    \end{align*}
    Con lo que claramente el límite en \eqref{limHz} es cero.
\end{sol}

\begin{prob}
    Considere un dominio \(D\) y \(f:D\rightarrow\set{C}\) analítica. Se define \(D^*=\{\overline{z}:z\in D\}\) y \(g(z)=\overline{f(\overline{z})}\) para \(z\in D^*\). Demuestre que \(g\) analítica y calcule su derivada.
\end{prob}

\begin{sol}

\end{sol}

\begin{prob}
    Considere \(f=u+iv\) analítica. Demuestre que
    \begin{enumerate}[label=(\alph*)]
        \item \(\abs{\nabla u}=\abs{\nabla v}=\abs{f'}\)
        \item \(\angled{\nabla u,\nabla v}=0\)
    \end{enumerate}
\end{prob}

\begin{sol}

\end{sol}

\begin{prob}
    Sea \(f:D\rightarrow\set{C}\) inyectiva y analítica. Demuestre que
    \[\text{Area}(f(D))=\int\int_D\abs{f'(z)}^2\d{x}\d{y}\]
\end{prob}

\begin{sol}

\end{sol}

\begin{prob}
    Decimos que una función \(f:D\subset\set{C}\rightarrow\set{C}\) es armónica si \(\Re(f)\) e \(\Im(f)\) son armónicas. Demuestre que si  \(h\) y \(zh\) son armónicas, entonces \(h\) es analítica.
\end{prob}

\begin{sol}
    Se recuerda que si una función \(f\) es armónica entonces \(\Delta f=0\), o equivalentemente, \(\ds\sum_{i=1}^n\frac{\partial^2f}{\partial x_i^2}=0\). Sea \(h=u+iv\) una función armónica tal que \(zh\) también lo sea, entonces \(\frac{\partial^2 u}{\partial x^2}+\frac{\partial^2 u}{\partial y^2}=0\) y \(\frac{\partial^2 v}{\partial x^2}+\frac{\partial^2 v}{\partial y^2}=0\), además se tiene lo siguiente:
    \begin{align*}
        0&=\frac{\partial^2 (ux-vy)}{\partial x^2}+\frac{\partial^2 (ux-vy)}{\partial y^2}\\
        &=x\frac{\partial^2 u}{\partial x^2}+\frac{\partial^2 u}{\partial x^2}+\frac{\partial u}{\partial x}-y\frac{\partial^2 v}{\partial x^2}-y\frac{\partial^2 v}{\partial y^2}-\frac{\partial^2 v}{\partial y^2}-\frac{\partial v}{\partial y}+x\frac{\partial^2 u}{\partial y^2}\\
        &=\frac{\partial^2 u}{\partial x^2}-\frac{\partial^2 v}{\partial y^2}+\frac{\partial u}{\partial x}-\frac{\partial v}{\partial y}
    \end{align*}
    Similarmente:
    \begin{align*}
        0&=\frac{\partial^2 (uy+vx)}{\partial x^2}+\frac{\partial^2 (uy+vx)}{\partial y^2}\\
        &=y\frac{\partial^2 u}{\partial x^2}+x\frac{\partial^2 v}{\partial x^2}+\frac{\partial^2 v}{\partial x^2}+\frac{\partial v}{\partial x}+\frac{\partial^2 (uy)}{\partial y^2}+x\frac{\partial^2 v}{\partial y^2}
    \end{align*}
\end{sol}

\end{document}
