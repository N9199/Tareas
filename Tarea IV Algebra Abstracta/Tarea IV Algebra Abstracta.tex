\documentclass[11pt]{article}
    \usepackage[spanish]{babel}
    \usepackage[utf8]{inputenc}
    \usepackage[margin=1in]{geometry}          
    \usepackage{graphicx}
    \usepackage{amsthm, amsmath, amssymb}
    \usepackage{mathtools}
    \usepackage{setspace}\onehalfspacing
    \usepackage[loose,nice]{units} 
    \usepackage{enumitem}
    \usepackage{hyperref}
    \hypersetup{
        colorlinks,
        citecolor=black,
        filecolor=black,
        linkcolor=black,
        urlcolor=black
    }
    
    \setcounter{secnumdepth}{0}
    
    \title{Tarea IV}
    \author{Nicholas Mc-Donnell}
    \date{2do semestre 2017}
    
    \renewcommand{\thesection}{}
    \renewcommand{\thesubsection}{}

    \renewcommand{\d}[1]{\ensuremath{\operatorname{d}\!{#1}}}
    \renewcommand{\vec}[1]{\mathbf{#1}}
    \newcommand{\set}[1]{\mathbb{#1}}
    \newcommand{\func}[5]{#1:#2\xrightarrow[#5]{#4}#3}
    \newcommand{\contr}{\rightarrow\leftarrow}
    
    \DeclareMathOperator{\Ima}{Im}
    
    \newtheorem{thm}{Teorema}[section]
    \newtheorem{lem}[thm]{Lema}
    \newtheorem{prop}[thm]{Proposición}
    \newtheorem*{cor}{Corolario}
    
    \theoremstyle{definition}
    \newtheorem{defn}{Definición}[section]
    \newtheorem{obs}{Observación}[section]
    \newtheorem{ejm}[thm]{Ejemplo:}

    \pagenumbering{gobble}

    \begin{document}
        \maketitle
        \newpage

        \pagenumbering{arabic}
        \tableofcontents
        \newpage
        \section{1. Definiciones de un Anillo}
        \subsection{3}
        Let $\alpha=\frac{1}{2}i$. Prove that the elements of $\set{Z}[\alpha]$ form a dense subset of the complex plane.
        \begin{proof}
            Recordemos que $a\in\set{Z}[\alpha]\implies a=\sum^n_{i=0}a_i\alpha^i\quad a_i\in\set{Z}$. Notemos que los siguientes elementos pertenecen a $\set{Z}[\alpha]$
            \[\frac{1}{2^n}\textrm{ y }\frac{1}{2^n}i\]
            Estos elementos se construyen de la siguiente forma:
            \[k=\left\lfloor\frac{n}{4}\right\rfloor+1\]
            \[\therefore\frac{1}{2^{4k}}=\alpha^{4k}\]
            Luego $n=4k-l, 1\leq l\leq 4$
            \[\implies\frac{1}{2^n}=\frac{1}{2^{4k}}\cdot2^l\]
            Similarmente:
            \[\frac{1}{2^n}i=\frac{1}{2^{4k+1}}i\cdot 2^{l+1}\]
            Por lo que:
            \[\set{Z}[\alpha]=\{a+bi:a,b\in\set{Z}[1/2]\}\]
            Recordemos que $\set{Z}[\frac{1}{2}]$ es denso en $\set{R}$, por lo que $\set{Z}[\alpha]$ es denso en $\set{C}$.

        \end{proof}

        \subsection{5}
        Prove that for all integers $n$, $\cos(2\pi/n)$ is an algebraic number.
        \begin{proof}
            Consideremos el siguiente polinomio en $\set{Z}[x]$:
            \[x^n-1\]
            Notamos que tiene la siguiente raíz:
            \[\cos(2\pi/n)+i\sin(2\pi/n)\]
            Por lo que también tiene la siguiente raíz:
            \[\cos(2\pi/n)-i\sin(2\pi/n)\]
            Sabemos que la suma de dos números algebraicos es un  número algebraico.
            \[\implies 2\cos(2\pi/n)\textrm{ es algebraico}\]
            Por lo que $\cos(2\pi/n)$ es algebraico
        \end{proof}

        \subsection{11}
        Describe the group of units in each ring.
        \begin{enumerate}[label=\textbf{(\alph*)}]
            \item $\set{Z}/12\set{Z}$

            \item $\set{Z}/7\set{Z}$

            \item $\set{Z}/8\set{Z}$

            \item $\set{Z}/n\set{Z}$
        \end{enumerate}
        \begin{proof}
            Recordamos que por Intro a Algebra en $\set{Z}/n\set{Z}$ los únicos elementos con inverso multiplicativo son los que son coprimos a $n$.
        \end{proof}
        \begin{enumerate}[label=\textbf{(\alph*)}]
            \item $\{1,5,7,11\}$

            \item $\{1,2,3,4,5,6\}$

            \item $\{1,3,5,7\}$

            \item $\{a\in\set{Z}:0\leq a<n,\textrm{mcd}(a,n)=1\}$
        \end{enumerate}
        \subsection{13}
        An element $x$ of a ring $R$ is called \textit{nilpotent} if some power of $x$ is zero. Prove that if $x$ is nilpotent, then $1+x$ is a unit in $R$.
        \begin{proof}
            Notamos que es suficiente construir el inverso:
            \[a=\sum^{n-1}_{i=0}(-1)^ix^i\]
            Donde $x^n=0$, notemos lo siguiente:
            \[a\cdot (1+x)=a+ax\]
            \[a\cdot (1+x)=\sum^{n-1}_{i=0}(-1)^ix^i+\sum^{n-1}_{i=0}(-1)^ix^{i+1}=1+x^n+\sum^{n-1}_{i=1}(-1)^ix^i+(-1)^{i-1}x^i\]
            Por lo que:
            \[a\cdot (1+x)=1+x^n=1\qedhere\]
        \end{proof}

        \subsection{14}
        Prove that the product set $R\times R'$ of two rings is a ring with component-wise addition and multiplication:
        \[(a,a')+(b,b')=(a+b,a'+b')\textrm{ and }(a,a')(b,b')=(ab,a'b')\]
        \begin{proof}
            Primero recordemos que el producto de grupos es un grupo, por lo que $R^+\times R'^+$ es un grupo con la suma. Luego veamos las propiedades de la multiplicación.
            \[(a,a'),(b,b'),(c,c')\in R\times R'\]
            \[\therefore ((a,a')(b,b'))(c,c')=(ab,a'b')(c,c')\]
            \[((a,a')(b,b'))(c,c')=(abc,a'b'c')\]
            \[((a,a')(b,b'))(c,c')=(a,a')(bc,b'c')\]
            \[\implies ((a,a')(b,b'))(c,c')=(a,a')((b,b')(c,c'))\]
            Por lo que es asociativa.
            \[(1_R,1_{R'})\in R\times R'\]
            \[(1_R,1_{R'})(a,a')=(1_Ra,1_{R'}a')=(a,a')\]
            \[(a,a')(1_R,1_{R'})=(a1_R,a'1_{R'})=(a,a')\]
            \[\implies (1_R,1_{R'})=1_{R\times R'}\]
            Por lo tiene una identidad. Ahora, veamos la distributividad.
            \[(a,a'),(b,b'),(c,c')\in R\times R'\]
            \[\left((a,a')+(b,b')\right)(c,c')=(a+b,a'+b')(c,c')\]
            \[\left((a,a')+(b,b')\right)(c,c')=(ac+bc,a'c'+b'c')\]
            \[\left((a,a')+(b,b')\right)(c,c')=(ac,a'c')+(bc,b'c')\]
            Similarmente:
            \[(c,c')\left((a,a')+(b,b')\right)=(ca,c'a')+(cb,c'b')\]
            Por lo que se cumple la distributividad. Lo que concluye esta demostración.
        \end{proof}

        \section{2. Construcción formal de los enteros y los polinomios}
        \subsection{7}
        Prove that the units of the polynomial ring $\set{R}[x]$ are the nonzero constant polynomials.
        \begin{proof}
            Recordemos que una unidad es un elemento $a$ de un anillo $R$, tal que existe $b$ que cumple $ab=ba=1_R$. Para el caso de $\set{R}[x]$ estos elementos serian:
            \[a,b\in\set{R}[x]:ab=ba=1_{\set{R}[x]}\]
            Sabemos que para todo elemento en $\set{R}[x]$ su grado es mayor igual a 0, y específicamente las constantes son de grado 0.
            \[\implies gr(a)+gr(b)=0\]
            Por lo mencionado anteriormente
            \[gr(a)=gr(b)=0\]
            Luego notamos que:
            \[\{a\in\set{R}[x]:gr(a)=o\}=\set{R}\]
            Sabemos que todos los números reales, excepto el cero, tienen inverso. Por lo que los únicas unidades de $\set{R}[x]$ son las constantes no cero.
        \end{proof}

        \section{3. Homomorfismos e Ideales}
        \subsection{1}
        Show that the inverse of a ring isomorphism $\varphi:R\rightarrow R'$ is an isomorphism.
        \begin{proof}
            Sean $\varphi(a)=a',\varphi(b)=b'$ donde $a,b\in R$
            \[a'+b'=\varphi(a)+\varphi(b)=\varphi(a+b)\]
            \[\therefore \varphi^{-1}(a'+b')=a+b=\varphi^{-1}(a')+\varphi^{-1}(b')\]
            También:
            \[a'\cdot b'=\varphi(a)\cdot\varphi(b)=\varphi(a\cdot b)\]
            \[\therefore \varphi^{-1}(a'\cdot b')=a\cdot b=\varphi^{-1}(a')\cdot\varphi^{-1}(b')\]
            Y por último:
            \[\varphi(1_R)=1_{R'}\implies \varphi^{-1}(1_{R'})=1_R\]
            Por lo que $\varphi^{-1}$ es un isomorfismo.
        \end{proof}

        \subsection{7}
        Prove that every nonzero ideal in the ring of Gauss integers contains a nonzero integer.
        \begin{proof}
            Sea $n=a+bi$ con $a,b\in\set{Z}$ y $n\in I$
            \[\therefore \bar{n}=a-bi\]
            Entonces tomamos lo siguiente:
            \[\bar{n}n=(a-bi)(a+bi)=a^2+b^2\in\set{Z}\wedge a^2+b^2\in I\]
            Por lo que todo ideal no cero contiene un entero no cero.
        \end{proof}

        \subsection{10}
        Describe the kernel of the homomorphism $\varphi:\set{C}[x,y,z]\rightarrow\set{C}[t]$ defined by $\varphi(x)=t,\varphi(y)=t^2,\varphi(z)=t^3$.
        \begin{proof}
            \[\ker\varphi=\{\alpha\in\set{C}[x,y,z]:\varphi(\alpha)=0\}\]
            Luego
            \[\varphi(x^2-y)=\varphi(x)^2-\varphi(y)=t^2-t^2=0\]
            \[\varphi(x^3-z)=\varphi(x)^3-\varphi(z)=t^3-t^3=0\]
            \[\varphi(y^3-z^2)=\varphi(y)^3-\varphi(z)^2=t^6-t^6=0\]
            Mas generalmente, $\forall a,b,c\in\set{C}[x,y,z]$ y $n,k,m\in\set{N}$
            \[\varphi(a(x^{2n}-y^n)+b(x^{3k}-z^k)+c(y^{3m}-z^{2m}))=0\]
            Ya que:
            \[\varphi(a(x^{2n}-y^n)+b(x^{3k}-z^k)+c(y^{3m}-z^{2m}))=\varphi(a)(\varphi(x^2)^n-\varphi(y)^n)+\varphi(b)(\varphi(x^3)^k-\varphi(z)^k)+\varphi(c)(\varphi(y^3)^m-\varphi(z^2)^m)\]
            \[\varphi(a(x^{2n}-y^n)+b(x^{3k}-z^k)+c(y^{3m}-z^{2m}))=\varphi(a)(t^{2n}-t^{2n})+\varphi(b)(t^{3k}-t^{3k})+\varphi(t^{6m}-t^{6m})\]
            \[\implies\varphi(a(x^{2n}-y^n)+b(x^{3k}-z^k)+c(y^{3m}-z^{2m}))=0\]
            Por lo que el kernel de $\varphi$ son los elementos de la siguiente forma:
            \[a(x^{2n}-y^n)+b(x^{3k}-z^k)+c(y^{3m}-z^{2m})\quad a,b,c\in\set{C}[x,y,z]\quad k,m,n\in\set{N}\qedhere\]
        \end{proof}

        \subsection{19}
        Let $R$ be a ring of characteristic $p$. Prove that the map $R\rightarrow R$ defined by $x\mapsto x^p$ is a ring homomorphism. This map is called the \textit{Frobenius homomorphism.}
        \begin{proof}
            Denotaremos $\varphi$ a la función.
            \[\varphi(1_R)=1_R^p=1_R\]
            \[\varphi(ab)=(ab)^p=a^pb^p=\varphi(a)\varphi(b)\]
            \[\varphi(a+b)=(a+b)^p=\sum^p_{n=0}\binom{p}{n}a^nb^{p-n}\]
            Notamos que $p$ primo $\implies \forall 1<n<p: n\nmid p$. Por lo que
            \[(a+b)^p=a^p+b^p=\varphi(a)+\varphi(b)\]
            Por lo que es un homomorfismo.
        \end{proof}

        \subsection{23}
        Let $R$ be a ring of characteristic $p$. Prove that if $a$ is nilpotent then $1+a$ is \textit{unipotent}, that is, some power of $1+a$ is equal to 1.
        \begin{proof}
            Sea $a^k=0$, tomamos $N$ tal que es múltiplo de $(k-1)!p$.
            \[\implies (1+a)^N=\sum^N_{i=0}\binom{N}{i}a^i=\sum^{k-1}_{i=0}\binom{N}{i}a^i\]
            \[\therefore \binom{N}{i}=\frac{N!}{(N-i)!i!}=\frac{N(N-1)(N-2)...(N-i-1)}{i!}\]
            Sabemos que $p\mid N$, y que $\forall i<k-1:i!\mid (k-1)!$
            \[\implies(1+a)^N=1\]
        \end{proof}

        \subsection{33}
        Prove or disprove. If $a^2=a$ for all $a$ in a ring $R$, then $R$ has characteristic 2
        \begin{proof}
            Sea $a\in R\setminus\{0\}$ (Se asume que $R$ tiene mas de un elemento), y se toma el homomorfismo de $\set{Z}$ a $R$.
            \[\therefore a^2=a\]
            \[a^2-a=0\]
            \[a(a-1)=0\]
            \[\varphi(n(n-1))=0\]
            Tomamos $n-1$
            \[\varphi((n-1)(n-2))=0\]
            \[\therefore \varphi(n^2-n)-\varphi(n^2-3n+2)=0\]
            \[\varphi(2n-2)=0\]
            \[\varphi(2)\varphi(n-1)=0\]
            Tomamos $n=2$
            \[\varphi(2)\varphi(1)=0\implies\varphi(2)=0\]
            Esto implica que $R$ tiene característica 2.
        \end{proof}

        \section{4. Anillos cocientes y relaciones en Anillos}
        \subsection{3}
        Describe each of the following rings
        \begin{enumerate}[label=\textbf{(\alph*)}]
            \item $\set{Z}[x]/(x^2-3,2x+4)$

            \item $\set{Z}[i]/(2+i)$
        \end{enumerate}
        \begin{enumerate}[label=\textbf{(\alph*)}]
            \item \begin{proof}
                Queremos probar que $\set{Z}[x]/(x^2-3,2x+4)\simeq\set{F}_2[\sqrt{3}]$. Para esto vamos a operar dentro del anillo:
                \[2x+4\equiv 0\]
                \[2(x+2)\equiv 0\]
                \[2(x+2)(x-2)\equiv 0\]
                \[2(x^2-4)\equiv 0\]
                \[\therefore -2\equiv 0\]
                Lo que es equivalente a: $2\equiv 0$. También vemos lo siguiente:
                \[x^2-3\equiv0\]
                \[\therefore x\equiv\sqrt{3}\]
                Por lo visto
                \[\implies \set{Z}[x]/(x^2-3,2x+4)\simeq \set{F}_2[\sqrt{3}]\]
                Que es lo que queríamos.
            \end{proof}

            \item \begin{proof}
                Queremos probar que $\set{Z}[i]/(2+i)\simeq\set{Z}/5\set{Z}$, para esto usaremos el primer teorema de isomorfismo.
                \[\varphi:\set{Z}\rightarrow \bar{R}\]
                Queremos que:
                \[\ker\varphi=5\set{Z}\]
                Y que:
                \[\Ima\varphi=\bar{R}\]
                Primero notamos que en $\bar{R}$:
                \[-2=i\]
                Por lo que el resto de $a+bi$ es el mismo que el de $a-2b$. Lo que implica que $\Ima\varphi=\bar{R}$.\\
                Tomemos un elemento $n\in\ker\varphi$.
                \[n=(a+bi)(2+i)\]
                \[n=(2a-b)+i(2b+a)\]
                Como $n$ es un entero $-2b=a$
                \[n=-5b\]
                O sea:
                \[n\in5\set{Z}\]
                Ahora notamos que $5=(2+i)(2-i)$, por lo que $5\in\ker\varphi$
                \[\implies \ker\varphi=5\set{Z}\]
                Por lo que $\set{Z}[i]/(2+i)\simeq\set{Z}/5\set{Z}$.
            \end{proof}
        \end{enumerate}

        \subsection{7}
        Let $I,J$ be ideals of a ring $R$ such that $I+J=R$
        \begin{enumerate}[label=\textbf{(\alph*)}]
            \item Prove that $IJ=I\cap J$.

            \item Prove the \textit{Chinese Remainder Theorem:} for any pair $a,b$ of elements of $R$, there is an element $x$ such that $x\equiv a\quad(\textrm{modulo }I)$ and $x\equiv b\quad(\textrm{modulo }J)$. [The notation $x\equiv a\quad(\textrm{modulo }I)$ means that $x-a\in I$]
        \end{enumerate}
        \begin{enumerate}[label=\textbf{(\alph*)}]
            \item \begin{proof}
                \underline{$\subseteq|$}
                \[a\in IJ\implies a=rsn=srn\quad r\in I,s\in J,n\in R\]
                \[\therefore a\in I\cap J\]
                \underline{$\supseteq|$} Para esto necesitamos ver que la suma de elementos en $IJ$ esta en $IJ$.
                \[n\in IJ\iff n=\sum_i \lambda_iI_iJ_i\quad I_i\in I,J_i\in J\]
                Luego sean $a,b\in IJ$
                \[a=\sum_i\lambda_iI_iJ_i\]
                \[b=\sum_j\lambda'_jI_jJ_j\]
                \[a+b=\sum_i\lambda_iI_iJ_i+\sum_j\lambda'_jI_jJ_j=\sum_k\lambda''_kI_kJ_k\]
                \[\implies a+b\in IJ\]
                Ahora, sea $x\in I\cap J$ y sea $1_R=r+s$ tal que $r\in I, s\in J$
                \[\therefore x=x\cdot 1_R=x\cdot (r+s)=xr+xs\]
                \[xr,xs\in IJ\]
                \[\implies xr+xs\in IJ\implies x\in IJ\]
                Por lo que $IJ=I\cap J$
            \end{proof}

            \item \begin{proof}
                Sean $r+s=1_R$ y $r\in I, s\in J$, definimos $x$ de la siguiente manera:
                \[x=ar+bs\]
                \[\therefore x=ar+b-br\]
                \[x-b=(a-b)r\in I\]
                Similarmente:
                \[x-a=(b-a)s\in J\]
                \[\implies x\equiv b\mod I\wedge x\equiv a\mod J\]
                Por lo que existe un $x$ que cumple lo pedidos
            \end{proof}
        \end{enumerate}
    \end{document}